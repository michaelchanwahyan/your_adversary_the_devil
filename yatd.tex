\documentclass{book}
\usepackage[letterpaper, portrait, margin=1cm]{geometry}
% ---------------------
% mini-table-of-content
% ---------------------
\usepackage{minitoc}
\setcounter{minitocdepth}{1}
\setlength{\mtcindent}{24pt}
%\renewcommand{\mtcfont}{\small\rm}
%\renewcommand{\mtcSfont}{\small\bf}
%\usepackage{setspace}
%\usepackage{tocloft}
%\setlength\cftparskip{-1.2pt}
%\setlength\cftbeforesecskip{1.3pt}
%\setlength\cftaftertoctitleskip{2pt}
%\renewcommand{\cftsecafterpnum}{\hspace*{02.0em}}
%\renewcommand{\cftsubsecafterpnum}{\hspace*{02.0em}}

% ---------------------------
% Chinese Characters Packages
% ---------------------------
\usepackage{fontspec} 
\usepackage{xeCJK}
\setmainfont{Times}
\setCJKmainfont{BiauKai}

\usepackage{ifpdf,cite,algorithmic,url,tikz}
\usepackage[cmex10]{amsmath}

% -------
% General
% -------
\usepackage{multicol}
\usepackage{multirow}
\usepackage{color,colortbl}
\usepackage{xparse}
\usepackage{pbox}
\usepackage{stackengine}
\usepackage{titlesec}% http://ctan.org/pkg/titlesec
\usepackage{tabularx}
\usepackage{titlesec}
\newcommand{\sectionbreak}{\clearpage}

\author{
    Editor, Michael Chan\\
        \texttt{chchan@link.cuhk.edu.hk}
}
\usepackage{tocloft}

\usepackage{hyperref}
\hypersetup{
    colorlinks=true, % set true if you want colored links
        linktoc   =all , % set to all if you want both sections and subsections linked
        linkcolor =blue, % choose some color if you want links to stand out
}

\begin{document}

\clearpage
%% temporary titles
% command to provide stretchy vertical space in proportion
\newcommand\nbvspace[1][3]{\vspace*{\stretch{#1}}}
% allow some slack to avoid under/overfull boxes
\newcommand\nbstretchyspace{\spaceskip0.5em plus 0.25em minus 0.25em}
% To improve spacing on titlepages
\newcommand{\nbtitlestretch}{\spaceskip0.6em}
\pagestyle{empty}
\begin{center}
\bfseries
\nbvspace[1]
\Huge
{%\nbtitlestretch
    \huge
        \textbf{認清你的仇敵 -- 魔鬼\\Your Adversary The Devil}}

        \nbvspace[1]

{\large
    Dwight Pentecost
}

\nbvspace[1]

% {\large
    % Revision: \texttt{v0.1}\\
        % Last Update: \today
        % }


        % \vfill
        % \begin{tikzpicture}
        %     \node (0,0) [opacity=0.03]{\includegraphics[width=15cm]{christ_on_cross.png}} ;
        % \end{tikzpicture}
        % \vfill

        \end{center}

        \newpage

        \setcounter{tocdepth}{0}
        \dominitoc
        %\begin{multicols}{3}
        %\begin{multicols}{1}
        \Large
        \addtocontents{toc}{\protect\hypertarget{toc}{}}
        \tableofcontents
        %\end{multicols}

        \Large
        %\twocolumn

        \definecolor{pink}{rgb}{0.68,0,0.68}
        \chapter*{序言}
        \label{subsec:prolog}
         「知己知彼,百戰不殆。」在戰場上,指揮員只有在深入了解敵人的基礎上才能有可能取得勝利。如果他只準備好從陸地上發動進攻卻忽略了敵人從空中或是從海上的襲擊,那麼他無疑是在走向失敗;如果他只考慮從陸上和海上攻擊敵人,卻忘了敵人或許會從空中來打擊他,那麼整場戰役也會面臨全軍覆沒的危險。

         同樣,除非知道那毀滅我們靈魂的敵人是誰,明白他的處事原則,他做事情的方法,迷惑的方式,否則沒有人能勝得過他。今天,我們很少聽到關於撒旦的事情,結果雖然很多認識到撒旦的存在以及撒旦就是那毀壞他們靈魂的人,卻都沒有做好充份準備去迎接撒旦的挑戰。我們忘記了這位敲打著我們心門的人的本性。我們不了解聖經上是如何關於他的本性,他的方式、方法、計劃等等方面的教導。結果,在同他的爭戰中我們就失利了。

         如果醫生已經給他的病人診斷出患有肺癌,卻只開了點藥膏治他小腳指頭上的雞眼,這個醫生是不是很愚蠢?跟醫生必須對症下藥的道理一樣,當我們接受耶穌基督為我們的救主,要過得勝生活的時候,我們必須要明白這本厚厚的聖經裡是如何描述我們的敵人以及他的工作的。我們希望通過查經,從中詳細地了解魔鬼的本性,他的詭計、陰謀、原則等,這樣我們就可以在每日的生活中發現他,我們也就有了得勝的可能。不過,獲勝還要依靠知識。我們相信那位得勝者會用這本書來幫助我們得勝。

         特別感謝Nancy Miller小姐和Reba Allen夫人為主的緣故幫忙整理手稿。這本書的出版離不開她們所付出的辛勤汗水。願主賜給她們極大的喜樂,同時也願主使用這本書,讓他得勝的道理能夠廣傳。

         Dwight Pentecost牧師

         于德州達拉斯

        \chapter{撒旦的墮落}
        \label{sec:ch01}
        \hyperlink{toc}{[返主目錄]}
        \hyperref[sec:ch02]{[下一章]}

        \begin{center}
        \noindent\fbox{%
            \parbox{0.8\textwidth}{%
                以西結書 28:11-27
                    \newline
                    28.11 耶和華的話臨到我說、
                    28.12 人子阿、你為推羅王作起哀歌、說、主耶和華如此說、你無所不備、智慧充足、全然美麗。
                    28.13 你曾在伊甸 神的園中、佩戴各樣寶石、就是紅寶石、紅璧璽、金鋼石、水蒼玉、紅瑪瑙、碧玉、藍寶石、綠寶石、紅玉、和黃金、又有精美的鼓笛在你那裡.都是在你受造之日預備齊全的。
                    28.14 你是那受膏遮掩約櫃的基路伯.我將你安置在 神的聖山上.你在發光如火的寶石中間往來。
                    28.15 你從受造之日所行的都完全.後來在你中間又察出不義。
                    28.16 因你貿易很多、就被強暴的事充滿、以致犯罪、所以我因你褻瀆聖地、就從 神的山驅逐你.遮掩約櫃的基路伯阿、我已將你從發光如火的寶石中除滅。
                    28.17 你因美麗心中高傲、又因榮光敗壞智慧、我已將你摔倒在地.使你倒在君王面前、好叫他們目睹眼見。
                    28.18 你因罪孽眾多、貿易不公、就褻瀆你那裡的聖所.故此、我使火從你中間發出、燒滅你、使你在所有觀看的人眼前、變為地上的爐灰。
                    28.19 各國民中、凡認識你的、都必為你驚奇.你令人驚恐、不再存留於世、直到永遠。
                    28.20 耶和華的話臨到我說、
                    28.21 人子阿、你要向西頓預言攻擊他、
                    28.22 說、主耶和華如此說、西頓哪、我與你為敵、我必在你中間得榮耀、我在你中間施行審判、顯為聖的時候、人就知道我是耶和華。
                    28.23 我必使瘟疫進入西頓、使血流在他街上.被殺的必在其中仆倒、四圍有刀劍臨到他、人就知道我是耶和華。
                    28.24 四圍恨惡以色列家的人、必不再向他們作刺人的荊棘、傷人的蒺藜、人就知道我是主耶和華。
                    28.25 主耶和華如此說、我將分散在萬民中的以色列家招聚回來、向他們在列邦人眼前顯為聖的時候、他們就在我賜給我僕人雅各之地、仍然居住。
                    28.26 他們要在這地上安然居住.我向四圍恨惡他們的眾人施行審判以後、他們要蓋造房屋栽種葡萄園、安然居住、就知道我是耶和華他們的 神。
            }%
        }
\end{center}

 撒旦從何而來?是神創造了魔鬼嗎?神和魔鬼之間是什麼關係?當信徒在神的聖潔的光照下從聖經中明白存在魔鬼的時候,這樣的問題通常會讓人感到很頭痛。哲學無法對這些問題作出令人滿意的回答。只有從神的話語裡,我們才能找到他給我們的答案,也只有這個答案才能真正讓人滿意。

 在以西結書25-32章裡,先知向世人宣佈了對以色列敵人的審判。他說神聖潔的審判要臨到那些逼迫以色列的國家。在28章1-10節裡,他提到了對推羅的審判。推羅位於聖經裡所講到的敘利亞的北部,由腓尼基人所盤踞,是以色列主要的敵人之一。 但在11-17節裡,先知所談到的「推羅王」指的不再是國家的君主,而是對在背後控制這個國家君主的人(原文此處用「推羅的王子」來表示推羅的國君)的審判。這個在背後進行操縱的人被稱為推羅王。我們從中應該注意到撒旦通過人來做他的工作。在許多情況下,他會借助政府的掌權者來達成他的目的。撒旦想剿滅以色列,這樣從神而來的彌賽亞就不能來到世上並通過以色列來祝福整個世界,所以撒旦才煽動加利利各國起來攻打以色列。那些加利利人在千方百計逼迫和屠殺以色列人的時候並不知道他們這一切都在撒旦的計劃裡,他們絲毫不知道中了撒旦的詭計。因此,先知才在1-10節裡對以色列的仇敵宣判之後,又講到對那操縱這些加利利國君們的撒旦的審判。

 我們稱之為撒旦的起先叫做路西非爾(Lucifer),是「明亮之星」或「早晨之子」的意思。從以西結書28:11-13裡,我們可以發現他的名字和他是多麼貼切。先知這樣對他宣判說:「耶和華的話臨到我說:『人子啊!你為推羅王(指撒旦)作起哀歌說:主耶和華如此說,你無所不備,智慧充足,全然美麗。」12節詳細為我們描述了路西非爾在墮落前的美麗。

 路西非爾是被創造出的。15節指出了這一點。「你從受造之日所行的都完全,後來又在你中間察出不義。」只有神自己是永恆的。只有神自己擁有永恆的生命,也就是說只有神不是被創造的。其他所有的生物之所以活著,都是因為神創造了他們。一切受創造物的生命和神的生命都是不同的,他們有的都是那種受創造的生命。神創造的工作是由創造一大群天使開始的,而其中之一就是路西非爾。作為被創造物,他就應該敬拜、事奉、聽從他的主人。創造撒旦的本意並不是要他在以後起來背叛,成為神的大仇敵。15節證實了這一點,「你(撒旦)從受造之日所行的都完全。」

 他不僅所行的都完全,12節裡還說他智慧充足,全然美麗。路西非爾在神所創造的眾天使中是最有智慧的。神讓撒旦做天使長,管理天使的事情。儘管所有的權力都來自神的寶座,神還是交給路西非爾一些管理的權力。神創造他就是為了讓他承擔一些責任。

 神向我們揭示了他設計、創造天使的意圖。在以弗所書1:21節裡我們會發現在他們有不同的等級,各有各的職責(中文和合本聖經此處是「各房」,原文這裡指列舉了「principalities, powers, mights and dominions」等各職份)。

 有些天使擔負著守護的職能。比如在希伯來書1:14裡,作者對我們說天使是服役的靈,也就是說他們是僕人,奉命保護那些已經得到救恩的人。撒旦會很樂意去阻止人們接受主耶穌,不讓人們進入天國;但由於天使的工作,他無法做到這一點。在詩篇91:11裡,詩人說神要為你吩咐他的使者,在你行的一切道上保護你。這很讓我感動,我知道有些天使在等著我的出生,他們一直在守護著我,看著我接受了耶穌基督為我的救主。當我上這條信主的道路的時候,我滿心地感謝聖經裡這段對我的教導。我們因而也就會明白那些天使都是神所創造出來的,是特地來保護那些已經得到救恩的人的。

 神也會通過一些天使來行神跡。使徒行傳5:19裡記載了主的使者在夜裡為眾使徒打開了監獄的大門。使徒行傳12:7,8裡又有相同的事情發生。這些都是神解救了他們,而神是通過天使來完成這個神跡的。

 在啟示錄16:1裡我們還會發現有的天使有審判的職責。「我聽見有大聲音從殿中出來,向那七位天使說:『你們去,把盛神大怒的七碗倒在地上。』」我們讀過啟示錄後就會發現, 最後時刻的審判是通過天使來進行的。這使我們想到了以色列的歷史,為了讓以色列人從埃及人的奴役下解脫出來,他審判了埃及人,而這也是通過天使滅殺了埃及人所有的長子。所以說天使也有行使審判的職責。

 在希伯來書2:2裡,我們還發現有些天使擔負著溝通的職責,他們是神的真理向世人顯明的管道。他在這裡說,「那藉著天使所傳的話既是確定的,凡干悖逆的,都受了該受的報應。我們若忽略了這麼大的救恩,怎能逃罪呢?…」在這裡,他可能是想到了通過眾天使在西乃山上向摩西頒佈律法時的情景。所以說天使還擔當著這項使命。

 你也許會發現,先前所提到的都是對人的。其實,還有些天使擔負的使命是和神有關的。在以賽亞書6:1裡,以賽亞說,「當烏西雅王崩的那年,我見主坐在高高的寶座上。他的衣裳垂下,遮滿聖殿。其上有撒拉弗侍立…」撒拉弗是某個等級的天使,他們的職責是做神的護衛。當時這些撒拉弗圍繞在神的寶座前,彼此呼喊著「聖哉!聖哉!聖哉!萬軍之耶和華,他的榮光充滿全地!」這些是敬拜的天使,防止神的寶座受到不潔的事物攻擊。

 在以西結書第一章裡,我們還可以發現先知以西結在第五節裡提到了「四個活物」。從13節裡我們可以知道他們「就如燒著火炭的形狀,又如火把的形狀。火在四活物中間上去下來,這火有光輝,從火中發出閃電。這活物往來奔走,好像電光一閃。」先知用「燒著」、「光輝」、「閃電」這樣的詞語來形容這些天使。以賽亞書6:2裡撒拉弗一詞的本意是燃燒。以西結書一章裡描述了這些天使發出光芒的樣子。

 在以西結書10:1裡這些被稱作「活物」的被叫做基路伯:「我觀看,見基路伯頭上的穹蒼之中,顯出藍寶石的形狀,仿彿寶座的形像。」3節裡說,「那人進去的時候,基路伯 站在殿的右邊,雲彩充滿了內院。耶和華的榮耀從基路伯那裡上昇…」這裡先知所說的基路伯是另一類別或等級的天使,他們的職責和寶座前的撒拉弗是不同的。

 聖經裡屢次提到了基路伯。創世記3:24節裡記著,在亞當和夏娃犯罪之後,神把他們從伊甸園裡趕出,並在入口安設了四面轉動發火焰的劍。另一處聽到基路伯 的地方在出埃及記25:18,摩西被吩咐去做一個約櫃,約櫃上還要有一個施恩座作為約櫃的蓋子,施恩座的兩頭還各有一個基路伯。在啟示錄的4章8、9節裡我們還能找到關於這些活物,也就是基路伯的描述。約翰說這些活物「晝夜不住地說:『聖哉!聖哉!聖哉!主神是昔在、今在、以後永在的全能者』…四活物將榮耀、尊貴、感謝歸給那坐在寶座上、活到永永遠遠者…」啟示錄4章裡提到的這些活物都在敬拜神。當撒拉弗喊著「聖哉!聖哉!聖哉!全能的神」的時候,他們是在從寶座上向下巡視,防止外來不潔淨的東西攻擊寶座。而當基路伯圍繞著寶座的時候,他們是在仰望寶座,稱頌那坐在寶座上的是「聖哉!聖哉!聖哉!主神是昔在、今在、以後永在的全能者。」創世記3章裡說基路伯守衛著伊甸園的人口,保護著神的聖潔。基路伯坐在約櫃的蓋子上,在施恩座的兩旁,宣告了血的祭將要滿足神的聖潔。在啟示錄裡,基路伯一直在讚美神是因為基督戰勝撒旦,表明瞭神的聖潔。

 當我們返回到以西結書28:14節時,我們會發現路西非爾也曾是個受過膏抹的基路伯。所以,從前面的討論中,你就可以知道路西非爾在他被造的時候位置很受尊崇。他並不是一般等級的普通天使。他可以仰望主的寶座,並因為神的聖潔而放聲稱頌、感謝、讚美神。現在如果讓我們重新再給天使們歸類,我們可以這樣說那可以仰望神、守護寶座的基路伯是位置最高的,在所有被創造的事物中,他們的特權最大。這都是神給了像路西非爾那樣的基路伯這些非凡的特權。

 撒旦不僅在所有被創造的事物中智慧最高,而且也最美麗。在以西結書28:13裡,先知向我們描述了他的美麗。先知把撒旦的美麗同各色名貴的寶石相比。他說,「佩帶各樣寶石,就是紅寶石、紅璧璽、金剛石、水蒼玉、紅瑪瑙、碧玉、藍寶石、綠寶石、紅玉…」真可謂是五彩繽紛了!不過,寶石自己是不會發光的。如果你把寶石拿到暗房裡,那麼它就不再發光,不再閃亮了。寶石的美麗並不在於它的本身,而是在於它能夠把普通的光線折射出漂亮的光芒!當神創造路西非爾的時候,他使得路西非爾有能力折射出神的榮耀,而且使他的這種本領比其他被造的事物都強。這等級最高的天使所表現出來的美麗不是他與生俱來的,而是在他被造的時候神給他的。他只不過是把這種美麗表現了出來。神才是那那真正讓路西非爾顯得美麗無比的光芒,任何被造之物都無法將神的榮耀完全表現出來。

 因為要讚美稱頌神,所以才有了樂器。路西非爾還有精美的鼓笛,因而先知說「又有精美的鼓笛在你那裡,都是在你受造之日預備齊全的。」因為路西非爾的美麗,他可以像樂師手中的樂器一樣為神的榮耀獻上贊歌。 他不必找人來吹奏樂器才能獻上讚美的詩篇,他本身就是一首讚美詩。通過路西非爾對神的讚美、榮耀和敬拜,我們從他身上可以看到神無比的美麗,再也沒有其他的受造之物可以如此顯現出神的榮耀。

 那麼受造之物的責任是什麼呢?那就是聽命于那創造他的人。受造之物必須要知道他是從神的手中被造出來的,創造主的地位要比被造之物的地位高。但當我們在讀以西結書28:16,17的時候,我們可以看到路西非爾拋開了受造之物的位置,篡奪了造物主的位置。「你因美麗心中高傲,又因榮光敗壞智慧」。神創造了這樣美麗而又充滿榮光的路西非爾,並以此顯明瞭他那超乎萬有的能力,但路西非爾卻不把創造他的神尊崇為主。路西非爾濫用了神給他的智慧,他說「像我這樣有智慧的應當做神;像我這般美麗的應該受到別人的敬拜,而不該去敬拜別人。」神對他豐富的賜予在他身上卻成了陷阱,他拋棄了自己應有的位置,背叛了神。創造他本來是要彰顯神的榮耀,但他卻把這些榮耀都歸給了自己。在當初創造路西非爾的時候,神是否知道驕傲可能會使路西非爾墮落?他當然知道,因為身神是無所不知的神。那神無法阻止這樣的事情發生嗎?神當然能,因為神是無所不能的神。那他為什麼沒有去阻止呢?沒有人知道這個答案。神就這樣進入了與這空中掌權者的紛爭,通過戰勝那數之不盡的罪惡,神可以向所有受造之物顯明他是榮耀的神,聖潔的神,滿有能力的神,是值得尊崇與敬拜的神。

 幾年前當我在費城地區做牧師的時候,有個從中西部來的人參加了我們的聚會,他當時在費城龐大的John Wanamaker的名貴珠寶部工作。在我鼓勵他參加養牧工作的中間,我們談到了他的工作以及他曾見過以及經手過的珠寶。一天,當我經過那家商店的時候,他招呼我進去並對我說:「我想讓你看一下我們剛進來的一顆鑽石。」他走進保險櫃,拎過來一個小鹿皮包,接著對我說,「伸出你的手來。」他打開裡面的小抽屜,把一塊石頭放進了我的手裡,然後問道:「你以前摸過價值50萬美圓的鑽石嗎?」我答道:「我不太常碰到這種事!」他放在我手心裡的是一顆價值50萬美圓的鑽石。我感到背後傳來一陣陣的寒意。可當我仔細觀察這塊大寶石的時候,我失望極了,它還不如我妻子手指上戴的那顆小寶石漂亮,一絲生氣也沒有。他很顯然看出了我在想什麼,於是他笑著對我說:「還是給我吧。」他伸手從下面拿出一塊黑色的絲絨,然後把鑽石放到了絲絨上面。霎時間,那顆鑽石有了生氣,發射出奪目的光輝。他向我解釋說,當我手裡拿著他的時候,它折射出的是我的皮膚的顏色,所以顯得沒有生氣。可把它放到黑色背景下的時候,它就折射出光來,所以我們就能看到它的美麗之處了。同樣,當神彰顯出他的聖潔與完美的時候,他以那些黑色的罪為背景,耶穌基督來到世上救贖了所有的罪,在他的聖潔與人的罪惡的鮮明對比下,神無比的榮耀和聖潔就被顯明瞭。

 我想沒有人會知道神到底為什麼會允許撒旦墮落。不過聖經上記載說神創造的這位最智慧、最美麗的天使不仰望神,而轉眼仰望自己。他沒有認清楚他自己到底是什麼,不明白他身上所有豐富的恩典都出自神的手中,不知道順服神應該是他的本份。在背離神的時候,他以自己為中心,私慾在他心中膨脹開來。從亞當犯罪的那一刻起,每個降生到這個世上的人都有著同他的魔鬼父親一樣的天性。這決定了罪人都是自私自利,以自我為中心的。每個人都充滿了驕傲,他們遠離神、背棄神,唯一不變的就是他們的魔鬼父親在他們身上留下的品性。除非你了解一些使撒旦墮落的自私、驕傲、以自我為中心的知識,否則你永遠不會明白你自己,你也不會看清楚每天來到這世上的誘惑。

 今天,可能還有人在走路西非爾的老路。或許他正在為自己所受過的教育、為自己的才智、成就而驕傲,卻沒有意識到所有這些都是來自神的恩賜。他可能在為自己豐富的物質享受而驕傲,為自己在某些領域裡受尊崇的地位而驕傲,卻沒有看清楚這些也都是出自神慷慨的賜予。當一個人把自己同神分開的時候,那麼他就是在重複著路西非爾的老路,走在罪的裡面。人也能從這條罪的路上脫離開來,唯一的辦法就是要接受耶穌基督為他個人的救助。這樣,他就獲得了重生,有了新的品性,他的自私也能因為關心他人而被替換掉。他也脫離了驕傲,得以重新看見得到新生的自己與神之間的關係,他會看到自己的卑微,認識到應該依靠父神。願神帶領你看到你本是魔鬼父親的孩子。你不是小路西非爾,而是小魔鬼。這其中有很大的區別。神願意帶領你脫離魔鬼的家庭,進入他的國度。你願意接受他作為你個人的救主嗎?

\chapter{撒旦的罪}
\label{sec:ch02}
\hyperref[sec:ch01]{[上一章]}
\hyperlink{toc}{[返主目錄]}
\hyperref[sec:ch03]{[下一章]}

\begin{center}
\noindent\fbox{%
    \parbox{0.8\textwidth}{%
        以賽亞書14:12-17
            \newline
            14.12 明亮之星、早晨之子阿、你何竟從天墜落.你這攻敗列國的、何竟被砍倒在地上。
            14.13 你心裡曾說、我要升到天上.我要高舉我的寶座在 神眾星以上.我要坐在聚會的山上、在北方的極處、
            14.14 我要升到高雲之上.我要與至上者同等。
            14.15 然而你必墜落陰間、到坑中極深之處。
            14.16 凡看見你的、都要定睛看你、留意看你、說、使大地戰抖、使列國震動、
            14.17 使世界如同荒野、使城邑傾覆、不釋放被擄的人歸家、是這個人麼。
    }%
}
\end{center}

 神曾讓這最有智慧、最美麗的路西非爾去統領所有圍繞在神寶座週圍的那些基路伯。受造之物就應該服從創造主的旨意,尤其像路西非爾這樣有著非同尋常的特權,就更應該自從這條原則。可正是這些使得路西非爾遠離了其他眾天使,走向了墮落。我們在前面的以西結書28章裡曾讀到過,路西非爾因為自己的美麗、驕傲、智慧、權力而心存驕傲。如果神不給我們顯明,我們可能永遠都不會注意到是什麼心思意念促使了路西非爾起來背叛神。在以賽亞書14;12-14裡,神一步一步向我們展示了撒旦當時的想法。

 在這幾節裡,我們可以看到撒旦在心底裡五次宣稱「我要」,神的意願和路西非爾自己的意願之間開始有了衝突。神本意並不是要創造一個墮落的撒旦,一個背叛神、與所有良善為敵、與神作對的形像。神要創造的是順服的路西非爾,但路西非爾也被賦予了選擇的權力。當神向路西非爾顯明他的旨意的時候,路西非爾可能違背主的計劃和旨意的可能性也就暴露了出來。當他把自己的意志擺放在神的意志之上,連續五次說「我要…我要…我要…我要…我要…」的時候,罪就在他身上產生了。每一次他都是在違背神的意願,他用自己的意願和計劃取代了神的意願和計劃。這五次宣稱非常重要,因為它想我們顯明瞭撒旦的陰謀。他的目的始終都沒有改變過,他也從沒有回心轉意;他仍然在妄圖實現他的那五個願望。

 以賽亞書14 :13裡說,「我要升到天上,我要高舉我的寶座在神眾星之上;我要坐在聚會的山上,在北方的極處;我要升到高雲之上,我要與至上者同等。」讓我們仔細思考一下撒旦的這五個「我要」吧。

 他首先說:「我要升到天上。」聖經中的天是通常指三層不同的空間。我們平常所說的天是第一層天,也就是大氣層,鳥兒可以在這一層天裡翱翔。第二層是星際空間,裡面有各種星球。再往外是第三層天,那裡是神的居所,神掌權的寶座就在那裡,神從那裡降旨意給其他的那兩層天。

 路西非爾住在第二層天。可他想從星際空間升到神的居所。他並不是想像遊客那樣去看看神的寶座到底是什麼樣子,因為路西非爾以及其他那些被創造的天使都住在第二層天,他們能夠到第三層天,也能到達神的寶座。以賽亞書六章從1節開始說,「當烏西雅王崩的時候, 我見主坐在高高的寶座上。他的衣裳垂下,遮滿聖殿。其上有撒拉弗侍立,各有六個翅膀:用兩個翅膀遮臉,兩個翅膀遮腳,兩個翅膀飛翔。彼此呼喊說:『聖哉!聖哉!聖哉!萬軍之耶和華,他的榮光充滿全地!』」在先知以賽亞對神的榮耀以及他的寶座的異象裡,先知看到了撒拉弗。你可能回憶起我們在前面學習的以西結書28章14節裡曾這樣描述路西非爾說,「你是那受膏遮掩的基路伯」;13節裡所說的,「你曾在伊甸神的園中」;還有14節裡說的,「我將你安置在神的聖山上,你在發光如火的寶石中間往來」。被膏過的路西非爾那時就守衛在神的寶座前,也就是說他在神的居所,在第三層天上。

 所以說以賽亞在14章13節裡說,「你心裡曾說,『我要升到天上』,」並不是指路西非爾想要以基路伯的身份在神的寶座前恪守護衛的職責。這受神的允許守衛寶座的路西非爾想的是到那裡永遠像神一樣發號施令;是想來到神的面前與神同等;是想把神趕走。他本來是由神創造的,現在卻想坐在神的寶座上,篡奪神的位置。他的第一個願望就是要違背神的旨意佔據神的居所,因此他心裡說「我要升到天上」。

 第二個我要是「我要高舉我的寶座在神眾星以上」。從約伯記38章7節裡我們可以找到「神眾星」的含義。神沒有給星辰生命,它們都是些沒有生命的物體。他們不能夠彰顯神的榮耀,正如詩人所說,「諸天述說神的榮耀,穹蒼傳揚他的手段。」不過星辰並不願意服從神。那麼當撒旦在說「我要高舉我的寶座在神眾星以上」的時候,他的心裡是怎麼想的呢?從約伯記38章裡我們可以知道神讓約伯看到了他在所造之物身上所顯現的威嚴和能力。17節裡問道:「那時晨星一同歌唱,神的眾子也都歡呼。」這裡的晨星與神的眾子是等同的。「晨星」在這裡被稱為神的眾子,指的就是那些被創造的天使,當神向他們顯現出榮耀與權柄的時候,他們就獻上讚美的詩歌。所以根據約伯記38章,我們可以得到這樣的結論:當路西非爾說:「我要高舉我的寶座在神眾星以上」的時候,他實際上在說:「我要篡奪神掌管所有天使的王權。」

 從神的話語中我們可以知道,天使是被創造出來的,應該聽命與權柄高于他們的主,因為受造之物應該服從創造他們的人。神按照自己的意願,給了路西非爾管理眾天使的權力。但這只是賦予給他的代表的權力;權力的源頭還是神本身。儘管神給了他管理眾天使的權力,不過他自己還要聽從神的旨意;儘管他可以管理眾天使,但他自己也還要在神的管理之下。當路西非爾說:「我要高舉我的寶座在神眾星以上」的時候,他其實是在說:「我要成為眾天使唯一的主宰,我自己也不要聽命于造物主。」當眾天使抬頭仰望路西非爾的時候,他們是在尋求那從上面通過路西非爾而來的旨意。路西非爾說:「我要成為唯一的絕對的,我要給眾天使下達我的旨意,我要把神趕走。」他是想要眾天使把他當作神,篡奪那隻屬於神的權力。他不僅想要霸佔天國,還想要篡奪那隻有神才配有的權力。

 在他第三個「我要」裡,他說:「我要坐在聚會的山上,在北方的極處。」他的這句話表明瞭他想控制整個宇宙萬物的野心。讓我們一同來查攷一下舊約裡面談到「聚會的山」和「北方的極處」的幾個地方。在以賽亞書2:2裡有,「末後的日子,耶和華殿的山必堅立,超乎諸山,高舉過於萬嶺,萬民都要流歸這山。」請注意這裡面用到的「山和嶺」。它們指的是神所擁有和掌管的權柄。這些都與那要來到世上稱王掌權的彌賽亞的全能有關。當他再來的時候,他要建立起他的寶座。他要在他的國度裡稱王掌權,而他的國度也就是這裡所說的山,萬民到時都要歸于他。詩人在詩篇48:2裡這樣描述耶路撒冷說:「錫安山,大君王的城,在北面居高華美,為全地所喜悅。」這裡的「北面」指的是大衛王時代屬於耶路撒冷的權柄。耶路撒冷是當時的國都,是權柄的寶座;王就在這裡統治和管理他的國家裡面的一切事物。

 通過以賽亞書2章以及詩篇48章的光照,我們知道當路西非爾說「我要坐在聚會的山上,在北方的極處」的時候,他其實是在說「我要統治地球,我要管理所有受創之物中的一切事情。」所以那個說「我要佔領天堂」,「我要所有天使都聽命于我」的路西非爾還要讓宇宙中所有的受造之物都在他的控制之下,都臣服在他的權柄下。

 他還說道:「我要升到高雲之上」。回到出埃及記16章,我們可以看到以色列的子民在神的指引下都離開埃及,到了曠野。出埃及記16:10裡寫到:「亞倫正對以色列全會眾說話的時候,他們向曠野觀看,不料,耶和華的榮光在雲中顯現。」對以色列人來說,雲的出現是他們可以看到的神與他們同在的表現,所以神會繼續帶領他們,在曠野中為他們開路。在出埃及記40:34我們還可以讀到,當摩西建造完會幕的時候,有雲彩遮蓋住了會幕,神的榮光也就充滿了帳幕。這向以色列民族表明的是神佔據了會幕,並以會幕裡的雲彩的形式向眾人表明神與他們同在。列王記上8:10裡還說,所羅門在建造完聖殿後,神閱納並住進殿裡的標記就是神以雲彩的形像顯現。「祭祀從聖所裡出來的時候,有雲充滿耶和華的殿。」在新約馬太福音24:30裡,基督應許他會再次回來,他告訴我們有能力、有大榮耀的人子要駕著天上的雲降臨。無論是在舊約還是在新約裡,雲彩都是神向他的子民顯現,表明他要與他們同在的象征。

 當路西非爾說:「我要升到高雲之上」的時候,他是在說「我的榮耀要比神更大。」你或許還記得以西結曾經用閃亮的珠寶來形容路西非爾的美麗和榮耀。但他的這些榮耀並不是屬於他本身的,那隻是從別處而來的榮耀顯現在他的身上。而神才是這榮耀的源泉,他讓自己的榮耀從他創造的事物中表現了出來。路西非爾妄想佔據神的寶座,統治所有的天使和宇宙中的萬物,把造物主的榮耀加添到自己的身上,使自己比神更榮耀。這種想法多麼荒謬!這好像是在說神的榮耀是有限的,而路西非爾能帶來比神還多的榮耀;好像他把神的榮耀搶奪過來之後,再加上自己創造的榮耀,就能夠使他成為整個宇宙中主宰,統治所有的一切。

 接著,他最後又說道:「我要與至上者同等。」路西非爾自己也不得不承認他是受造之物這個事實。他所擁有的只是一種被創造的生命,而不是永遠的生命。他是有開始的。那麼像他這樣的一個受造之物怎麼可能會像造物主一樣呢?他怎麼可能與至上者同等呢?不錯,他在神所創造的萬物之中是最有智慧的,但他不是無所不知的;他並不了解所有的一切。他是神所創造的萬物中最有權柄的,但他不是無所不能的。他又怎麼可能與至上者同等?只有一個辦法,那就是完完全全地離棄在他之外的所有一切,那麼他就能像神那樣了,不過他所能統治的只有他自己一個。撒旦的夢想就是到天國佔據神的寶座,完全獨立地行使他對所有天使的權力,使得地球和宇宙萬物都在他的統治之下,奪走那本屬於神的榮耀,把它加添到自己的身上。

 是什麼促使他竟然會對權力和榮耀產生如此荒謬、令人難以理解的貪欲呢?我們可以從以西結書找到這樣一條線索。以西結書28:17這樣寫到,「你因美麗心中高傲(違背了神的旨意),又因榮光敗壞智慧,我已將你摔倒在地。使你倒在君王面前,好叫他們目睹眼見。你因罪孽眾多,貿易不公,就褻瀆你那裡的聖所。故此,我使火從你中間發出,燒滅你,使你在所有觀看人的眼前,變為地上的爐灰。」他所說的「我使火從你中間發出」是什麼意思?在以賽亞書6章裡,撒拉弗的意思是「燃燒、燒紅著的」。神曾說,「在我所造之物中,你是最明亮的。」有火從他的中間發出是因為他的榮耀、美麗和權力都在燃燒。正是給他身上的這些東西成了他心中熊熊燃燒的情慾。這情慾使他妄圖坐上神的寶座,統治眾天使以及宇宙的萬物,搶奪神的榮耀加添給他自己,脫離神,最終走上了背叛和永遠滅亡的道路。

 當基督要為拯救以色列民族而奉獻出自己的那時起,他就開始不斷向當時的那些宗教領袖們發出信息。他要他們首先要認罪悔改,仰望神,以便從神那裡得到公義。那些領袖們於是彼此問對方認罪悔改和仰望神對他們來說到底是什麼意思。基督對他們說,「我是世界的光,跟從我的,就不在黑暗裡走,必要得著生命的光。」他們意識到,如果他們承認基督是世上的光,那麼他們就不得不承認他們是在黑暗裡,而且他們所教導的所有那些律法也都是黑暗的。他們知道如果他們跟隨耶穌並承認耶穌基督是生命之路,那他們就不得不承認他們過去不是在引導人走生命的路,而是在走死亡之路。於是以色列的那些領袖們就起來反對耶穌,拋棄了耶穌賜給他們的救恩。他們為什麼要這樣做呢?基督在約翰福音8:44裡親手指著那些要從基督手裡搶走信徒的領袖們說,「你們是出於你們的父魔鬼,你們父的私慾,你們偏要行。」基督在這裡說他們正在重複犯路西非爾的罪,這是為什麼呢?這些領袖們由於自己的位置所產生的驕傲,由他們的權柄而產生的驕傲,由他們的才智功勣而產生的驕傲,由於他們對舊約律法的知識而產生的驕傲,無法承認他們自己是錯誤的。耶穌說他們是在行欺騙和詭詐,他們寧願棄絕生命和光的根源耶穌基督,也不願意承認他們的教導是錯誤的。正是驕傲使得那些法利賽人頑梗不化,始終不相信基督。

 今天,路西非爾的驕傲在不信的人的身上一而再,再而三的重複著。那些不信的人說,「如果我承認耶穌基督是我的救主,那就等於是說我自己就不是個好人。我就得承認我不夠聰明,看不到神的真理,我給自己選的道路和神的道路是不一樣的,靠我自己的力量無法拯救自己。」對于一個受過良好教育、生活安康的獨立的人來說,來到神面前說「我是個罪人」是個恥辱。這就是從你的父魔鬼那裡來的驕傲正在阻止你親近耶穌基督。

 撒旦的罪不僅在不信的人的身上重演著,我們在一些神的兒女的身上也同樣可以看到。這就是為什麼保羅在提摩太前書3:6裡會對那些聚會中的長老們寫上這麼一段話。在談到資格的時候,他說不能讓那些新進入教會的人做長老。為什麼?「恐怕他們自高自大,就落在魔鬼所受的刑罰裡。」魔鬼會極大地迷惑剛剛相信的信徒,讓他們自以為是因為他們的能力、智慧、學識和他們所立下的榜樣才給了他們這樣的位置,承擔了這樣的責任。這樣就會使他們重複路西非爾的罪,使自己離棄神。沒有神的兒女會脫離這種誘惑,他們都可能像路西非爾那樣,犯驕傲的罪,離棄神,不順服神的旨意。

 人類歷史上最有智慧,最通曉人的本性的人在箴言書16:18裡這樣說,「驕傲在敗壞以先,狂心在跌倒之前。心裡謙卑與窮乏人來往,強如將擄物與驕傲人共分。」驕傲是通向毀滅的路。在羅馬書12章裡,保羅告訴我們一個基督徒在被聖靈充滿的時候都會有些什麼樣的美德。大使徒在第三節裡開始說,「我憑著所賜我的恩對那麼各人說:不要看自己過于所當看的;要照著神所分給各人信心的大小,看得合乎中道。」甚至是在使徒討論基督的兒女到底應該怎麼行事的時候,他還是先從驕傲談起,因為這在路西非爾中間燃燒的罪會反復地迷惑人們,讓他們離棄神。

 民數記12:3裡說摩西是個極其謙和的人。同他那個時代的以色列相比,他有更多的理由去驕傲。他曾在法老的宮廷中受過教育,無疑比當時任何以色列人的知識都更加淵博。摩西有理由驕傲。他出身貴族,又是法老女兒的繼承人。誰有他的財富多,誰有他的權勢大?但摩西卻被形容為極其謙和,勝過世上的眾人。他的謙卑並不是因為他們有什麼可驕傲的,那是因為神在他的心裡的工作使撒旦對他的誘惑無法成功。你不認為摩西也曾因為自己的學識、財富、權勢等等而受到誘惑嗎?只是他抵擋住了這種誘惑。神用摩西並不是因為他的才學,他所受過的訓練或是他的能力。神用他是因為他勝過了驕傲的養活。摩西懷著清醒的心看到了真正的光。他知道所有這些不是因為自己的原因,而是神對他的賜予。因而他才說道,「我什麼也不是。」神要用的正是這樣的人。

 如果神時常使用你,那不是因為你將來會懂得些什麼,或是你會取得什麼樣的成就或者是你有什麼。當你起來抵擋「魔鬼的詛咒」也就是說是驕傲的時候,聖靈就會使用你。你會明白你所有的一切都來自神,你也將全心全意地依靠神。那時,也只有到那時,你才是神要用的人。我想沒有什麼比來自驕傲的誘惑更經常也更能迷惑我們的了,因為撒旦始終在想方設法要在別人身上重複自己的罪。因此,任何人都「不要看自己過於所當看的」,免得我們也按照魔鬼的思路那樣去想。

\chapter{撒旦位格}
\label{sec:ch03}
\hyperref[sec:ch02]{[上一章]}
\hyperlink{toc}{[返主目錄]}
\hyperref[sec:ch04]{[下一章]}

\begin{center}
\noindent\fbox{%
    \parbox{0.8\textwidth}{%
        以弗所書6:10-17
            \newline
            6.10 我還有末了的話、你們要靠著主、倚賴他的大能大力、作剛強的人。
            6.11 要穿戴 神所賜的全副軍裝、就能抵擋魔鬼的詭計。
            6.12 因我們並不是與屬血氣的爭戰、乃是與那些執政的、掌權的、管轄這幽暗世界的、以及天空屬靈氣的惡魔爭戰。〔兩爭戰原文都作摔跤〕
            6.13 所以要拿起 神所賜的全副軍裝、好在磨難的日子、抵擋仇敵、並且成就了一切、還能站立得住。
            6.14 所以要站穩了、用真理當作帶子束腰、用公義當作護心鏡遮胸.
            6.15 又用平安的福音、當作預備走路的鞋穿在腳上.
            6.16 此外又拿著信德當作籐牌、可以滅盡那惡者一切的火箭.
            6.17 並戴上救恩的頭盔、拿著聖靈的寶劍、就是 神的道.
    }%
}
\end{center}

 路西非爾在神所創造的天使中是最有智慧、最美麗的,神給了他無人可比的權力,讓他管理所有被創造的天使。但這無比的權力也成了使他跌倒的陷阱。因為自己的美麗和智慧而產生的高傲,使路西非爾妄圖搶奪神的榮耀,把它們都加到自己的身上。為了達到這個目的,路西非爾就想方設法要佔領神的居所,把那裡變成自己的家。他想統治所有的天使,控制整個宇宙,想做王掌權。如果撒旦要想行使神的權柄,那麼他就必須先把神對萬物的管理的權力搶奪過來。

 讓我們先來看看撒旦想要統治所有天使的陰謀,然後再看看他是如何計劃去控制所有一切,滿足他要做至高者這一妄想的。

 普通的信徒一般對天使的事情了解很少。當我們懷裡抱著一個嬰兒的時候,如果碰巧他正睡著,我們可能會望著他的小臉說他真像個天使。可如果他醒著,或是正在又哭又鬧,那我們也許就不會叫他小天使了。這樣的事情會發生是因為我們從未親眼見過天使,我們對天使的品性、活動、存在的形式以及他們的目的等等都不清楚。其實,神已經清楚地向我們顯明瞭天使的國度。在我們了解撒旦之前,我們有必要先了解一下天使國度裡一部份重要情況。

 我們生活在物質世界裡,習慣于用重量、大小、形狀等等來描述事物,很少會想到天使或是天使的國度。其實當神告訴我們他是從什麼時候起創造萬物的時候,他指得並不僅僅是這個物質世界,也不是單指的球上的一切,其中也包括了他所創造的天使的國度。靠著神的話,有許多天使被創造了出來。使徒保羅在歌羅西書1:16裡告訴我們說:「因為萬有都是靠他(主)造的,無論是天上的,地上的;能看見的,不能看見的;或是有位的,主治的,執政的,掌權的,一概都是藉著他造的,又是為他造的。」在談到神創造的工作的時候,使徒保羅把所造之物分為截然不同的兩類。不能看見的是屬天的一類,能看見的是屬於在地上的國度。我們不能認為屬天的那一個國度是看不見的,就認為它不像後者那樣真實可信。使徒保羅總結說所有一切都是出自神子的工作,他不僅是地上以及在居住在地上的一切事物的創造者,也是天使的國度以及眾天使的創造者。所以說天使也是受造之物,他們是由神的權柄通過神子的能力而被造出來的。

 天使也有人的品性。他們並不是一種力量或能力,他們僅僅是一些具有人性的個體。神的話語指出他們具有所有的人性。從詩篇148:2裡我們可以知道天使敬拜神。這是一種有意識的行為,說明他們擁有自己的意志。天使們敬拜神是因為他們了解神,他們有學習知識的能力;他們敬拜神是因為他們明白神是值得他們去愛的神,值得他們去順服、去服侍的神。在馬太福音24:36裡,神告訴我們天使知道一些事情,但也有些事是他們所不知道的。在聖經裡,天使被看作是擁有一定人性、智慧、情感和意志的個體,和人一樣,他們也有自己的身份和存在的空間。

 我們從希伯來書1:14裡可以進一步發現創造天使是用來服侍的。使徒保羅說,「他們是服役的靈,奉差遣為那將要承受救恩的人效力」。雖然天使可能有不同的職責,當他們都是僕人,神造他們是要他們執行他的旨意,神對這個世界所發出的旨意就是通過這些天使來完成的。眾天使看護著所有人的生命,奉命保守那些承受救恩的人。神對這世上的審判往往也是通過他們來執行的。因此說,創造天使不是要讓他們設計出神的計劃,而是要他們去執行神已經做好的計劃,神才是管理一切的主,只有他才有權柄去管理那些生活在地球上的人們。

 天使不會死去。在主還在地上的時候,主的敵人曾來到他的面前並試探他有關復活的真理,主當時在馬太福音22:28-30 裡告訴說,復活的人跟天使一樣,既不娶也不嫁。這幾節告訴我們,天使的位置不會因死亡而空缺,因而他們沒有必要像人那樣繁衍子孫,以保持一定的數目。每個天使的生命都是被創造出來的,都是靜止的生命。

 天使沒有像人一樣的肉體,但這並不意味著他們沒有身體。對許多人來說,這一點很難搞明白。人們有時想天使就像是一股清煙,有時顯現有時隱藏,沒有一定的存在形式。保羅在教導我們有關身體重生的問題時,在歌林多前書15章裡告訴我們有不同的形體和肉體。適合在地上生活的叫做地上的形體,另外還有一種是天上的形體,跟地上的形體都是一樣確實可信的。

 我們不知道天上的形體到底有什麼特性,不過借助于觀察我們的主耶穌復活的前後我們還是可以了解其中的一二。那種形體有形狀、模樣和重量,但顯然又和我們的血肉之軀有所不同,不受血肉軀體各種原則的限制,而且是永遠不會毀壞。我們主復活而榮耀的身體也並不受時間或空間的限制。他可能此刻出現在耶路撒冷,瞬間有出現在加利利。一個滿心恐懼的猶太人可能會把房門緊閉,蜷縮在屋子裡,但是那緊鎖的房門卻擋不住主耶穌的路。我們無法用自然規律來解釋所有的這一切,但聖經告訴我們復活後的榮耀的身體有一些特別之處。

 跟耶穌的屬天的身體一樣,天使也一定不會受到時間和空間的束縛。我們可以從但以理書9章裡神給但以理的啟示中得到驗證。21節裡說:「我正禱告的時候,先前在異象中所見的那位加百利,奉命迅速飛來,約在獻晚祭的時候,按手在我身上。」天使可以風馳電掣般從一個地方到另一個地方去,這一點引起了但以理的注意。天使都居住在屬天的國度裡,他們本來就和住在地上的生物不同,不必靠呼吸空氣而生存。而對于人來說,離了空氣就無法生存下去。當宇航員們飛離地球進入太空的時候,他們必須要攜帶著空氣才能維持他們的生命。他們的太空倉就是個縮微後的地球,而當他們離開太空倉,在太空中行走的時候,為他們特制的宇航服裡也充滿著空氣。人沒有空氣就沒辦法生存。但創造天使的時候並不是按照他們要在地球或要靠大氣來生活的要求來做的。他們是按著在天國裡生活和存在的要求被造的。神特別指明瞭這一點。在馬可福音13 :32裡,神說「但那日子,那時辰,沒有人知道,連天上的使者也不知道」。「天上的」說明瞭到處創造天使時到底是要他們生活在什麼地方的。

 因為每個人都有自己的守護天使,所以我們可以說天使的總數一定不會小於生活在地上或將要在地上生活的人。聖經裡沒有告訴我們天使的數目到底有多少,只簡單地說是數也數不過來。神真是偉大,竟然可以創造出無以記數合他心意的天使,而且這些天使都如同神的僕人一樣在時刻準備著去執行神的旨意。

 在路西非爾在背叛神、篡奪神的寶座之前,有一件事他必須要首先做到,那就是控制住眾天使。我們可以從啟示錄12:4裡找到有關撒旦最初背叛神的一些線索。在此之前,約翰讓我們先看了他得到的屬天的異象,在3節裡他說到有一條大紅龍,在9節裡他解釋到這龍就是撒旦。4節中說:「它的尾巴拖拉著天上星辰的三分之一」。這裡的星辰似乎說得就是天使,這好像是在告訴我們撒旦當初也勸服了一些天使跟隨他一起叛亂,而且他已經成功三分之一了。神創造天使的時候都給了他們各自的意志力和選擇的能力,此刻,他們就面臨著一個選擇:要麼留在天國和神在一起,要麼相信撒旦要高舉他們的地位的許諾,跟隨撒旦一起背叛神。路西非爾不僅要高舉自己,而且還要高舉所有那些跟從他的天使,好使他能夠最終奪得期盼已久的勢力和權能,篡奪神的寶座。所以在路西非爾反叛神的時候,他率領著那三分之一相信他更應該執掌王權的天使一起背叛了神。

 撒旦照著天國的樣子也同那些跟隨他的天使們建立了一個國度。在後面的學習中我們將了解到,除了當初背叛神的計劃是撒旦所設計的之外,他其實再也沒有任何新創的東西。撒旦只是個模仿者,絕不是一個創造者;他背叛神而謀劃他自己的國度的時候,他用得還是神的管理模式。這一點在以弗所書6:12裡得到了驗證:當講到信徒們要面對的征戰的時候,使徒保羅說:「我們並不是與屬血氣的爭戰,乃是與那些執政的,掌權的,管轄這幽暗世界的,以及天空屬靈氣的惡魔爭戰。」值得注意的是,這裡所說的與歌羅西書1:16裡神的安排是一致的。撒旦給跟隨他的一些天使安排了位置,給了他們一定的權力。例如:在巴勒斯坦之上安排了一個與米迦勒的職份差不多的人,在這人之下也安排了一個與加百利差不多的人,而在這人之下還有人執掌12支派,而每個支派也都相應有人管理下面的人。可見,撒旦是完全在照搬神管理天國的模式。

 我們把那些最初跟隨撒旦叛亂的稱為魔鬼。這些魔鬼都是墮落的天使,他們仍然擁有著他們背叛神之前就有的能力。他們過去有什麼樣的權勢和智慧,在叛亂之後他們也還依然擁有。和天使們一樣,這些魔鬼也不受時間和空間的限制。他們的數目雖然比跟隨神的天使少,但也仍然是數不勝數。撒旦不是無所不在的,他不可能同時出現在你的家裡,又跑到我這裡來。我這一點上得到了一些安慰。撒旦不是單靠他自己來做工,而是靠他下面那些跟他一起叛亂的魔鬼來進行的。每時每刻,神的兒女的週圍都被一些墮落的天使圍著,但同時也有神所指派的天使守護著他。我們要對付的並不是些虛幻的東西,也不是與神的原則相對的魔鬼的原則;我們的敵人是那些奉命要阻攔和攻擊那些要順服神的旨意的他兒女。撒旦手下的魔鬼對他都十分忠誠,始終都在履行撒旦給他們的職責。他們不像職員那樣早晚打卡上下班,每天有半個小時的午休和兩個茶休時間來休息一下。作為屬靈的個體,他們不受時間和空間的限制。不管你在哪裡或者你在做什麼,他們都會看著你。每時每刻,這些撒旦的看門狗都在你的左右,而神派來的天使也一直都守護在你的身旁,因為你是神所救贖的兒女。那些選擇跟隨撒旦的魔鬼都在全心全意地遵守著撒旦的旨意;而他的旨意就是要時刻在你的生活裡攻擊神給你的旨意。

 我們知道在神將來要把這些魔鬼都下到火湖裡去。正如主在馬太福音25:41裡所說的,那「永火」是為魔鬼和他的使者所準備的。他們都註定了要承受這永永遠遠的咒詛和審判,遠離神的永生,忍受地獄烈火的煎熬。儘管他們已經揹負了這咒詛,可他們的活動仍然很猖獗,這似乎是因為他們相信對他們的審判一定會降臨,所以更加激發了他們這最後的瘋狂。這是一幅多麼恐怖的畫面。不過我們如果通過神的話語把這真理向你顯明,讓你明白你的敵人的這些情況,你就可以在我們的幫助下去仔細準備迎接耶穌基督所應許給我們的勝利。當你歡喜地度過每一天的時候,你並沒有脫離撒旦和他手下那些魔鬼的攻擊。但當你知道撒旦在他墮落之後,為了推翻神建立起他自己的國度,你就會明白他其實和耶穌一樣在時刻看著你。撒旦妄圖打敗神的計劃只有在你裡面,通過你才能夠實行。所以,我們才每天、每時、每刻都受到撒旦的攻擊。因為我們無法用我們的眼睛看到我們的敵人,因為我們所要面對的是一個想要顛覆神的權能並妄圖把撒旦推上寶座的的黑暗的組織,所以我們必須要了解他的目的是什麼。我們也必須了解他工作的方法,這樣我們才能像使徒勸告我們說的「拿起神所賜的軍裝,好在磨難的日子抵擋仇敵,並且成就了一切,還能站立得住」。撒旦不可能召集來所有的天使,他也不會完全實現他的願望,不過他確實召集來了很多天使,並模仿天國的樣式建立起他的國度,他就是想靠這些跟隨他的天使來奪取本屬於神的榮耀,打敗天國裡神所帶領的眾天使。

 你知道嗎?每一次你不遵守神的旨意而屈服與撒旦的誘惑,你就是又給撒旦投上了一票。如果你不知道耶穌基督是你個人的救主,是否可以說你生來就是撒旦黑暗權勢的有部份;你生來就是叛逆神的;你生來就在他的權勢之下,撒旦的權勢之下。撒旦是他的世界裡神,而你跟隨著他就像是你根本就和創造了你的神毫無關係。唯一一條脫離他黑暗權勢統治的路途就是在一個新家裡重生,獲得一個新的生命。基督的死就是要使我們能夠脫離撒旦的權勢,把我們帶進神的大愛的國度。如果你接受耶穌基督作為你個人的救主,神不僅會赦免你的罪,使你成為他的兒女,把你帶進神家,而且他還要打破撒旦捆綁在你身上的權勢,把你解救出來。

 我把這位救主介紹給你,他能夠使你脫離黑暗,使你從這個世界的神的捆綁下解救出來。

\chapter{撒旦對這個世界的征服}
\label{sec:ch04}
\hyperref[sec:ch03]{[上一章]}
\hyperlink{toc}{[返主目錄]}
\hyperref[sec:ch05]{[下一章]}

\begin{center}
\noindent\fbox{%
    \parbox{0.8\textwidth}{%
        創世記 3:1-7
            \newline
            3.1 耶和華 神所造的、惟有蛇比田野一切的活物更狡猾。蛇對女人說、 神豈是真說、不許你們喫園中所有樹上的果子麼。
            3.2 女人對蛇說、園中樹上的果子我們可以喫.
            3.3 惟有園當中那棵樹上的果子、 神曾說、你們不可喫、也不可摸、免得你們死。
            3.4 蛇對女人說、你們不一定死、
            3.5 因為 神知道、你們喫的日子眼睛就明亮了、你們便如 神能知道善惡。
            3.6 於是女人見那棵樹的果子好作食物、也悅人的眼目、且是可喜愛的、能使人有智慧、就摘下果子來喫了.又給他丈夫、他丈夫也喫了。
            3.7 他們二人的眼睛就明亮了、纔知道自己是赤身露體、便拿無花果樹的葉子、為自己編作裙子。
    }%
}
\end{center}

 正如我們發現的那樣,路西非爾渴望把那本屬於無限和永久的神的榮耀加到自己的身上,所以他把無數的天使迷惑到他的管轄之下。我們在啟示錄12:4裡可以讀到,撒旦率領著三分之一的天使一起背叛了神。但路西非爾的野心並不僅僅是要篡奪神的權能,他還想要在地上稱王掌權,那樣他就能宣告自己是一個獨立的神,他的權能與神的權能相當。

 為了達到這個目的,他開始了他的計劃。在創世記的第一章裡,神在創造人的時候說(26,27節):「我們要照著我們的形像,按著我們的樣式造人,使他們管理海浬的魚、空中的鳥、地上的牲畜和全地,並地上所爬的一切昆蟲。神就照著自己的形像造人,乃是照著他的形像造男造女。」可見,當神創造人的時候就把人放在地球上,並且給了人管理地球的權力。人和神之間一開始就有依靠的關係。人對神的依靠使他能夠認定神的王權,也就是說承認神有管理和統治的權力,他是榮耀之神。而神又讓人代表神來管理地上的萬物和神的國度。人成了統治者,但他是在神的許可下來進行管理的。撒旦要想控制整個地上的世界,唯一的辦法就是攻擊人。

 當我們翻到創世記3章的時候,我們會發現撒旦對這地上的世界所進行的第一次攻擊的對象就是代表著神的人。我們對這裡所記載的迷惑都不會感到太陌生。所有相信神的話語的信徒都相信這裡所記載的事件絲毫沒有誇張的成份,根本就和那些神話傳說毫不相干。它也不是一些思想深邃的先賢為解釋罪的存在而杜撰出來的,用來表明他們那些深奧觀念而憑空想像出來的事例。這是歷史。神創造人後把人放到了伊甸園,而路西非爾為了使人的心背離神也進入到了園中。

 神當初把亞當放到伊甸園裡的時候,伊甸園就像是天國一樣的完美。神在創世記2:16,17裡說,「耶和華神吩咐他說:『園中各樣樹上的果實,你可以隨意吃;只是分別善惡樹上的果子,你不可吃,因為你吃的日子必定死!』」亞當從未想過去問問神到底為什麼會有這樣的規定。他也從未想過神會心懷嫉妒,把一些對他有好處、有幫助的東西藏起來不給他。神的恩典是無邊的,他已經把所有他所造之物可能會用得到的、需要的和想要的東西都給了他們,他不讓亞當做的是要讓他有真正的自由。可路西非爾試探亞當的正是神告誡亞當必須禁止的事情。

 從創世記3章第一節裡我們知道蛇是神所造的活物中最狡猾的一個。我們從這裡可以開始了解到撒旦做事的方法,他無法在地上顯現出來他在天國時的外貌。如果他要在地上以人能看得見的形像出現,他必須要扮成其他物體的樣子並在這樣的樣式下工作。而永生神的兒子能夠以肉身的形式顯現。在舊約裡,耶和華的使者是要來到世上的主耶穌基督的半肉身形式的表現。耶和華的使者能以肉身顯現,還可以與人同行、和人交談。但撒旦沒有這樣的能力,而且他必須靠他所顯現出的樣子才能控制人或動物。當撒旦進入伊甸園要試探亞當和夏娃的時候,為了要讓他們聽從他的意思,撒旦選擇了蛇的樣子出現。

 我們並不是要說某個爬行動物開始了那項陰謀,也不是在說某個爬行動物違背了神的旨意或是某個爬行動物對亞當和夏娃所做的決定有興趣。那個爬行動物只是給撒旦提供了它的身體,而且經上記的是蛇(此時被撒旦所控制)比田野裡一切的活物都更狡猾。因為沒有任何一種動物曾經有過要背叛神的想法。神所創造的動物都十分順服神的旨意。當耶穌在曠野裡受撒旦誘惑40天的時候,福音書中記著說唯有野獸陪伴著耶穌。這其中有什麼特別嗎?是的,這是因為除了人以外所有生物都認得神的權柄。當時陪著耶穌受40天試探的時候,所有的野獸都順服了耶穌的權柄。那些野獸中並不包括蛇,這是因為它不僅比其他的野獸都更聰明,而且還是撒旦用來接近夏娃的一件很好的工具。正是這比任何野獸都更狡猾的蛇促使夏娃背叛了神的旨意。

 撒旦以蛇的樣子接近夏娃的好處就是他可以掩飾住他本來的樣子和他最終的企圖。因為如果撒旦直接來到夏娃的面前開誠佈公地告訴她他與神為敵,並勸她也一起違背神的旨意的話,他清楚亞當和夏娃肯定會拒絕他並斥責他的誘惑,那樣撒旦想要征服整個世界的妄想就根本無法實現了,所以撒旦必須要把自己喬裝裝扮一番才行。翻開新約歌林多後書11:13,14,我們會發現使徒保羅在這裡也記載了撒旦慣用的這種手法。歌林多後書11:13,14節說,「那等人是假使徒,行事詭詐,裝作基督使徒的模樣。這也不足為怪,因為連撒旦也裝作光明的天使。」啟示錄12章裡在談到撒旦的時候也指出他喬裝打扮的事。9節裡說,「大龍就是那古蛇,名叫魔鬼,又叫撒旦,是迷惑普天下的;…」

 因為魔鬼和撒旦有「騙子」和「說謊者」的意思,現在已經成了特殊的詞彙。當撒旦起來背叛神的時候,他就像騙子一樣污衊神的品格和他的愛,離間亞當、夏娃和神之間的關係,讓他們背離神的旨意。記住這條原則:撒旦的工作總是通過污衊神的美好和聖潔來離間人與神之間的關系,從而使他們背離神的旨意而實現的。那條狡詐的蛇懷著欺騙的心思問了這樣一個問題,「神豈是真說,不許那麼吃園中所有樹上的果子嗎?」他問夏娃這個問題是想看看夏娃到底對神的話理解了多少。如果撒旦想要欺騙一個人,他一定會首先了解那個人懂得多少東西;他的這條原則直到今天也沒有改變。如果那個人一點也不知道神的話語,也不了解神的品性以及神的要求的話,那麼撒旦就可以很容易地使他相信他和神之間根本沒有什麼罪的問題。可如果那個人明白神的話語,並且也知道神的聖潔和他自己的罪,那撒旦就很難把他捆綁在黑暗之中。

 撒旦於是首先來到夏娃的面前探聽探聽她到底對神的話語了解多少,因此撒旦問了這樣一個問題,「神豈是真說,不許那麼吃園中所有樹上的果子嗎?」夏娃於是向撒旦承認神的確有條禁令,她正確地回答說,「園中樹上的果子我們可以吃,惟有園當中那棵樹上的果子,神曾說:『你們不可吃,也不可摸,免得你們死。』」我們應該注意到夏娃明白神的禁令,也知道違反它的懲罰是什麼。現在我們已經知道了很多:夏娃熟悉神曾說過的話,即神要他們遵守他的話,而且神已經定下了要如何懲罰那些不順服他的話的人。撒旦在了解到這些基礎上一步步開始實行了他的詭計。

 撒旦對夏娃的回答直接提出了反對意見。「蛇對女人說,你們不一定死。」這直接反對了神的話語。別的生物從未有這樣惡毒地攻擊神,這是在直接了當地說神是個騙子。撒旦是在宣告神的話是騙人的。如此性格扭曲、叛亂成性的撒旦有什麼資格來宣告聖潔而公義的神是在欺騙人?

 撒旦接著在5節裡向夏娃解釋說,「你們不一定死,因為神知道,你們吃的日子眼睛就明亮了,你們便如神能知道的善惡。」為了看清撒旦的意圖,我們在這裡有必要對這一節進行仔細的分析。「你們便如神能知道的…」裡的「神」原文是Elohim,這是舊約裡對神的稱呼。亞當和夏娃對假神並不知曉。撒旦說這句話的時候,他其實是在說如果你吃了這樹上的果子,你就和神一樣了。還記得先知以賽亞說撒旦的野心是什麼嗎?「我要與至上者同等。」撒旦對夏娃的誘惑是,如果夏娃不順服神的話,伸出手去摘下那果子並吃掉它,那麼她就能把自己抬高到和神同等的位置。撒旦明白他若能使人順服他的話,那麼他就有權受到那人的敬拜,這是因為他已經成了那人的主。他也知道如果他能誘使夏娃背離神的道,那麼這種背叛對他來說就是一種順服,他就有權得到夏娃對他的敬拜。而如果人都遵從了撒旦敬拜撒旦,那麼他就篡奪了神在他所創造之人中間的王權,撒旦也就真的與神同等了。撒旦話語中的含義其實是:「神很嫉妒;神想自己獨享所有的權力。神不想這榮耀與其他人一起來分享。神知道如果你吃下這個果子,你就和他一樣了。所以他故意不讓你吃這果子,好使你永遠在神之下。如果你吃下它,你就和神沒有兩樣了。」

 經過撒旦的這番鼓動,夏娃的心底裡昇起了一股渴望,她想與神同等,讓自己的身上也有來自那神的寶座的榮耀,她也想坐在那榮耀的寶座上。於是夏娃伸出手去摘下那果子並吃了下去,還讓亞當也吃了那果子。我們從7節裡可以看到其結果是,「他們二人的眼睛就明亮了,才知道自己是赤身露體。」沒有人知道他們赤身露體,伊甸園裡的動物也不懂得他們是赤身露體的,甚至他們在撒旦的眼裡也不是赤身露體的。他們只有在神的眼裡才是赤身露體的,因為在他們的身上已經再沒有什麼能遮蓋住他們叛逆的生活和行為了。什麼東西也無法遮掩住由叛逆神而產生的罪惡。

 亞當在接過夏娃遞過來的果子的同時,他放棄神給他的「治理這地」的權力。亞當不可能同時一手掌管神給他的權力,一手又拿著神禁止他們吃的果子。他只有在順服神的時候才能夠實行治理和管理的權力。而在此刻,路西非爾就篡取了亞當丟下的權力。

 我們回頭再來看看以弗所書2章2節,使徒這樣寫到:「那時,你們在其中行事為人,隨從今世的風俗,順服空中掌權者的首領」。從這裡我們可以知道,撒旦已經篡奪了神在天使中間的王權。在歌林多後書4:4裡,我們還可以發現使徒保羅認定撒旦已經篡奪了神在空中的王權。「此等不信之人被這世界的神弄瞎了心眼,不叫基督的榮耀福音的光照著他們。基督本是神的象。」撒旦在這裡被稱做是這世界的神。撒旦率領一部份天使背叛了神,使他自己成了空中的掌權者;而通過誘使亞當和夏娃違背神的旨意,他又成了這世界的神。撒旦篡奪了這兩處的王權,把本屬於神的榮耀加到了自己的身上。在一大群墮落天使的跟隨下,他宣告自己和神沒有關係,他和神是平等的,甚至是高于神的。從亞當的墮落開始,所有的人都跟隨了撒旦。撒旦篡奪了本屬於創造者的權力,他現在是這世界的王。

 當思考伊甸園裡受誘惑的這段經歷的時候,我們會吃驚地發現原來夏娃知道神給她的話到底是什麼意思。她很清楚神對她的命令是什麼。她也明白神的旨意。她本來在樂園裡與神有著和諧而美好的關係,靠著神的愛享受著園中一切美好的事物。而當她通過人的想法來懷疑神給她的道的時候,她的罪就開始了。她聽著撒旦在她耳邊說那些質問神的話,時候就開始違背神的權能。

 當一個人用自己的心思意念開始來接觸神的話語的時候,他的心底一定會有各種懷疑和不信;用人的思維來辨察屬神的事不過是給他自己打開了否定神的大門。使徒在羅馬書1章裡講到這個世界都不認識神,這個世界上的聰明人對屬神的智慧都一無所知。歌林多前書1:18-25更清楚地表明了這些:

 因為十字架的道理,在那滅亡的人為愚拙;在我們得救的人卻為神的大能。就如經上所記:「我要滅絕智慧人的智慧,廢棄聰明人的聰明。」智慧人在哪裡?文士在哪裡?這世上的辯士在哪裡?神豈不是叫這世上的智慧變成愚拙嗎?世人憑自己的智慧,既不認識神,神就樂意用人所當作愚拙的道理拯救那些信的人;這就是神的智慧了。猶太人是要神跡,希利尼人是求智慧,我們卻是傳釘十字架的基督,在猶太人為拌腳石,在外邦人為愚拙的;但在那蒙召的,無論是猶太人,希利尼人,基督總為神的能力,神的智慧。因神的愚拙總比人智慧;神的軟弱總比人強壯。

 當使徒來到歌林多時才這樣說:「我曾定了主意,在你們中間不知道別的,只知道耶穌基督並他釘十字架。」

 如果今天還有人相信屬世的智慧、屬世的哲學,那麼他永遠不會明白神和神的道,這是因為人憑借自己的智慧永遠無法認識神。只有當他把自己的聰明放到一邊,接受了神給他的啟示,他才能真正地來到神的面前、認識神。無論他的名字後面拖了多少個官銜、稱號,只有當他真正地憑著信心相信神的話語,他才能開始了解神的國裡面的事情。

 路西非爾問夏娃說,「你到底對神的話了解多少?」他這種對神否定的態度加深了夏娃心中對神的懷疑,而這種懷疑最終產生出對神的叛逆。這種對神的背叛最終導致了神給亞當管理地上萬物的權力落入了撒旦的手中。由於夏娃對神的真理猶疑不定,由於夏娃和亞當順服了撒旦,把撒旦當作神來敬拜,使得神的榮耀被撒旦據為己有。今天,你仍然可能重複犯這樣的罪。除非你把自己完全交托給神的真理和他的道,除非你把這作為你永遠的生命的根基以及每日生活的標準,你仍舊有可能被誘惑,走上黑暗之路。撒旦渴望著能把你捆綁在他的權勢之下,受他的控制,服從他的指派。除非你憑著信心接受耶穌基督為你個人的救主,你無法再找到其它的辦法脫離那黑暗的權勢,進入神的國。只有你接受了耶穌基督,你才能得到神的幫助,才能得到永遠的生命。神給了你這樣一位獨一無二的救主,他就是神的智慧,神的權能,神的公義,只有他才能把你從撒旦的控制下解救出來。

\chapter{騙子撒旦}
\label{sec:ch05}
\hyperref[sec:ch04]{[上一章]}
\hyperlink{toc}{[返主目錄]}
\hyperref[sec:ch06]{[下一章]}

\begin{center}
\noindent\fbox{%
    \parbox{0.8\textwidth}{%
        提摩太前書4:1-8
            \newline
            4.1 聖靈明說、在後來的時候、必有人離棄真道、聽從那引誘人的邪靈、和鬼魔的道理。
            4.2 這是因為說謊之人的假冒.這等人的良心、如同被熱鐵烙慣了一般。
            4.3 他們禁止嫁娶、又禁戒食物、〔或作又叫人戒葷〕就是 神所造叫那信而明白真道的人、感謝著領受的。
            4.4 凡 神所造的物、都是好的.若感謝著領受、就沒有一樣可棄的.
            4.5 都因 神的道和人的祈求、成為聖潔了。
            4.6 你若將這些事提醒弟兄們、便是基督耶穌的好執事、在真道的話語、和你向來所服從的善道上、得了教育。
            4.7 只是要棄絕那世俗的言語、和老婦荒渺的話、在敬虔上操練自己。
            4.8 操練身體、益處還少.惟獨敬虔、凡事都有益處.因有今生和來生的應許.
    }%
}
\end{center}

 早在神創造這世界之前,路西非爾就已經對神的榮耀和他的權能以及能力垂涎三尺了。為了讓自己能得到這種榮耀並且施展這樣的權能,他必須得哄騙一些受造之物離開神,順服於他的指揮。神創造萬物是要讓他們聽從他的旨意,榮耀他的名。路西非爾沒有創造的能力。他要稱王,只能欺騙原先屬於神的受造之物,把他們納入自己的麾下。當撒旦在一部份天使中間篡奪了統領的權力之後,他就稱自己是那個世界裡的神。

 撒旦是個大騙子。為了達到他的目的,他把那否認、拒絕真理的世界變成了謊言的國度。路西非爾不可能對那些天使說出真相:如果他們跟隨他,他們最終的結果就是和神永遠相隔,最後被投入火湖之中;這樣他根本無法勸服那些天使跟隨他一起背叛神。當撒旦來到夏娃面前的時候,他也不可能告訴夏娃如果她真的聽從他的話,她就要和他和那些墮落的天使一道墜入永遠燃燒的地獄烈火之中。他必須撒謊才能達到他的目的。這就是為什麼當主面對那些拒絕真理的法利賽人時,在約翰福音8:44說:「那麼是出於你們的父魔鬼,你們父的私慾,那麼偏要行。」基督說那些法利賽人正在按照撒旦的思路進行思考,撒旦怎麼讓他們想,他們就怎麼想。他指出他們此刻擁有撒旦身上的兩點特徵,撒旦「從起初是殺人的」,而那麼現在正在盤算著如何殺掉我。接著他又對那些宗教領袖們說,撒旦「不守真理,因他心裡沒有真理。他說謊是出於自己,因他本來是說謊的,也是說謊之人的父。」

 在啟示錄12章,約翰講到撒旦在破壞神的計劃和旨意的那段患難時期裡要進行的一些工作。接著他在9節裡指明他一直那條大龍來象征的「就是那古蛇,名叫魔鬼,又叫撒旦,是迷惑普天下的」。他欺騙了夏娃來跟隨他;結果他的騙局一直都籠罩在亞當所有子孫后代的心頭,整個世界都被他騙了。約翰沒有忘記創世記3章裡我們所看過的撒旦質問夏娃對神的了解的事。當撒旦發現夏娃知道神的話語之後,他就對她撒了慌。他在4節裡對她說,「你們不一定死。」這是個彌天大謊,因為神說過,「分別善惡樹上的果子,你不可吃,因為你吃的日子必定死!」神所說的死是指人與造物主分開了,人從靈上與主分離,身體的死亡,也就是身體與靈魂的分離,是靈的死的結果。

 真實的神曾經說過:「必定死。」路西非爾(說謊之人的父)膽子碩大無比,他竟然膽敢來到夏娃的面前扯著彌天大謊說,「神說的不是實話;你們不一定死。」夏娃此刻面臨著一個抉擇,她應該相信誰?當你遇到兩個截然相反的論斷的時候,這兩個論斷不可能同時都是正確的。夏娃一定經過反復的考慮。到底是神說慌了,還是路西非爾說的「你們不一定死」是謊話?結果夏娃最後判定神是在說謊。這是連想也不該想的,簡直是對神莫大的侮辱,可夏娃的結論就是:神是在說謊,撒旦講的是真話。由於她聽信了撒旦的謊話,把神當成是騙人的,她和亞當都吃下了那禁果。夏娃和亞當一吃下那果子就立即發現撒旦才是那騙子,神才是真實的;於是他們就連忙用無花果樹的葉子給自己編了裙子,不讓神看到自己在他的眼裡是赤身露體的。這些證明瞭亞當、夏娃知道他們從那時起就受到咒詛,因為神的話是確實的。

 翻到新約提摩太前書2:14,我們可以看到使徒保羅是如何評論這件事情的。「不是亞當被迷惑,乃是女人被迷惑,陷在罪裡面。」保羅追根溯源,指出夏娃是犯罪的源頭;是她上當受騙,相信了謊話。

 神的話語向我們揭示了這個騙局的始末,使我們明白了撒旦行事的首要原則是什麼。在這裡我們想強調一下這方面的真理。我們必須要清楚地明白撒旦在任何情況下也無法通過實話來達到他的目的,他只有靠謊言和欺騙才能實現他的心願。明白了這些,我們就可以對他給我們的這種誘惑和鼓動做好準備。

 神的話語非常明確地指出了撒旦的一些謊言。讓我們首先來看看歌林多後書4章裡使徒保羅強調指出撒旦首先是詆譭神的話語。在歌林多教會,一些蔑視他的教訓的人起來攻擊保羅的權柄。他們無法詆譭他的教訓裡的真理。因此他們就紛紛污衊說保羅來歌林多並沒有神所賜的權柄,也沒有使徒的權柄和傳遞神的話語的權柄。他們說保羅來講的是他自己的話。面對這種攻擊,使徒在4章1-2節裡說,「受了這職份,就不喪膽;乃將那些暗昧可恥的事棄絕了…」注意到這句話嗎?「乃將那些暗昧可恥的事棄絕了」,保羅是在說,「我們的講道和教訓裡沒有絲毫謊言。我們行事為人沒有什麼虛假的。」暗昧就是虛偽,不敢讓事物顯現出原來的樣子。在我們的生活裡就有這種虛偽。使徒接著說,「不謬講神的道理」。我們向你傳講的神的道都是千真萬確的。

 保羅在這裡把自己同那些詆譭神福音的救恩,否定人靠著對耶穌基督的信心就可以得救的那些假師傅對立起來。這些來到歌林多的假師傅宣講著他們編造的謊話。他們不相信保羅所傳講的福音,為了讓更多的人相信他們的謊話,他們就詆譭保羅是生活在謊言裡的騙子。那些人宣稱他們的教訓來自於神,到處對人說我們說的才是真理,可其實那些都只不過是他們自己的哲學。於是保羅在這種情況說,「我們既然蒙憐憫,受了這職份,就不喪膽;乃將那些暗昧可恥的事棄絕了,不行詭詐,不謬講神的道理;只將真理表明出來,好在神面前把自己薦予各人的良心。」

 撒旦如果要控制那些被他的謊話所迷惑的人的心志,他就一定得弄瞎他們的心眼,不讓他們知道神的福音。神的福音就是神親自把自己顯明給我們看,他把我們需要了解的一切都通過他的話語告訴了我們。由於只有這個才是神的真道,所以撒旦千方百計設法欺騙人,不讓他們明白神的話。如果你去研究一下從神給人降下福音那時起直到今天的各種教導,你就會發現撒旦對神的攻擊首先就是否認聖經是沒有謬誤的,而且還否定人們對聖經的權柄和靈性的認識。

 今天,很多人都在公開地詆譭神的道,無數牧師站在講臺上,他們手中的聖經裝禎精美,可他們卻完全忽略了其中的真理。他們公開地否定神的道,而同時卻又把自己稱做是神的子民。在各種院校裡,人們肆無忌憚地攻擊著神的道,攻擊聖經的真實性,否認聖經無謬誤,否認聖經的權柄。看起來似乎撒旦已經征服並控制了著世界的想法,控制了那些智慧人、聰明人、學者、專家;好像只有一小部份人還堅信聖經的真實和權柄。

 這只是撒旦的陰謀的一部份。他想要聖經不是神所默示的謊話在人們中間廣為傳播,他想讓人們只為了滿足自己對神一時的好奇心而來研究聖經,不讓它成為我們生活的一部份。我們堅信神的福音是完全真實的,是有權柄的。我們相信聖經是人在聖靈的引導下有神默示而寫成的,它是我們信心生活和日常生活的原則和方向。任何不符合神的道的行為都會陷入魔鬼的謊言之中。神把我們從迷惑中解救出來,使我們不在懷疑他的福音的權柄。

 讓我們再一起看看提摩太前書4章,撒旦還在人中間以及耶穌基督的工作裡傳播謊言。在提摩太前書4:1裡保羅說,「聖靈明說,在後來的時候,必有人離棄真道,聽從那迷惑人的邪靈和鬼魔的道理。」保羅在這裡是說,「他們應該留意那些假教師,那些人要迷惑人離開神的真理,小心魔鬼所傳播的那些言論。」使徒已經預測到了我們今天面臨的這種景況。他預言說撒旦將會統治宗教界,那些嘴邊總掛著基督的人在行為上卻絲毫看不出與聖經裡所講的基督徒有任何相同之處,那些在主日學校頂著神的名在講臺上講課的老師一再像撒旦那樣否定著神的真道以及耶穌和他在世上的工作。保羅說,「我要提醒你們小心以後要來的那些假師傅,這些人會像撒旦當初欺騙夏娃那樣再來迷惑你們,他們所教訓的都是撒旦的道。」你或許會想,當撒旦來到教會之前會在大門口停住,說:「這裡沒有我的位置。」噢,事情可不會這樣。撒旦已經進入了教會,他的教訓已經在教會裡生了根。他已經到過許多講壇之上,精心而又巧妙地篡奪了傳道和牧養的權力。毫無疑問,神的真理在那裡被人棄絕,遭到誹謗和攻擊。

 使徒約翰在約翰壹書2:21,22裡也預示了同樣的情景。約翰說,「我寫信給你們,不是因為你們不知道真理,正是因你們知道,並且知道沒有虛謊是從真理出來的。誰是說謊話的呢?不是那不認耶穌為基督的嗎?不認父與子的,這就是敵基督的。」我們可以看到約翰在提到是誰說謊話的時候所指的非常明確。那個說謊者就是不承認耶穌基督是彌賽亞的人,是那不承認耶穌是神計劃來拯救世人並要在他的子民中做王掌權的人。約翰壹書4:1裡也指明瞭同樣的真理。約翰說,「親愛的弟兄啊,一切的靈(每個師傅都宣稱自己是奉神的名來的),你們不可都信,總要試驗那些靈是出於神的不是…」你們總要試驗試驗,看看那些靈是不是出於神。每個師傅來了都說自己是奉神的呼召來的,辨別的辦法就是用神的道來試驗他們。聖經是神給我們的標準。要「試驗那些靈是出於神的不是,因為世上有許多假先知已經出來了。凡靈認耶穌基督是成了肉身來的,就是出於神的,從此你們可以認出神的靈來。凡靈不認耶穌,就不是出於神,這是那敵基督者的靈。你們從前聽見他要來,現在已經在世上了…」約翰說對耶穌和他在世上的工作的教訓必須要與神的話語一致,如果有人不承認耶穌基督是神的兒子,道成肉身來到世間,那麼他就是屬魔鬼的,他在傳播謊言。那個人已經被撒旦誘惑而墮落,並且還成了撒旦誘惑別人的工具。

 撒旦一直在想方設法讓那些屬他的子民按照他的方式來行事為人。正如你知道的那樣,神也在照他自己的計劃讓那些屬神的兒女活出他的樣式。神和撒旦都在讓屬他們的子民照他們的樣式生活,撒旦每次在你裡面建造的都是謊言,而神在你裡面建造的都是真理。我們在新約裡可以很清楚地看到這一點,使徒們在給信徒的講道中曾經多次講到舌頭。為什麼,這是因為人經常由於言語上的過失而給魔鬼留下了空子。更直截了當地說,就是撒謊。在以弗所書4:29,使徒寫到:「污穢的言語,一句不可出口,只要隨事說造就人的好話,叫聽見的人得益處。」污穢的話是什麼?就是謊話!在歌羅西書4:6裡,我們可以找到:「你們的言語要常常帶著和氣,好像用鹽調和,就可知道該怎樣回答各人。」在以弗所書裡他說, 「你們要棄絕謊言。」在歌羅西書3章裡他說,「不要彼此說謊」。

 儘管這些可能已經很明顯了,根本用不著反復再說,可我們還是想對你說如果撒旦想介入你的生活,他肯定會以一場騙局來開始他的陰謀。撒旦始終在歪曲真理,想方設法要使我們公開地否認神的真道,成為魔鬼的子民。有哪個漁夫不想讓自己釣到的魚兒更大一些呢?有哪個高爾夫球手不想用更少的杆數來贏得比賽呢?你是否也曾靜下來仔細想想撒旦一直想讓你走他的路,讓你也變成一個說謊的人呢?我們撒謊,而撒旦的品格就在我們的身上生根了。我們從本性上來說都是些愛誇口、好吹牛的人。我們喜歡讓別人為我們的工作成勣和所取得的成就而感到驚訝,總是想要把我們自己的事說得誇張一些,而只說真話實話往往很難讓我們感到滿足。我們總是習慣把我們的收入說得多一些,好讓人誇耀自己。這就是魔鬼在我們身上工作結果。有多少次由於我們的言行,人們不再輕易相信我們的話。我們和別人相約,卻又無法按照約好的時間赴約。沒有任何解釋,也沒有去個電話告訴對方自己會遲到或失約的理由。這就是在說謊,我們就這樣在模仿我們魔鬼的父親。

 我們是不是經常說謊?一個人來到我們面前說,「我需要什麼什麼,你還記得要為我禱告嗎?」「當然了,我在禱告中提到你的這件事了。」可你根本沒有去想要為他禱告,你在說謊。在我們生活的裡面到處都有各種各樣的誘惑,而我們一天一天迷失在這些惡毒的誘惑裡。當有人來電話要你去做一件你並不想去做的事情,想一想你當時用的是什麼借口吧!你的借口是實話,還是謊話?當撒旦站在你的門口的時候,他能一腳踢開門闖進去。撒旦是個大騙子,儘管他可能無法把神的真理從你那裡搶走,可他卻能夠促使你去否定真理。因此,他能夠利用你不嚴謹的舌頭闖進你的生活,在你生活的基礎上打開一個缺口。

 撒旦想要抵賴的另一個事實是耶穌基督是使人罪得赦免的唯一辦法。很多人都在宣稱自己是福音的使者,可他們所宣揚的卻是謊言,而不告訴人們必須要接受耶穌基督為他們的個人救主。他們自己提出了別的辦法。Mary Baker Patterson Glover Eddy用別的辦法。Joseph Smith也有別的辦法。Ellen G. White也有別的辦法。各地那些虛假的敬拜都是撒旦對神的真理的詆譭。它們都是撒旦的謊話,是要欺騙人們不去相信只要靠著對耶穌基督的信心就能得到救贖。

 作為宣揚神的福音的權柄的一名使者,我要告訴你耶穌基督就是道路、真理和生命;靠著對耶穌的信心你就可以得救;除了耶穌基督,根本就沒有第二條得救的辦法。接受神給你的這份福音的禮物,你的生命就會得到完全的救贖。這就是神的真道。這裡沒有謊言。我們把這份耶穌基督介紹給你,只有他才能在人面前說,「我就是道路、真理、生命。若不靠著我,沒有人能到父那裡去。」我們邀請你從心底裡接受他,你將不會再聽隨魔鬼的謊言,你會明白真理就是耶穌基督。

\chapter{撒旦誘使人偏離神的道}
\label{sec:ch06}
\hyperref[sec:ch05]{[上一章]}
\hyperlink{toc}{[返主目錄]}
\hyperref[sec:ch07]{[下一章]}

\begin{center}
\noindent\fbox{%
    \parbox{0.8\textwidth}{%
        以賽亞書5:8-23
            \newline
            5.8 禍哉、那些以房接房、以地連地、以致不留餘地的、只顧自己獨居境內。
            5.9 我耳聞萬軍之耶和華說、必有許多又大又美的房屋、成為荒涼、無人居住。
            5.10 三十畝葡萄園只出一罷特酒、一賀梅珥穀種只結一伊法糧食。
            5.11 禍哉、那些清早起來、追求濃酒、留連到夜深、甚至因酒發燒的人.
            5.12 他們在筵席上彈琴、鼓瑟、擊鼓、吹笛、飲酒、卻不顧念耶和華的作為、也不留心他手所作的。
            5.13 所以我的百姓、因無知就被擄去.他們的尊貴人甚是飢餓、群眾極其乾渴。
            5.14 故此、陰間擴張其欲、開了無限量的口.他們的榮耀、群眾、繁華、並快樂的人、都落在其中。
            5.15 卑賤人被壓服、尊貴人降為卑、眼目高傲的人也降為卑.
            5.16 惟有萬軍之耶和華、因公平而崇高、聖者 神、因公義顯為聖。
            5.17 那時羊羔必來喫草、如同在自己的草場、豐肥人的荒場被遊行的人喫盡。
            5.18 禍哉、那些以虛假之細繩牽罪孽的人、他們又像以套繩拉罪惡。
            5.19 說、任他急速行、趕快成就他的作為、使我們看看.任以色列聖者所謀劃的臨近成就、使我們知道。
            5.20 禍哉、那些稱惡為善、稱善為惡、以暗為光、以光為暗、以苦為甜、以甜為苦的人。
            5.21 禍哉、那些自以為有智慧、自看為通達的人。
            5.22 禍哉、那些勇於飲酒、以能力調濃酒的人。
            5.23 他們因受賄賂、就稱惡人為義、將義人的義奪去。
    }%
}
\end{center}

 當使徒談到歌林多教會在教訓上和道德上的不足時,他很坦白地說他們的行為虧缺了救贖他們的神和為他們死的救主的榮耀。在追溯他們那些不好的行為的最終根源時,他指明撒旦是所有這些不和、懷疑、不信、迷惑以及其他一切使人偏離神的道的元凶。在歌林多後書2:11裡保羅說,「 免得撒旦趁著機會勝過我們,因我們並非不曉得他的的詭計。」使徒保羅在對撒旦的長期爭戰中積累了豐富的經驗,當撒旦來到他們中間的時候他可以很清楚地判斷出撒旦的腳蹤。他清楚撒旦的策略,他明白撒旦的意圖,他也知道撒旦工作的方式方法。他寫信給歌林多人告訴他們他可以告訴他們關於撒旦行事的那些方式方法,免得他們像夏娃當初那樣受了撒旦的當。

 通過我們以前的學習,我們知道撒旦是個騙子。下面讓我們一起來認識以下神給我們的另一個啟示,撒旦誘使人離開神的道,讓人得不到本來神賜給人的那些祝福和好處。伊甸園裡的那一幕不僅證明撒旦不僅要曲解神的真理,而且他還要誘使人偏離神的計劃。

 在創世記3:5裡當路西非爾來到神所造的夏娃面前的時候,他試探她神是否說過要是吃了分別善惡樹的果子就要死。夏娃的回答是肯定的。接著撒旦就撒了慌,他騙夏娃說,「你們不一定死。」說完這句謊言後,他緊接著就開始誘使夏娃偏離神的道。他說,「神知道你們吃的日子眼睛就明亮了,你們便如神能知道善惡。」撒旦對夏娃說這些話的意思是,「神沒有把所有真理都告訴你們,因為他知道你們一吃了這果子就和神一樣了,神不願意有任何人和他一樣平起平坐。所以神不讓你們吃這些果子不是因為它們對你有害,而是神出於嫉妒不想讓你們分享他的一切。」

 為了讓亞當和夏娃順服他,撒旦完全曲解、篡改了神的用意。人只要順服神的旨意就可以保持和神之間的親密關係。人主動自願地依靠順服神的旨意,人和神就能永遠地擁有這種親密的關係。而撒旦卻故意違背神的意思,歪曲地解釋了為什麼神要禁止他們吃分別善惡樹上果子。從這一點上我們就可以看出撒旦是那誘使人偏離神的道的惡者。

 任何從神而來的祝福、恩賜,撒旦都想在它們上面做破壞的工作,也沒有什麼從神而來的恩賜可以遠離撒旦這種瘋狂的破壞。下面這段對食物的話就是一個例子。在創世記1:29裡神說,「看哪!我將遍地上一切結種子的菜蔬和一切樹上所結有核的果子,全賜給你們作食物。」而在2:16裡,「耶和華神吩咐他說:『園中各樣樹上的果子,你可以隨意吃。』」這就是神在伊甸園裡所定的原則:我把這裡交給你們好讓你們有吃的,你們沒有什麼限制,可以隨便吃。16節裡的重點是隨意這個詞,它強調神讓人隨著自己的心意,想吃什麼就可以吃什麼!

 看起來撒旦很難曲解這麼簡單的話,可我們在聖經裡會發現撒旦的的確確在這件事上篡改了神的話。撒旦是怎麼做的?讓我們一起翻開歌羅西書2:21。歌羅西人顯然在飲食方面有一些規矩,所以使徒保羅才會說他們有「不可拿、不可嘗、不可摸」的規條。由此可見那些已經得救的歌羅西人所守的規矩和神在伊甸園裡的話不是一致的。神究竟給過他們什麼旨意叫他們不吃某些東西?歌羅西教會裡慢慢滋生出一個限制歌羅西人飲食的體系。這些外邦人所想的還是要去守神給以色列民族的律法,雖然他們已經不再在摩西律法以下,可他們卻還在受到這些律法的束縛。在他們中間,屬靈與否還要從他們的吃喝上來判斷。使徒說歌羅西教會裡的屬靈生活正在敗壞,因為那裡的人被撒旦的工作所蒙蔽了。

 我們可以從提摩太前書4:3,4裡發現同樣的事。假教師已經進入到了提摩太所管理的以弗所教會。「他們禁止嫁娶,又禁戒食物,就是神所造,叫那信而明白真道的人感謝著領受的。凡神所造的物都是好的,若感謝著領受,就沒有一樣可棄的;都因神的道和人的祈求,成為聖潔了。」這裡又是一個對神的旨意的曲解,在4章1節裡它被稱為「鬼魔的道理」。這些魔鬼和邪靈在講些什麼呢?他們在講基督徒的屬靈生活要遵守一定的飲食上的規定,如果你想聖靈充滿、想在靈性上成熟,就必須吃這個,不能吃那個。保羅說這些都是鬼魔的道理。撒旦想幹什麼?他是在誘使神的兒女得不到天父豐富的恩典。

 在飲食方面,撒旦還有另一種破壞的方法。我們從彼得前書4:3裡可以讀到,「因為往日隨從外邦人的心意行邪淫、惡欲、醉酒、荒宴、群飲,並可惡拜偶像的事,時候已經夠了。」我在這裡想特別強調的是這一節中關於放蕩飲食的部份。撒旦對他們做了些什麼呢?他使得那些人暴飲暴食,更改了神給人飲食的初衷。讓我們一起來看一看申命記21:20裡摩西的命令。父母要對本城的長老這樣說,「我們這兒子頑梗悖逆,不聽從我們的話,是貪食好酒的人。本城的眾人就要用石頭將他打死。這樣,就把那惡從你們之間除掉….」知道嗎?在摩西律法下,神把貪食的罪看得和姦淫、殺人的罪一樣重。在箴言23:1,2裡,所羅門這樣教導說,「你若與官長坐席,要留意在你面前的是誰。你若是貪食的,就當拿刀放在喉嚨上。」這裡說得已經十分清楚了。他說寧肯把刀子放到喉嚨上你也不要吃得臃腫不堪。在這一章的20、21節裡還有:「好飲酒的,好吃肉的,不要與他們來往,因為好酒貪食的,必致貧窮;好睡覺的,必穿破爛衣服。」在腓立比書3:19裡保羅對那些腓立比人說,「他們的結局就是沉淪,他們的神就是自己的肚腹…」

 我們現在在說些什麼呢?神在創造的時候把食物作為一種祝福賜給了人,好叫人的身體能夠得以支撐。那撒旦使人對神的這種恩典到底怎麼樣了呢?他破壞了神給人食物的初衷。人類一直在同肥胖的問題做斗爭。心臟病、高血壓以及其他許多種疾病都和肥胖密切相關。人棄絕了恩典的生活而沉浸于飲食當中,正好跟從了撒旦。無論是在主日的餐桌上和週末的酒館裡,你都可能順了撒旦的心意。

 神給人的另一個恩典就是從葡萄中可以榨取出的汁液。這本是神的恩典,可魔鬼為了他個人的目的在人中間曲解了神的意思。這本是神給人的一份恩賜,使徒保羅在提摩太前書5:23裡也說,「再不要照常喝水,可以稍微用點酒。」顯而易見,作為使徒保羅傳播福音的同伴,提摩太的身體一直都不太好。他追隨使徒保羅走遍了小亞細亞,而他的胃口一直都給他帶來許多困擾。保羅說有種藥對他這種病很有好處。這種藥就是葡萄酒,它對痢疾有效,而其他治療痢疾的藥卻很少。提摩太為什麼對使用葡萄酒如此謹慎呢?讓我們回到第三章,使徒保羅在這裡為作監督和作執事的資格定下了標準。保羅在3:3裡說做監督的不能因酒滋事,他要求做監督的必須完全戒酒。在談到做執事的資格的時候,保羅再次講到執事不能被酒所左右,成了酒的奴仆。提摩太把這些教訓都謹記在心底,於是他寧肯受到痢疾的折磨,也不願把葡萄酒當作藥來服用,免得有人因此而覺得他沒有做監督和執事的資格。保羅的觀念是凡是神所賜的都是一種祝福和恩典。

 可是保羅的這種觀念竟然被有些自稱為屬基督的和其他一些人篡改、扭曲了!在我們當今社會裡,由酗酒以及販賣和使用酒精飲料所引發的種種問題已經成了我們所面臨的最大的問題之一,它體現在經濟和社會的各個方面,而我們自私的貪欲和從它而來的稅收卻使得我們對這個問題視而不見。神的兒女一不小心就跌進了撒旦的圈套。在我們的成長過程中我們對酗酒都不會感到太陌生,我們對他都已經適應了。當我們沒有意識到這是撒旦又一個詭詐的圈套,他為了他自己的目的再次破壞了神給人的這一祝福和恩典。

 這就是為什麼使徒在以弗所書5:15裡提醒眾人不要醉酒而要被聖靈充滿的原因。一個人如果不憑著信心,靠著聖靈來引導他經歷那些艱難險阻,酒精就會取代聖靈在他身體中所起的作用。而那些依靠酒精來面對試探的人永遠也不會懂得如何靠著信心和聖靈來行在這個世界上。你想路西非爾會願意你靠著信心和聖靈行在這個世上嗎?他絕對不會願意。那他要怎樣去破壞你的信心生活呢?他會替換神給人的這份恩典,說「別靠著聖靈來跟隨神給你指引的道路,讓酒精來給你加添點力量吧。」這就是他要做的破壞的工作。

 那麼我們生活中的那些道德規範呢?我們從這裡可以再一次看到撒旦破壞的工作。兩性關係中所謂的新道德或自由主義再次向我們顯明瞭撒旦工作的方法。使徒保羅說神給人滿足身體需要的方法是通過婚嫁而產生的夫妻關係。在歌林多前書7章使徒很清楚地表明婚姻關係是神所應允的滿足肉體情慾的方法。而神給人的這一最大的祝福和恩賜卻在撒旦的破壞下扭曲了。使徒保羅在羅馬書1:25-27節裡清楚地寫到:「他們將神的真實變為虛謊,去敬拜山峰受造之物,不敬拜那造物的主。主乃是可稱頌的,直到永遠。阿們。因此神任憑他們放縱可羞恥的情慾。他們的女人把順性的用處變為逆性的用處;男人也是如此,棄了女人順性的用處,慾火攻心,彼此貪戀,男和男行可恥的事,就在自己身上受這妄為當得的報應。」

 所有這些破壞的工作,這些淫蕩、亂交,都是撒旦破壞神給人的祝福和恩典的明證。撒旦把它們搶奪過來,歪曲地對它們進行解釋,使它們違背了神原來的用意,製造出各種不潔淨、不道德的事情來。為此,使徒不得不一封又一封得寫信去幫助教會對付撒旦的工作。他反復強調關於婚姻的法則,惟恐做丈夫或做妻子的對對方不忠;為了不讓少年人墮入撒旦的圈套,犯了聖經上所禁止的跟亂倫一樣的婚前性行為的罪,他也提到該如何去做。跟我們的主在馬太福音19章裡做的一樣,他也不得不在歌林多前書7章裡又談到關於離婚的問題,因為撒旦也在這事上做了他破壞神的恩典的工作。

 有許多我們認為的神的祝福已經被撒旦所竊取,並按照他的私慾被扭曲了,背離了神的道。我們在這裡所提到的只是其中的一小部份。不管神對人有什麼樣的恩典和祝福,無論是物質上的擁有還是人的聰明才智、社會的傳統、物質財富,撒旦都要對它們進行破壞。不管什麼都能成為他要使用的工具。

 撒旦不僅能破壞我們肉體上的這些事情,他還能破壞神的兒女與神之間的關係,使人背離神的旨意,順從了他的意思。一說到這裡,我們自然就想到了約拿。這位先知一生都在服侍主,榮耀主的名。神啟示他到外邦人中傳揚神要來審判的信息,而約拿卻違背神的意思去到約帕,在那裡他坐著小船想儘量地遠避神。約拿為什麼會這樣做呢?我們從約拿書第四章裡找到答案。他知道他如果去傳揚神的話,那麼神一定會祝福那些聽他的話的人。他知道如果他去傳講主的道,神就會使那些外邦人認罪悔改,神的審判就不會臨到這些外邦人的身上。約拿對神要把祝福降給外邦人這件事並不太高興。他希望那些外邦人受到神的審判而不是也得到神的祝福。撒旦此刻就在約拿的身上動了他邪惡的工作,他對約拿說,「神要你做的太多了。神要你放棄的也太多了。你何必非要去做這件事呢?」撒旦就這樣歪曲了神給約拿的旨意,使得約拿最終跟從了撒旦為他設計的歪路。

 我們從路加福音第九章也可以發現類似的事。在23-26節裡,主耶穌給那些自稱為他的門徒的人一種選擇。耶穌說,「若有人跟從我,就當舍己,天天背起他的十字架來跟從我。因為凡要救自己生命的,必喪掉生命;凡為我喪掉生命的,必救了生命。」這些人不得不做出他們的決定:到底是要做基督的門徒還是法利賽人的門徒;到底是接受基督還是反對基督。其結果正如57節裡記下的那樣有些人來到基督面前,一個人說,「你無論往哪裡去,我要跟從你。」我們的主於是提醒他說他沒有居所,甚至連飛鳥、狐狸也趕不上。那個人聽了就離開了,因為他覺得跟隨基督要付的代價實在太大了。又有一個人說,「主,容我先回去埋葬我的父親。」此刻他的父親還沒有死去。要是耶穌答應他等到他的父親去世後再來跟從他,這個人就能得到他應該分得的財產。到時他就不必付出太多的代價了。可當主說「跟從我來」的時候,他也掉轉過頭去。基督的旨意對他來說要付出的代價實在太大了。又有一個人來到耶穌面前說,「主,我要跟從你,但容我先去辭別我家裡的人。」耶穌回絕了他。這個人對地上的事有太多牽掛。還記得14章裡那些人推辭的借口嗎?有人說他剛購置了些財物,需要照管;有人說他剛買了一些牛,要去試一試;還有人說他剛娶了妻子不能去。

 這些本想跟隨耶穌基督的人怎麼會找出這麼多的借口?這都是因為撒旦向他們歪曲了神給他們的旨意,讓他們以為如果他們跟從了基督他們就得丟下一切物質的東西,而耶穌也不會給他們任何東西。撒旦鼓惑他們說他們應該為了他們的安逸和享樂先積攢一定的物質財富,然後再考慮他們該為主做些什麼。這種想法就是撒旦對神的旨意歪曲、謬解的產物。那個年輕的大財主來到耶穌面前顯然是為了要尋求他腦海中那個大難題的答案,也就是他如何才能得到永遠的生命。這個年輕人相信的是他的財富,因為他已經很富有了。而耶穌基督卻要他丟棄他的財富,把他所有的都分給別人然後再來跟隨他。這個財主就離開了,因為他很富有;他的心裡充滿了懮傷,因為他的財富擋住了他與主之間的關係。

 我們通過這些事情想對你說的就是撒旦會來到人的面前,對這人說如果你把基督放在你生活的首要位置,你就會失去一些東西,你就會失敗。他對經商的人說如果他們按照基督徒的原則來做生意,他的那些競爭對手們就會勝過他們,如果他恪守誠實的心來經營,那麼他就會失敗。這就是撒旦想要告訴你的:神的旨意執行起來要付上的代價太大了。多少次當耶穌基督要人聽從、順服他的時候,人們因為考慮要付上的代價而撇開頭去?他們都跟從了撒旦歪曲的道路。

 撒旦還會謬解神聖潔的標準。這就是先知以賽亞在5章裡所談的問題,可以用20節總結來說就是,「禍哉!那些稱惡為善,稱善為惡,以暗為光,以光為暗,以苦為甜,以甜為苦的人。」我們現在生活的這個時代有兩個倫理道德規範。一個就是神的話,聖經反映了神的聖潔的品性。這本聖經向我們揭示了神以及他對那些要與他同行的人的期望。神有一個倫理、道德、行為標準的體系。而撒旦把這些標準進行了歪曲和謬解,使得這個世界上為人們所接受的標準與神的標準不一致。人把善的當作惡的,把惡的當作善的,為達到目的不擇手段。人與人之間的不信任正在動搖著我們國家的根基。有些位居高官的人採取了撒旦的雙重標准。這些不僅僅在搞政治的人中間存在,也表現在其他各個領域裡。都是因為撒旦做的敗壞的工作,人才把善的當作惡的,把惡的當作善的。

 撒旦還會在救贖的問題上做歪曲、謬解的工作。神只有一個標準。人只有像神那樣的公義,才能見主的面。神給這個世界提供了一位拯救者。基督可以向他自己說,「我就是道路、真理、生命。若不藉著我沒有人能到父那裡去。」撒旦為了歪曲、謬解這條看來極為簡單的真理而在世上散佈了許多中哲學思想和教訓,而所有這些東西都是否定耶穌基督的存在和他的工作的。許多人在撒旦的蠱惑下相信他們並不需要耶穌基督的救贖,這些人都在通向地獄的路上行走,他們的盡頭就是地獄,而他們自己卻並不知曉撒旦這些敗壞的工作就是要他們同神永遠分離。

 對於信徒們來說,我們很有必要去了解撒旦的那些詭計,這樣我們才能防備不上他的當。撒旦是個大騙子,他謬解了神給世人的一切美好的祝福。使徒保羅希望歌林多教會能夠明白他在歌林多前書12-14章裡的話,當他們從神那裡得到任何恩賜的時候,有必要先確定一下這些恩賜到底是在主權的控制下,還是已經受到了魔鬼的轄制。我們能夠想到的那些恩賜沒有一樣是可以避免撒旦敗壞的工作,不被他所歪曲、謬解的。不管一件事物有多好,撒旦都能歪曲它、謬解它,如果我們容忍下去,他還會最終毀壞它。不過靠著神的恩典,我們可以覺察出並阻止撒旦敗壞的工作。

\chapter{撒旦,造假者}
\label{sec:ch07}
\hyperref[sec:ch06]{[上一章]}
\hyperlink{toc}{[返主目錄]}
\hyperref[sec:ch08]{[下一章]}

\begin{center}
\noindent\fbox{%
    \parbox{0.8\textwidth}{%
        歌林多後書11:1-15
            \newline
            11.1 但願你們寬容我這一點愚妄.其實你們原是寬容我的。
            11.2 我為你們起的憤恨、原是 神那樣的憤恨.因為我曾把你們許配一個丈夫、要把你們如同貞潔的童女、獻給基督。
            11.3 我只怕你們的心或偏於邪、失去那向基督所存純一清潔的心、就像蛇用詭詐誘惑了夏娃一樣。
            11.4 假如有人來、另傳一個耶穌、不是我們所傳過的.或者你們另受一個靈、不是你們所受過的.或者另得一個福音、不是你們所得過的.你們容讓他也就罷了.
            11.5 但我想、我一點不在那些最大的使徒以下。
            11.6 我的言語雖然粗俗、我的知識卻不粗俗.這是我們在凡事上、向你們眾人顯明出來的。
            11.7 我因為白白傳 神的福音給你們、就自居卑微、叫你們高升、這算是我犯罪麼。
            11.8 我虧負了別的教會、向他們取了工價來、給你們效力。
            11.9 我在你們那裡缺乏的時候、並沒有累著你們一個人.因我所缺乏的、那從馬其頓來的弟兄們都補足了.我向來凡事謹守、後來也必謹守、總不至於累著你們。
            11.10 既有基督的誠實在我裡面、就無人能在亞該亞一帶地方阻擋我這自誇。
            11.11 為甚麼呢.是因我不愛你們麼.這有 神知道。
            11.12 我現在所作的、後來還要作、為要斷絕那些尋機會人的機會、使他們在所誇的事上、也不過與我們一樣。
            11.13 那等人是假使徒、行事詭詐、裝作基督使徒的模樣。
            11.14 這也不足為怪.因為連撒但也裝作光明的天使。
            11.15 所以他的差役、若裝作仁義的差役、也不算希奇.他們的結局、必然照著他們的行為。
    }%
}
\end{center}

 撒旦不僅是說謊者,也是個大騙子,歌林多後書11:13-15節裡說他還是做假的人。創世記3章記著撒旦對夏娃說,「你們吃的日子眼睛就明亮了,你們便如神能知道善惡。」撒旦哄騙夏娃她說要要讓她儘可能地像神一樣,但它的前提條件是離開神。撒旦的計劃和他實行的過程總是想要去模仿神,他騙人去按照他的計劃行事,並讓他們相信他們在跟從他的模仿的同時就是在跟隨神的腳蹤。

 在歌林多後書11:13-15裡使徒保羅寫到:「那等人是假使徒,行事詭詐,裝作基督使徒的模樣。」注意這一句,「裝作基督使徒的模樣。」「這也不足為怪,因為連撒旦也裝作光明的天使。所以他的差役,若裝作仁義的差役,也不算希奇。他們的結局必然照著他們的行為。」在保羅寫給提摩太的信裡,他在提摩太前書6:16裡談到了永遠的神,那可稱頌的、獨一的真神,萬王之王,萬主之主,那唯一不死的,保羅接著說:「住在人不能靠近的光裡,是人未曾看見,也是不能看見的。」聖經很清楚地表明沒有人能夠看見永恆的神的真面目。保羅把這作為一個絕對真理,神住在光裡,沒有人曾看見也沒有人能看見神。當摩西要看神的面容的時候,神提醒摩西說(出埃及記33章)人見了他的面沒有能活著的。但神又說他要把摩西放在磐石穴中,用他的手遮掩摩西,然後神還要將他的手收回,好使摩西能看得見神的背影,他的榮耀,即使是摩西也無法看見神的面容。摩西按照神的辦法看到了神的大榮耀,心滿意足地回去了。

 根據這一原則,我們可以推出當亞當和夏娃在伊甸園中與主同行的時候,與亞當和夏娃交談的神被一團光包圍著。他們倆一看見神榮耀的光輝,就知道神就在他們中間。以後在會幕和聖殿裡的情況也大致相同。以色列人看見遮蓋在會幕上的雲彩和充滿了帳幕的榮光,以及夜間的火柱就知道神與他們同在了。在高山上看見耶穌改變形像的那三個人知道神在基督裡面並通過基督顯現出來,這是因為他們看到了神的榮耀,突然出現的那「一朵光明的雲彩」充滿了神的榮耀和真理。

 神來到亞當和夏娃面前用光遮住他們,他們就因神的出現、神的甜美和與神之間的親密關係而歡喜快樂。神從榮耀的光中向他們說話。在他們交談的時候,神的聲音變得親密而熟悉。但當夏娃再次見到一道光亮的時候,從那光中發出的聲音卻讓夏娃覺得有些陌生。顯然這就是使徒保羅在寫歌林多後書11:14時所想到的,「連撒旦也裝作光明的天使。」我們認為使徒保羅在這裡說的就是當時撒旦來到夏娃面前時的形像。他並不是通過田野中蛇的身體來的,他利用了蛇的身體,而後又用一團光芒遮掩住自己,假扮成神再次來到園中的樣子來騙夏娃的。夏娃沒有對這蛇產生懷疑,因為蛇和控制這蛇的撒旦都藏在亮光的後面。就這樣撒旦來到了夏娃的面前,夏娃以為向她而來的那光和對她說的話都是神的旨意,是神要她做的事,是要她享受和神之間的美好關係,因此神又一次以光的形像來到她的面前。但是夏娃這一次聽到的和她以前聽過的不一樣,這聲音要她去吃那果子好使她能跟神一樣。正是這光的出現才使得夏娃相信那是神要她吃的。撒旦就這樣靠著假扮神的樣子欺騙了在園中的受造之物。他把自己裝成是光明的天使。

 夏娃的心裡似乎從來也沒有想到過神不會說出自相矛盾的話;神自己剛剛說過「你不可吃,因為你吃的日子必定死」,他不會現在有改口說「你們不一定死,因為神知道,那麼吃的日子眼睛就明亮了,你們便如神能知道善惡。」夏娃整個的心思都被那裝作光明的天使的撒旦給騙住了。而在今天,那在園中假扮神的撒旦仍然在用同樣的伎倆在行騙。撒旦仍然把他自己假扮成光明的天使,而撒旦的那些差役也學著他的樣子裝扮成仁義的樣子(歌林多後書11:15)。在他們宣揚那些給罪人帶來心靈上的死亡和黑暗的時候,他們把自己說成是光明神的使徒,他們的教訓都出自于光明的神。這些都是撒旦騙術的一部份。

 我們可以從聖經中找到好幾處揭示這一道理的地方。你是否還記得當摩西被神派遣要到法老面前讓他放走以色列人的時候,摩西對此感到有些懷疑,因為他想他連以色列人都勸服不了,法老又怎麼會聽他的呢。「他們若問我說:『他叫什麼名字?』我要對他們說什麼呢?」於是神對他說,「我是自有永有的。」是他要你去的。摩西想到要去使那些頑梗的人相信他、聽從他的話很難,就又向神求教。於是神給摩西顯現神跡的權柄,讓他代表神行事。從出埃及記7:11,12裡我們可以讀到當摩西顯現第一個神跡,把杖變成蛇之後,法老就把博士和術士召來,「…他們是埃及行法術的,也用邪術照樣而行。他們工人丟下自己的杖,杖就變作蛇…」在22節也寫到假扮的事,「埃及行法術的,也用邪術照樣而行。」注意到發生了什麼嗎?神派了一個使者去,要把他的兒女以色列人解救出來,使他們重得自由,他用神跡給他的命令加添了權柄。可撒旦和他的差役、那些術士,也假冒了摩西所施的神跡。結果,法老在看了那些術士假冒的神跡之後,心裡反而剛硬了,「不容以色列人去。」他不肯聽摩西和神的話,這都是因為撒旦假冒神跡欺騙了法老和埃及人。

 保羅的經歷也是個例子。使徒被神派到外邦人中宣揚神的恩典和福音。在保羅的第一次傳道旅途中,他來到一個叫帕弗的地方。「…在那裡遇見一個有法術、假充先知的猶太人,名叫巴耶穌。這人常和方伯士求保羅同在。士求保羅是個通達人,他請了巴拿巴和掃羅來,要聽神的道。只是那行法術的以呂馬敵擋使徒,要叫方伯不信真道」(使徒行傳13:6-8)。發生了什麼事?保羅遇到了一位渴望神的真道的人,於是就把神恩典的福音介紹給他。可撒旦和他自稱為公義的差役們卻出來攔阻,他們對這人說他聽到的不是真道而是謊言。

 啟示錄13章對此講述地更明白。在苦難的時候有獸被奉為敬拜的對象,而操縱那售的假先知「…有行大奇事,甚至在人面前,叫火從天降在地上(假冒以利亞的神跡)。它因賜給它權柄在售面前能行奇事,就迷惑住在地上的人,說:『要給那受刀傷還活著的售做個像。』又有權柄賜給它,叫售像有生氣,並且能說話,又叫所有不拜獸像的人都被殺害。」當神派兩個證人去宣告審判臨到耶路撒冷,並用羔羊的血洗淨他們的時候(啟示錄11:3-12),撒旦也把他的一個差役變作公義的使者的樣子,並讓他假冒神所顯的神跡,就是以利亞和摩西所顯過的神跡,好使這個世界相信他所說的才是真道。神派了使者帶著權柄和神跡去宣揚神的旨意,而撒旦也派遣他的使者去施行了同樣的神跡去欺騙世人,不讓他們聽信神的真道。

 撒旦的假冒不止于此。在使徒行傳1:8裡神對使徒說,「…要在耶路撒冷、猶太全地,和撒瑪利亞,直到地極,作我的見證。」你要為我作見證!你通讀使徒行傳的時候你會發現使徒們走遍全地宣揚的只有一個人;他們講的是福音,基督為我們的罪而受死的道理,使徒們所傳講的都是在圍繞著耶穌基督的人和他在世上的工作。除了耶穌和他被釘十字架之外,他們什麼也不知道。但當撒旦看到神派人做他的見證的用意之後,他也派人出去傳講否定這一真理的東西。請留意以下歌林多後書11章中幾處和此有關的內容。在第三節,保羅對歌林多人說,「我祇怕那麼的心或偏于邪,失去那向基督所存純一清潔的心,就像蛇用詭詐誘惑了夏娃一樣。假如有人來(請注意),另傳一個耶穌,不是我們所傳講過的;或者你們另受一個靈,不是你們所領受過的;或者另得一個福音,不是你們所得過的;你們容讓他也就罷了。」靠著另一個靈的能力去傳講另一個耶穌,宣揚另一個福音,這在提摩太前書4章裡也記載。保羅說,「聖靈明說,在後來的時候,必有人離棄真道,聽從那迷惑人的邪靈和鬼魔的道理。」這種所說的魔鬼的道理就是指魔鬼要來假冒跟耶穌和他在世上的工作有關的真理。

 別忘了,魔鬼也在宣揚道理。他們要宣揚的是魔鬼的道。他們有另一個福音、另一個耶穌,他們渴望借助人把邪靈的能力表現出來。這些魔鬼散播邪靈的活動很猖獗。在約翰壹書4:1裡,約翰也向我們教導了這一真理。他說,「一切的靈(或老師),你們不可都信,總要(用聖經的標准)試驗那些靈是出於神的不是;因為世上有許多假先知已經出來了。凡靈認耶穌是基督是成了肉身來的,就是出於神的,從此你們可以認出神的靈來。凡靈不認耶穌,就不是出於神,這是敵基督者的靈。你們以前聽說他們要來,現在已經在世上了。」

 讓我們返回到2:22 ,約翰在這裡說了同樣的話,「誰是說謊話的呢?不是那不認耶穌為基督的嗎?不認父與子的,這就是敵基督的。」保羅和約翰都估計到撒旦下一步的活動,也就是撒旦要讓這個世界都充滿了敵耶穌基督的勢力,叫傳道人處處為難。撒旦可能會在其他任何方面跟你妥協,但他在任何情況下也絕不會承認使徒們所傳講的主耶穌基督死而復活的真理。因為基督和他在世上的工作是整個福音的核心所在。所以,保羅和約翰告戒教會說會有人來傳講另一個福音,宣揚另一個基督,讓人領受另一個靈。

 那麼到底該留意些什麼呢?神與人的溝通是通過傳道,宣揚耶穌基督的真理來實現的。而撒旦也搞不出別的什麼花樣,他也模仿神的方法,派出他那些自稱為公義的差役,叫他們去模仿從神而來的聖徒們去傳道。但他們靠的是撒旦的能力而不是神的能力,他們傳的是另一個福音,他們的目的是要用謊言來迷惑人,欺騙人,要把真理從人的心思意念裡除掉。

 保羅在他寫給加拉太教會的書信裡強調了這一點。在加拉太書1:6-9裡他說,「我希奇你們這麼快離開那藉著基督之恩召你們的,去從別的福音。那並不是福音,不過有些人攪擾你們,要把基督的福音更改了。但無論是我們,是天上來的使者,若傳福音給你們,與我們所傳給你們的不同,他就應當被咒詛。我們已經說了,現在又說,若有人傳福音給你們,與你們所領受的不同,他就應當被咒詛。」保羅在這裡所說的那些傳另一個福音的應當被咒詛的語氣非常強烈,希臘文裡再沒有什麼比保羅在這裡用的更強烈的了,而且保羅在這裡反復說了兩次(8、9節)。保羅才心裡很清楚撒旦的那些詭計,他知道撒旦要模仿神的樣子,派他的差役到各地假冒神差遣的名傳講假福音。

 我們需要注意一個人們常有的誤解。當我們看到貧民窟裡那些廢墟的時候,我們會說,「看撒旦干的好事!這又是他的傑作!」我不太同意這種觀點。撒旦跟你們一樣,也非常討厭那些東西。雖然這些確實都和撒旦相關,可他也不願意去碰它們。撒旦真正的傑作是那些那些好人,那些正直、誠實、值得尊敬、受人愛戴而又覺得他們不需要耶穌、不需要神,以為不必靠神的幫助,只靠他自己的力量就能夠得到救贖的人。這種人才品格優秀,又有很好的聲譽,可他不接受神在他的生命裡,這種人才是真正讓撒旦高興的人,他們才是撒旦的傑作。撒旦始終都在使他們的子民儘量地學像神的樣式,但這些人根本沒有信心,不把耶穌基督作為他們個人的救主。

 撒旦也在始終努力追求完美。他在想法造一個跟基督毫不相干而又完美的人。接著他就可以說,「你很像神,你若跟著我就可以真的像神那樣。」進化的方向是什麼?不就是要通過自然的過程,在控制外界環境和遺傳的條件下創造出一個完美的人類的樣本嗎?按照這種觀點,人看起來好像是在沒有神的恩典的幫助下,靠著自己的力量在不斷完善自己。進化論是撒旦所宣講的理論,他的目的就是要使人想人類和神差不多,不需要神的真道或是和神的關係等等,這樣就達成了撒旦的心意了。

 最近我們讀到在加洲出現一種撒旦式的婚禮,新娘在結束時要身穿猩紅色的裝束。婚禮當中會有個赤身露體的女人會躺在在聖壇上,這是表示要他們過縱慾的生活而不是謹守自己的身體。兩個人一起來到壇前奉撒旦的名舉行婚禮。這一新的撒旦教的最明顯的特徵就是在他們沒有罪的概念,因此他們也就都當自己沒有罪。創立這一教派的人正在寫一部新的聖經,而這部聖經將是撒旦聖經。聽了這個消息有誰會不吃驚呢?你認為撒旦會不會為此而驕傲呢?不。他會感到羞辱的。他會跟所有正常的神的兒女們一樣為此感到羞辱,因為這樣就暴露了撒旦的本來面目。撒旦肯定不願意讓人看清這一點。他一定會想辦法遮蓋住他的樣子,把這種醜行藏起來的。

 那麼撒旦會為什麼而驕傲呢?是那些裝作光明的使者的他的差役。在德州達拉斯的一家大飯店裡,一些宣揚主的福音的代表們聚集到一起討論基督徒的教育問題。有個人站起來對來自全國各地的代表們說把孩子送到主日學校裡去學習實在是再危險不過了,因為孩子們長大以後就會意識到他們不該再相信小時在那裡學過的聖經故事。所以說把孩子送到主日學校裡學習實在是太危險了。這個人不是什麼公義的使者,他是鬼魔的差役。有個主教一再公開地否定聖經裡的每條主要原則,後來他所在的那個支派實在忍受不了就把他驅逐出去。後來他來到德州的一所大學,他對年輕學生們說:「別讓任何人或任何人告訴你罪是什麼。你應該自己決定你的罪到底是什麼。只有當你自己認為那件事對你來說是種罪,那才是一種罪。如果你覺得它不是,那它就不是。」他是鬼魔的差役,是裝作光明的使者來到這世上的。

 還有個人,他自稱自己是迪斯科牧師,他整天從這個酒吧逛到另一個酒吧說是給人傳福音,他公開說:「我不想用耶穌基督的道。我搞不懂怎麼會道成肉身。我想談談關於耶穌的事,因為我能理解平常的人。」看,這個人雖然被立為光明的使者,可他所說的卻像是出自魔鬼的兒女口中的話。

 那些明白主的道並且敬畏聖經權柄的人需要對那些自稱是耶穌基督的使者的人進行一番檢驗,看看他們是否能通過下面這個基本的攷察:他對耶穌基督和他在這世上的工作是怎麼個態度?不分良莠地去聽福音廣播是危險的,因為撒旦是空中掌權者,他能控制那些自詡為公義的使者的人散佈一些不敬畏神的話。聖經裡命令說你要檢驗那些自稱為基督福音的使者的話。用聖經的話來檢驗他們,因為聖經就是標準,是權柄。如果一個人對聖經的權柄、完整性、無誤性等有懷疑,他所教導的沒有聖經方面的依據,那麼他就已經成了撒旦的差役。如果有人否認耶穌基督為人的罪而死,他的死可以救贖世人的話,不管他有什麼樣的關係,處在什麼樣的位置上,他都是撒旦的差役,他是黑暗裡的墮落天使,裝作光明的天使來到這個世上。如果有人否認耶穌基督是永恆神的兒子,他道成肉身來救贖罪人,不管他的言語有多麼虔誠,不管他在教會中得到多少人的肯定,他仍然是撒旦的差役。撒旦每時每刻都在精心地假冒真理,做著說謊欺騙的勾當。

 或許你們當中有人現在才知道原來牧養自己的不是那光明的使者,而是黑暗的使者。我們願意讓聖經的權柄來見證世上只有一位救主,他就是耶穌基督;救贖的計劃只有一個,那就是耶穌基督要為你的而罪;你只有一個辦法能得到神的拯救,那就是接受耶穌基督為你個人的救主。這就是耶穌基督光明的使者要告訴你的。我們願意把這位救主介紹給你。你願意相信他並接受他為你的救主嗎?

\chapter{撒旦,不法的人}
\label{sec:ch08}
\hyperref[sec:ch07]{[上一章]}
\hyperlink{toc}{[返主目錄]}
\hyperref[sec:ch09]{[下一章]}

\begin{center}
\noindent\fbox{%
    \parbox{0.8\textwidth}{%
        帖撒羅尼迦後書 2:1-12
            \newline
            2.1 弟兄們、論到我們主耶穌基督降臨、和我們到他那裡聚集、
            2.2 我勸你們、無論有靈、有言語、有冒我名的書信、說主的日子現在到了、〔現在或作就〕不要輕易動心、也不要驚慌。
            2.3 人不拘用甚麼法子、你們總不要被他誘惑.因為那日子以前、必有離道反教的事.並有那大罪人、就是沉淪之子、顯露出來.
            2.4 他是抵擋主、高抬自己、超過一切稱為 神的、和一切受人敬拜的.甚至坐在 神的殿裡、自稱是 神。
            2.5 我還在你們那裡的時候、曾把這些事告訴你們、你們不記得麼。
            2.6 現在你們也知道那攔阻他的是甚麼、是叫他到了的時候、纔可以顯露.
            2.7 因為那不法的隱意已經發動.只是現在有一個攔阻的、等到那攔阻的被除去.
            2.8 那時這不法的人、必顯露出來.主耶穌要用口中的氣滅絕他、用降臨的榮光廢掉他。
            2.9 這不法的人來、是照撒但的運動、行各樣的異能神蹟、和一切虛假的奇事、
            2.10 並且在那沉淪的人身上、行各樣出於不義的詭詐.因他們不領受愛真理的心、使他們得救。
            2.11 故此、 神就給他們一個生發錯誤的心、叫他們信從虛謊.
            2.12 使一切不信真理、倒喜愛不義的人、都被定罪。
    }%
}
\end{center}

 一個人去照相館照相的時候他總會告訴攝影師從這個或從那個角度去拍照,因為人們相信他從某個角度看上去會更好看一些。而攝影師可能回有不同的想法並作出不同的判斷。這個人會選擇他覺得更為滿意的方案來拍照。

 但無論從什麼角度來看,撒旦都沒有好的一面。他是個騙子,滿口謊話,他到處行詭詐,弄虛作假。我們接下來願意把我們從聖經裡看到的關於撒旦的另一個方面與你分享,這就是聖經說撒旦是不法的人。

 使徒保羅在帖撒羅尼迦後書2章裡提醒我們說撒旦要在我們苦難的日子來到這地上進行他的計劃,這世上要來大罪人,沉淪之子,或者更準確地說是不法的人。這不法的人要在先知的事上做許多工作,僅此就證明瞭撒旦和他陰謀的本質。除非我們認識到那個說謊者、大騙子也是個不法的人這個事實,否則我們不會明白我們每天都在和什麼在爭戰,我們也不會明白當今世界裡許多發生在我們週圍的事。

 撒旦是神造的,是眾天使中的一個。他的存在是因為神口中所出的話。從本質上來說,撒旦是個受造之物,他有責任去順服創造他的主的權柄。使徒保羅在羅馬書9:20裡說得很清楚,由那些複雜的推選的原則可以知道,受造之物根本沒有權力對創造主說,「你為什麼把我做成這樣?」很久以前,到底多久我們也不知道,路西非爾一直在他受造的位置上。作為天使長,他是神的意願的執行者,管理著眾天使。當他行使神的能力的時候,榮耀和光明包裹著他的全身,他是所有受造之物中最美的一個,他的權柄在眾天使之上。但他一想到在他的上面還有一位神,他就變得煩躁不安。他對屬於神的權柄垂涎三尺,而且他還想篡奪神獨一無二的位置。當路西非爾看到神獨立於萬物,而他自己卻要順服神,他就在他的心底裡說,(以賽亞書14:14)「我要與至上者同等。」這句話成了路西非爾背叛神的座右銘。但他永遠也不能像創造主那樣,因為他本身是個受造之物。他不能像神那樣有永遠的生命,因為他的生命是被創造出來的。他只能在一個方面像神,那就是不靠別人只靠自己。因此,他在這種貪欲的鼓動下,帶領無數天使起來反叛神的權柄。神對此有這樣的評述。

 凡看見你的,都要定睛看你,留意看你,說:「使大地戰抖,使列國震動,使世界如同荒野,使城邑傾覆,不釋放被擄的人歸家,是這個人嗎?」(以賽亞書14:16,17)

 要去弄明白以賽亞到底在說些什麼,我們有不要再來想想撒旦誘使夏娃上當的詭計。撒旦來到夏娃面前查問她對神的道理認識。他對神的愛提出疑問,說神是出於嫉妒不想讓別人分享他的位置,所以神才會不叫夏娃吃那分別善惡樹的果子,不讓他們得到這份恩惠。他還對神完美的品格提出疑問。結果,「女人見那棵樹的果子好作食物,也悅人眼目,且是可喜愛的,能使人有智慧,就摘下果子來吃了;有給她丈夫,她丈夫也吃了。」夏娃犯的第一條罪就是她對神的品格產生了懷疑。這種懷疑最後導致了背叛;夏娃因此背離了神,成了這地上第一個背叛神的人。亞當隨後立刻也加入到這種背叛中來。當亞當和夏娃的子女降生的時候,他們也生來就成了叛逆神的人。這種背叛在他們長大後表現為該隱殺死了他的兄弟亞伯。這是不法的罪。

 從創世記前面的部份我們可以發現整個世界都變得極為墮落,充滿了罪惡。「耶和華見人在地上罪惡很大,終日所思想的盡都是惡;耶和華就後悔造人在地上,心中懮傷。耶和華說,『我要將所造的人和走獸,並昆蟲,以及空中的飛鳥,都從地上除滅,因為我造他們後悔了。』」(創世記6:5-7)

 現在我們是不是可以開始明白了先知以賽亞在說到撒旦時所講的那些話,「使大地戰抖,使列國震動,使世界如同荒野,使城邑傾覆,不釋放被擄的人歸家,是這個人嗎?」他談的是他對人類的肆意做那些背叛和不法的事,背棄律法和規則並離棄神的震驚。人已經成了一個不法的群體。神不得不用大洪水來清洗整個大地,毀滅整個人類。因為撒旦的工作已經使得大地戰抖,列國震動,使世界如同荒野。

 不過,神做這些的目的是要使這個地球重新昌盛起來。為此,他使用了諾亞以及他的妻子兒女,神向這個家庭彰顯了他自己,而這個家庭也對神充滿了信心。大洪水過後,整個地球又重新生長出萬物。為了防止撒旦的不法再次使大地戰抖、列國震動,我們從創世記9章裡可以知道神建立起了人的政權。這個政權被賦予了很多權力,甚至可以斷定人的生死,人可以通過這種辦法來制止不法的事,使得人可以在這地上成為公義。創世記9:6裡記載了其中的一條主要原則:「凡流人血的,他的血也必被人所流;因為神造人,是照他自己的形像造的。」

 使徒保羅在他給羅馬的信中談到了政府的特權和建立政權的目的。保羅這樣說,「在上有權柄的,人人當順服他,因為沒有權柄不是出於神的。凡掌權的都是神所命的。所以抗拒掌權的就是抗拒神的命;抗拒的必自取刑罰。作官的原不是叫行善的懼怕,乃是叫作惡的懼怕。你願意不懼怕掌權的嗎?你只要行善,就可得他的稱讚;因為他是神的用人,是與你有益的。你若作惡,卻當害怕,因為他不是空空的配劍,他是神的用人,是伸冤的,刑罰那作惡的。」在這一段裡,保羅非常清楚地告訴我們地上掌權的和統治的人是神所指派的,是神的用人。他們不是神挑選的傳福音的用人,而是行使律法的用人,是來判定善惡和公義的,是伸冤的,刑罰作惡的人。聖經裡在別的地方也有這樣的教導,在彼得前書2:13,保羅說我們要順服,「…你們為主的緣故,要順服人的一切制度,或是在上的君王,或是君王所派罰惡賞善的臣宰。」

 我們在提多書裡可以再次發現使徒保羅類似的教導。保羅在這裡向那些在耶穌基督裡得自由的信徒們強調說他們應該知道自己應該順服政府,因為政府是神所建立的,是用來對付撒旦要在地上散佈的一切惡行和不法的事。我們要提醒你的是,使徒保羅就生活在當時世界上最腐敗、最集權的社會裡。但保羅從未自己去批評政府的形式,而且他還對信徒們說他們有責任順服這個政府。

 我們可以得到這樣一個結論,凡是那些合乎神建立政府的目的,即那些維持了公正和秩序的政府,那些阻止了不法和混亂的政府,就是神所允許的政府。信徒就有責任去順服這個政府。我們想使你深刻地明白神對人類的政府有他的旨意。政府是必須的,因為撒旦是不法的,要使社會混亂,他真正的目的是要使人也變成不法的,叛逆社會,叛逆神,叛逆神的律法。為了不讓撒旦使列國震動,,神建立起了政府這個形式。

 撒旦的目的與此大相徑庭。我們從創世記6章裡關於大洪水的敘述就可以看出這一點。撒旦的目的是要使整個世界如同荒野。他想使人和國家都起來反對神的王權,反對神的道所設立的道德標準,以便使整個世界變得沒有律法,亂成一團。下面我們將看到一段保羅描述在末世撒旦的工作完全沒有限制的時候整個社會的景象。提摩太后書3:1-5節寫到:

  你該知道,末世必有危險的日子來到。因為那時人要專顧自己,貪 愛錢財,自誇、狂傲、謗(讀言)、違背父母、忘恩負義、心不聖 潔、無親情、不解怨、好說讒言、不能自約、性情兇暴、不愛良 善、賣主賣友、任意妄為、自高自大、愛宴樂、不愛神,有敬虔的 外貌,卻背了敬虔的實意,這等人你要避開。

 這就是撒旦想要在末世使這地球變成的樣子。

 在今天的這個世界裡,每個信徒都要面臨著撒旦這樣的工作。不僅如此,他還會經歷到撒旦對他內心的攻擊,這是因為撒旦是通過使社會裡的個人變成不法的人從而使整個社會變地混亂不堪的。撒旦就是要通過這種方式把大地戰抖,列國震動。

 當人出生的時候,他就帶著不法和背叛的本性。這就是你我在得救前的樣子。當我們得救的時候,神把我們帶到他的權柄下,使我們受耶穌基督的管理。不過,我們的身上還保留著所有那些舊的惡習,與神的聖潔和公義格格不入。除非我們在聖靈的引導下不斷順服,否則我們的本性終有一天要顯露出來。

 在世界各國的歷史上,沒有哪個國家在一開始創立的時候不設立律法和規矩的。而每一個國家在滅亡之前都是律法和根據先被毀壞掉了。讓我們想一想巴比倫吧。它是舊約裡最強大的一個國家,其他各國都不同程度地受到它的影響。從巴比倫誕生了一些最偉大的法典,這些法典後來成為其他各國的標準。但當你翻開但以理書讀到巴比倫滅亡的前夜的時候,你找到了些什麼?你會看到巴比倫王正在巨大的讌會廳一邊狂飲,一邊斥責一切的神靈。這個國家本來是建立在律法之上的,但是它此刻棄絕了律法。結果一夜之間所有一切都煙消雲散,所有的權勢都破裂了。羅馬的輝煌有如何?羅馬它的法律,叫做Lex Romana。這個從羅馬發展而來的法律體系成了羅馬帝國的根基。但如果你讀過吉本斯(Gibbons)的羅馬帝國興衰史之後,你又會發現什麼呢?在羅馬帝國墮落的時候,羅馬人開始拋棄律法,並拒絕受到上一級的管理。不管是國家還是社會,人們都按照自己喜歡的方式去行事。後來,從北方來的敵人把他們給消滅了。就從他們七絕律法的那刻起,羅馬開始走向了衰落。

 我們的國家是在神的道上被建立起來的,我們的國家尊重神的權柄,也尊重社會的法制。但在近來的一段時期裡,一些不法的行為竟然也開始被社會接受。這種公然叛逆,旨在破壞當今地上律法的現象是我們國家在背離神的道的這條路上行進的標尺。甚至有些教會裡的教牧委員會和一些所謂的基督教組織也加入到這中悖逆神的撒旦的活動中來,想棄絕地上的律法和和法律相關的各種條例。有些教會也參與到撒旦這個詭計中來,想把政府存在的這一根基給毀掉。這和以個人為單位的叛逆有很大的不同。它是一種有組織的反對國家政權的不法的活動。民眾的不順服會改變政府或促使政府作出一些符合百姓意願的事,這些都是撒旦想方設法要讓人棄絕律法的詭計的一部份。不法的事不會因為是有組織的行為就變成合法的事情。

 要是有位居高位的官員說若去指望人去遵守一些他根本就不喜歡的律法純粹是天方夜譚,那麼這些官員就是在宣揚不法的事。若有人的職責就是維護法律和社會的秩序,而他卻在因他對某些律法不太滿意而倡導不去遵守它們,那他就是在幫助撒旦推行他的詭計,要使整個社會變得混亂無序。我們值得去考究一下是否我們今天在不法的路上比當初羅馬人走得更遠。對於羅馬人來說,他們所面臨的結局只有滅亡。我們也值得去考究一下是否我們今天是不是比當初的巴比倫更加混亂。我們是否現在正處在那「飲酒,讚美金銀銅鐵木石所造的神」的伯沙撒王的殿中(但以理書5:4)。神最終審判並拋棄了那個國家!而同樣不法的事情也正充斥在我們當今社會之中。這就是先知以賽亞在他談到撒旦要來使人棄絕律法從而使列國震動時的含義。

 如今,神為了不讓單個的人掉進撒旦的陷阱中去,神在各個不同的領域裡都設立了一些權柄。信徒應該向這些權柄順服。政府只是這些領域中的一個。此外還有在家中的權柄,神已經指定丈夫要在家裡做頭掌權,這種權柄是神給的。這種權柄給了丈夫和父親就是要減少或避免那些不法的事情發生。因為丈夫自己本身就是不法的人,所以他們必須要絕對地順服神。孩子的本性就無拘無束,沒有律法的概念,他們應該順服他們的父母。同樣妻子也該順服丈夫以減少不法的事情發生。撒旦一定回在家庭裡尋找機會製造不法的事情,我們只有通過按照神所賜的權柄去行才有可能避免不法的事情在我們家裡發生。

 聖經告訴我們神在教會裡建立了權柄。神把這一權柄交給了就會裡的長老,並且神要教會裡的信徒去順服那些長老(彼前5:5)為什麼會這樣?這是因為信徒會像其他人一樣的叛逆,不守規矩。神不僅在這個國家和人的家庭裡分配了權柄,他也在教會當中配置了權柄。你會注意到這其實是人生活的三個領域:社會或者說是國家的領域,教會生活的領域和家庭的領域。凡在我們生活的領域裡,神都配備了權柄,這對應付我們的對頭撒旦非常必要。

 還有另外一種權柄,這種權柄屬於那稱為萬王之王,萬主之主的耶穌基督。神不僅給他做救主的權柄,還給了他做我們的主的權柄。當你稱耶穌基督為「主」的時候,你就承認了神給了他權柄,而你在這權柄之下,受這權柄的管轄,受他控制。這其中的目的是什麼?是為要阻止撒旦在你的裡面製造不法的事。那些順服耶穌基督權柄的人也應該在國家、在家庭、在教會裡順服。不過神要我們在這些領域裡面順服的原則是先要順服耶穌基督。

 順服耶穌基督的兒女應該順服他們的父母,因為他們的父母是基督工作的管道。順服基督的神的兒女也順服國家,因為他知道國家也是基督行使他的權柄的一個通道。順服主耶穌基督的神的兒女也順服長老,因為他知道教會的長老也是基督行使權柄的一個通道。

 神的兒女必須要作出決定,到底他要順服哪些權柄。撒旦也為讓你順服他的權柄而給你許諾。十字架上的耶穌也伸出了滿是釘痕的手邀請你和你共背十字架。他邀請你背起你的十字架,把他作你生命的主來跟隨,他本來就有這樣的權柄。你自己沒有能力去和不法的撒旦去爭斗,但你可以不讓他轄制你。你只有把你的生命完全交托給主耶穌基督,他才能在你的身上行使主人對僕人的權柄,成為你的主。那個大騙子會來弄瞎你的眼睛讓你看不到真理,那個說慌的人會來對你否認真理,那不法的叛逆者會來鼓惑你起來反對真理;你只有把你自己放到主耶穌基督的手裡,你才能從這一切的罪惡中脫離出來。

\chapter{撒旦,叛逆者}
\label{sec:ch09}
\hyperref[sec:ch08]{[上一章]}
\hyperlink{toc}{[返主目錄]}
\hyperref[sec:ch10]{[下一章]}

\begin{center}
\noindent\fbox{%
    \parbox{0.8\textwidth}{%
        約伯記1:13-22
            \newline
            1.13 有一天、約伯的兒女正在他們長兄的家裡、喫飯喝酒、
            1.14 有報信的來見約伯、說、牛正耕地、驢在旁邊喫草.
            1.15 示巴人忽然闖來、把牲畜擄去、並用刀殺了僕人.惟有我一人逃脫、來報信給你。
            1.16 他還說話的時候、又有人來說、 神從天上降下火來、將群羊、和僕人、都燒滅了.惟有我一人逃脫、來報信給你。
            1.17 他還說話的時候、又有人來說、迦勒底人分作三隊、忽然闖來、把駱駝擄去、並用刀殺了僕人.惟有我一人逃脫、來報信給你。
            1.18 他還說話的時候、又有人來說、你的兒女正在他們長兄的家裡、喫飯喝酒.
            1.19 不料、有狂風從曠野颳來、擊打房屋的四角、房屋倒塌在少年人身上、他們就都死了.惟有我一人逃脫、來報信給你。
            1.20 約伯便起來、撕裂外袍、剃了頭、伏在地上下拜、
            1.21 說、我赤身出於母胎、也必赤身歸回.賞賜的是耶和華.收取的也是耶和華。耶和華的名是應當稱頌的。
            1.22 在這一切的事上、約伯並不犯罪、也不以 神為愚妄。〔或作也不妄評 神〕
    }%
}
\end{center}

 神是創造者,滿有榮耀和威嚴,能力,所以所有的能力、主權和權柄都歸給了他。作為受造者,首要的責任就是要順服他的創造者。路西非爾在神所造之物中是最有智慧、最美麗的,而他卻背叛了這一基本原則。他的願望就是把神從寶座上趕下來,自己來取代神成為神所造的宇宙中最有權柄的。在這場叛亂裡,他糾集了一大群被造的天使做他的同盟,這些天使都願意看到他的成為叛逆者。

 當神創造亞當並把他安置在伊甸園裡的時候,他也要求亞當做到:順服神、遵守神的旨意。為此就要有機會試一試亞當他們是否真的順服神,於是神就在園中放了一棵分辨善惡樹並對亞當說,「分別善惡樹上的果子,你不可吃,因為你吃的日子必定死。(創世紀2:17)」在撒旦這個悖逆之子的裡面,情慾的烈火熊熊地燃燒著,這股情慾促使他去迷惑亞當和夏娃去背叛神,這是迷惑人去叛逆。夏娃掉進了這個誘惑,亞當隨後也跟著陷入了誘惑,由他們而繁衍下來的整個人類也都因此成了不法、悖逆的人。

 撒旦要迷惑人背叛的計劃恐怕要算約伯記的前兩章裡所記下的神、撒旦和約伯三者之間互相見面的過程最為清楚。從這段記載中我們會發現撒旦對約伯的意圖和他當初對亞當的一樣。他的對你的企圖也是一樣。撒旦最大心願就是迷惑你背叛神。他對此這件事非常關心,甚至可以說比要使你跌入那些罪惡中的事還要關心,因為背叛是所有的罪的開始。約伯的經歷就清楚地證明瞭這一點。

 在約伯那個時代,可怕沒有人在物質上比約伯得到過神更大的祝福。聖經上說神對他和他的家人都大大的賜福,他有7個兒子三個女兒。神也大大豐富了他的物質財富,他有7千隻羊和3千匹駱駝,5百頭耕牛和5百條母驢,還有一個非常大的家產(就是許多的僕人和奴隸,他們照看著那些牲口,為這個家庭提供所需的一切)。所有這些都被記作是神的祝福。儘管這些好像是約伯靠自己的聰明才智和他的能力建立起的這些家產,可他並沒有把這些都歸功給自己。在他得到這些財富的時候,他就來敬拜讚美神。他知道是神給了他這一切,神也可能轉瞬之間就把這一切都拿走。在第一節中就有見證說約伯完全正直,敬畏神,遠離惡事。約伯完全正直表現在他敬畏神。他不僅在他的家人和朋友的眼中是完全正直的,而且在神的眼中也是如此,因為約伯遵守了受造之物與造物主之間的正確關係,這些表明瞭他的完全正直。他畏懼神的權柄,尊敬他的權柄,他願意想神交托,願意在神的權柄前屈膝,他知道他所有的一切都來自於神的祝福。

 約伯生活的一大特點就是始終都敬拜神。他不是只在一些特別的時候才來回首往事,點數神的恩典,才來感謝主。他每天都懷著一顆感謝的心,我們從一章15節裡讀到:「…筵宴的日子過了,約伯打發人去叫他們自潔。他清早起來,按著他們眾人的數目獻燔祭;因為他說:『恐怕我兒子犯了罪,心中棄掉神。』約伯常常這樣行。」約伯是家裡的祭司,他每天都向神獻祭。他用獻祭來堅定他要依靠順服神的心。作為祭司,他這樣宣佈在他的權柄之下的他的兒子也在神的權柄之下。

 最使約伯擔心的事莫過於害怕他的兒子會背叛神,就像他所說的那樣:「恐怕我兒子犯了罪,心中棄掉神。」他為什麼會擔心他的兒子心中棄掉神呢?不管怎麼說,他們都是在這個家庭裡長大成人的,他曾經也訓練過他們、教導過他們,並且還為他們樹立了嚴格依靠神、順服神的榜樣呀。可約伯還是害怕他的兒子會起來背叛神。約伯知道在他自己的心裡都有些什麼。很顯然,他已經感覺到了撒旦在迷惑他叛逆。因為作為一家之主,他既是精神上的父親,也是肉體上的父親。他自己感覺到了那些對他的誘惑,因此也就知道他的兒子也要受到同樣的迷惑。因此他不斷向神獻祭,要以此來守護家人不起來做叛逆神,惟恐他的家人在心底裡棄掉了神。

 撒旦不可能在那裡苦等著那些全心全意願意順服神的受造之物自己來順服撒旦的旨意。似乎撒旦把全部的注意力都放到了約伯這個人的身上,而對於他家裡其他的那些人,甚至是他那些已經起來叛逆神的約伯的朋友,撒旦也都沒太在意。正因為約伯依靠神,所以撒旦最想的還是要使約伯起來叛逆神,並要使他順服他。

 願我們已經讓你明白這樣一個道理:只要神的兒女在聖靈的引導下願意過一個完全討主耶穌基督喜悅的生活,願意完全依靠神與神同行,那麼他同時也會因此而招惹來撒旦這個惡者的攻擊。你不要想你願意順服神的意思,這樣的順服就會使得你遠離與各種誘惑的爭戰。很不幸,這正是你要面臨各種爭戰的開始。只要你走在叛逆神的道路上,撒旦就不會太管你。當你下定決心要走討神喜悅的道路的時候,撒旦就會把你當作他要攻擊的特別目標。他對約伯就是這麼做的。

 當撒旦有這麼個機會到神的寶座前的時候,神為了給撒旦一個客觀的教訓,他暫時離開了約伯。神從約伯的敬拜中得到了喜樂,他看到了約伯的順服並為此而得到了滿足。撒旦就起來指責神收買了約伯。他說,「約伯敬畏神豈是無故呢?」(約伯記1:9)如果神不配得到敬拜,那麼這種對神的敬拜毫無意義。只有受造之物甘心誠意地來敬拜神,那麼神才會得到榮耀。這就是為什麼聖經裡把敬拜當作是嚴重甘心樂意的獻祭的原因。有些獻祭是必須的,而有些獻祭卻是人主動來獻的。甘心來獻祭的這種順服的心才真得會使神得到滿足。在這種甘心樂意的獻祭中,神得到人的敬拜,這是因為這些人真的愛他,因為他們敬慕他,因為他們知道他理應得到敬拜。

 撒旦來到神的面前侮辱神。他實際上在對神說,「你用各種的祝福收買了約伯,他的敬拜、順服都是你收買來的。如果你不給他物質上的祝福,他就不會再這樣對你了。」撒旦挑戰神說,「你且伸手,毀他一切所有的;他必當面棄掉你。(約伯記1:11)」撒旦最想讓約伯做什麼?是侮辱神。是叛逆神。在撒旦的腦海浬,神如果不給各樣的恩賜,那他是不配得到讚美的。撒旦不需要自己來張嘴辱罵神,只要約伯抬眼向天問道,「這是為什麼?」這就已經可以算做是叛逆神了,就可以說約伯中了撒旦的詭計。這個看似無罪的發問就可以虧損神的榮耀。報信的來到約伯的面前。14節說所有的牛和驢子不是被擄去了,就是被殺了。16節中又說所有的羊群和僕人都死了。第三次在17節裡說所有的駱駝都被迦勒底人搶走了。第四次,在18和19節裡說,他的10個兒女都在一場風暴中死去了。片刻之間,約伯失去了所有的一切。他沒有任何的財產和兒女,成了乞丐。

 即使是在這種情況下,約伯的信心也從未動搖過。雖然撒旦來迷惑他去「棄掉神」,可在20節裡記著說,「約伯便起來,撕裂外袍,剃了頭。」這些是悲痛的表示。約伯並沒有對眼前的一切視而不見,他沒有否認現實,也沒有試圖去逃避現實。他撕裂外袍的行為說明他對報信人的話非常重視。但他沒有「棄掉神」,而是「伏在地上下拜。」他在此刻還是像以前那樣敬拜神。這是撒旦吃過的最糟糕的一次敗仗。很少有人遇到過約伯遭受過的撒旦對他的這樣攻擊。約伯認定了要依靠神,他看清了神的良善,他知道神的智慧是值得相信的,這些使約伯得以支撐下來,沒有使撒旦聽到他想聽的話。撒旦聽到的是從神的聖徒口中傳來的讚美神的話。

 一個人面臨試探的時候再要學習敬拜,再學習相信神,再學習憑著信心行走就太遲了。在試探來臨之前每日與主同行,這樣人才能勝利地通過撒旦的試探。約伯在受試探的時候也年敬拜神是因為敬拜已經成了他生活的一部份。他早些時候順服的心一直沒有改變,即使是在受到撒旦攻擊的時候也是如此。

 儘管撒旦遭受了最丟人的一次失敗,可他隨即又準備好來向神挑戰。神再一次指向了約伯,約伯已經成了撒旦在這地上最不願記起的人。神說,「你曾用心察看我的僕人約伯沒有?地上再沒有人像他完全正直,敬畏神,遠離惡事。(約伯記2:3)」換句話來說就是神問到,「撒旦,你怎麼駕駛約伯這樣的人?」唯一正確的答案應該是:「約伯知道應該依靠你,因為你是值得讚美的。」但撒旦卻沒有這樣解釋,他反過來再次挑戰神說,「人以皮代皮,情願舍去一切所有的保全性命。你且伸手,傷他的骨頭和他的肉,他必當面棄掉你。(約伯記2:4,5)」撒旦到底想使約伯做些什麼?是叛逆!

 神於是再一次允許撒旦向約伯施展他的詭計,撒旦就開始對約伯做工,想讓他像那些墮落的天使和亞當、夏娃當初犯罪一樣,也起來叛逆神。我們從8節可以讀到,「約伯就坐在爐灰中」。在城外才有一堆一堆的爐灰,那裡不是個潔淨的地方,那是無家可歸著、流氓、乞丐住的地方。約伯此刻已經離開他自己的家,離開了他唯一的親人,他的妻子,離開了他的朋友,他一定覺得自己被拋棄了。就在這個時候,對他的迷惑開始了。我們注意到撒旦這一次非常小心,他沒有直接來到約伯的面前,而是通過他的妻子來迷惑約伯,他的妻子成了撒旦誘惑的工具。當撒旦以受造之物的樣子來到夏娃面前的時候,夏娃被誘惑了。這一次,約伯所受的試探比夏娃的還要大,因為這種誘惑來自約伯最親密也是裡他最近的人。撒旦的打算就是要誘惑那些最難受誘惑的人,不過撒旦誘惑人的目的始終未不會變。約伯的妻子來到約伯面前說,「你仍然持守你的純正(依靠神)嗎?你棄掉神,死了吧!」這些話更顯明瞭撒旦的陰謀,也就是撒旦最想使約伯做的事情:不依靠神,不順服神,不聽神的話,悖離神。

 約伯的回答是如何榮耀了神呢(10節)?他斥責了他的妻子,說:「你說話像愚頑的婦人一樣。」在聖經裡愚頑總是和忘記神的存在聯繫在一起的。他是在說,「你說的話就像那些不信神的人一樣,你在要我做撒旦想讓整個人類做的那些事情。」「難道我們從神手裡得福,不也受禍嗎?在這一切的事上,約伯並不以口犯罪。(約伯記2:10)」約伯拒絕與神脫離關係,這是因為他對神的品格的信心,他知道自己有責任順服神。

 撒旦對約伯的這些用意也是他對你對我的用意。他並不想讓你去搶劫銀行,或者從你的老闆那裡盜取一大筆錢,也不想讓你捲入某些倫理道德方面的醜聞。那根本就不是他對你的計劃。他要的是讓你背叛神,讓你對神說「不」;讓你否認神對你有絕對的權柄。當你這樣去做的時候,你就掉進了夏娃的罪的裡面;你就搶奪了神和神的榮耀,他應該被遵守和敬拜的權利。

 約伯不是唯一遇到過這種試探的人。在馬太福音16章我們會看到當主與他的門徒一起走路的時候,他回過頭來問他們說,「人說我人子是誰?」他的門徒於是就開始重複他們聽到的那些話。有的人說是施洗的約翰,有人說是以利亞,有人說是耶利米或是先知裡的一位。在聽完門徒們說的這些話之後,耶穌回過來直接問門徒們說,「你們說我是誰?」彼得代表門徒們出來回答說,「你是基督(彌賽亞),是永生神的兒子。」基督隨後進一步向他們顯明,「從此耶穌才指示門徒,他必須上耶路撒冷去,受長老、祭司長、文士許多的苦,並且被殺,第三日復活。」這一切都是先知預言說彌賽亞一定要經歷的。在以賽亞書53章裡已經說到彌賽亞,就是彼得剛剛承認的,是要來受苦,把自己獻上作為贖罪祭,用自己所灑的血為罪人帶來救贖。當我們的主聽了彼得的話就承認說,「是的,我 就是彌賽亞。我現在就要到耶路撒冷去為世人的罪而死。」彼得走過來把雙手放在耶穌的肩頭搖著他,就像要讓他清醒過來過來那樣。他說,「主啊,萬不可如此!這事必不臨到你身上。」注意我們主的回答,「撒旦,退我後邊去吧。」

 就像當初撒旦用約伯的妻子做他的陰謀的管道,想讓約伯違背神的旨意一樣,撒旦在這裡也利用了耶穌的門徒來迷惑耶穌。他的目的是什麼?就是要讓他違背神的旨意,使耶穌不照神的計劃死在十字架上,而是悖離神的計劃,離開耶路撒冷,逃到安全的加利利地區。耶穌基督認出那話的用意所在,就說了那話。他知道那話是出於撒旦的誘惑,話雖然是從彼得可口中出來的,但撒旦才是那真正說這話的人,是他給了彼得這種想法。因此他說,「撒旦,退我後邊去吧!你是絆我腳的,因為你不體貼神的意思,只體貼人的意思。」撒旦沒有更大的陰謀,他只是始終在想法使耶穌基督不按照神的意思去做,使他不順服神。

 不久,主就進入了客西馬尼園。那誘惑人的撒旦甚至也跟到了那裡。而基督也在那裡和同樣的誘惑進行了一番鬥爭。那誘惑人的甚至又出現在了那裡。耶穌抵擋住了那要他違背神意願的誘惑,他在天父面前跪拜說:「然而,不要照我的意思,乃要照你的意思。」耶穌在死的問題上順服了神。這也正是為什麼希伯來書的作者邀請我們「仰望為我們信心創始成終的耶穌,他因那擺在前面的喜樂,就輕看羞辱…」的原因。

 撒旦使約伯遭害與撒旦在基督上花費工夫的目的都是一樣的---都是想使他們背離神的旨意,不順服神,這也是撒旦每時每刻對你的最大渴望。要儆醒,不要叛逆了神的旨意。神要你繼續如一地順服他,依靠他,作認識神在你生命中的權柄的敬拜者。

\chapter{吼叫的獅子追趕}
\label{sec:ch10}
\hyperref[sec:ch09]{[上一章]}
\hyperlink{toc}{[返主目錄]}
\hyperref[sec:ch11]{[下一章]}

\begin{center}
\noindent\fbox{%
    \parbox{0.8\textwidth}{%
        彼得前書5:1-11
            \newline
            5.1 我這作長老、作基督受苦的見證、同享後來所要顯現之榮耀的、勸你們中間與我同作長老的人。
            5.2 務要牧養在你們中間 神的群羊、按著 神旨意照管他們.不是出於勉強、乃是出於甘心.也不是因為貪財、乃是出於樂意。
            5.3 也不是轄制所託付你們的、乃是作群羊的榜樣。
            5.4 到了牧長顯現的時候、你們必得那永不衰殘的榮耀冠冕。
            5.5 你們年幼的、也要順服年長的。就是你們眾人、也都要以謙卑束腰、彼此順服.因為 神阻擋驕傲的人、賜恩給謙卑的人。
            5.6 所以你們要自卑、服在 神大能的手下、到了時候他必叫你們升高。
            5.7 你們要將一切的憂慮卸給 神、因為他顧念你們。
            5.8 務要謹守、儆醒.因為你們的仇敵魔鬼、如同吼叫的獅子、遍地游行、尋找可吞喫的人.
            5.9 你們要用堅固的信心抵擋他、因為知道你們在世上的眾弟兄、也是經歷這樣的苦難。
            5.10 那賜諸般恩典的 神、曾在基督裡召你們、得享他永遠的榮耀、等你們暫受苦難之後、必要親自成全你們、堅固你們、賜力量給你們。
            5.11 願權能歸給他、直到永永遠遠。阿們。
    }%
}
\end{center}

 作為一個忠實的牧羊人,使徒彼得在一天天為他的群羊準備著他們要過的那種生活。這是種非常艱難的生活,因為福音在他們要去的那個社會裡是不被人所接受的。從宗教的角度來說,那裡的人自得其樂,而且很鄙視這種威脅到他們原有的信仰的新東西。從政治的角度來說,基督教也不被接受,因為它使人們期盼那要再來這地上的萬王之王,萬主之主,他要建立一個國度,使所有國家都要臣服在他的權柄之下。從經濟的角度來說,在教會的那些人因為脫離了原有體系而遭受著窮困和缺乏。

 使徒彼得在看到這些問題後,就在他的第一封書信中勸勉那些散在四方、正在受苦的信徒們在他們所遭受的逼迫中勇敢地生存下來。在經歷過可能是最嚴酷的逼迫之後,他說從撒旦而來的逼迫才是最大的逼迫。他在這封短信的最後以牧人的口吻說,「務要謹守、儆醒;因為你們的仇敵魔鬼,如同吼叫的獅子,遍地遊行,尋找可吞吃的人。(彼得前書5:8)」別忘了彼得給我們的告誡:謹守和儆醒。

 聖經裡所說的儆醒一般和喝酒之類的事情沒有什麼關係,它主要是指態度上的嚴肅認真,與我們對生活的認識有關。態度認真的人往往會看到事物的本質,因為他們有聖經的價值觀,生活在耶穌基督給他的新生活的原則中。那些不了解這世界的本質,對攻擊我們的敵人撒旦毫無戒心的人的生活方式常常是輕率毛草的。而以耶穌的態度來看待生活的人,他一定有一個全新的態度,他這種生活的標誌之一就是謹守。

 除了要有嚴肅認真的態度之外,信徒還應該保持儆醒。儆醒是指小心留意週圍的動靜。當人發現在週圍出現了敵人的行蹤的時候,他就會開始儆醒。如此看來,謹守是指信徒內心的態度,儆醒是指對外部的防禦。由於攻擊都是憑空出現的,所以持有這種新態度非常重要。神的兒女必須要明白他正生活在他的敵人中間,也就是說他的週圍有許多看不見的敵人正在尋找機會想要毀滅他。他應該知道他必須要超過一片沼澤地,而那裡可能到處都隱藏著他的敵人,可能隨時遇到埋伏。我們無法想象,如果我們的士兵在越南戰場上不謹守、儆醒的話,他們如何能襲擊敵人的領地。或許他們看到敵人的子彈穿透同伴的身體,就開始變得謹守了;當他們認識到在每一團灌木的背後都可能隱藏著一個正在想他們瞄準的敵人的時候,就開始儆醒了。如果哪個人來到這樣一個戰火紛飛的戰場卻還保持著一種別樣的態度,那他一定是個十足的傻瓜。可在我們當中又有多少神的兒女把生活當成了主日學校的野餐會,仿彿我們只不過是要走到桌子前拿一杯飲料!

 在命令我們要謹守和儆醒之後,彼得接著告訴我們這樣做為何如此重要。「因為你們的仇敵魔鬼,如同吼叫的獅子,遍地遊行,尋找可吞吃的人。」彼得所說的你們是指信徒來說的。對已經進入撒旦的家庭裡的人來說,撒旦並不是他們的敵人。他只與那些憑著信心進入耶穌基督裡面得到重生的人為敵。這些人曾經屈服與他,把他當作是他們的父、他們的首領、他們的主、他們的神。現在,這些人都認識了主耶穌基督,他們發現基督有更大的權柄,耶穌的話比撒旦的命令更值得去遵守。他們看清了撒旦的謊言,他們接受了耶穌基督這個真理。因為他們憑著信心棄絕了撒旦,接受了耶穌基督為他們的救主和他們的主,和撒旦對立起來。

 撒旦對他的目標不會甘心樂意地輕易放棄。人被做了標記、脫離了受地獄的火煎熬的命運,成為神的兒女的事實會使撒旦怒不可遏,因為這人從此就成了他恥辱的記號。因為你接受了耶穌基督並起來與撒旦為敵,成了魔鬼的敵人。撒旦不會同你講和的,如果你想撒旦對你的現在所做的無所謂,他也不在乎你離開了他,那麼這一定是撒旦在欺騙你。如果是你,你可能會這樣做,因為你自己的麻煩事已經實在太多了;可這不是撒旦的做事原則。就在你成為神的兒女的那一刻起,撒旦就已經成了你的敵人。

 撒旦不是無所不在的。只有父、子、聖靈三位一體的神才是無所不在的,只有他會給你單獨的看顧,就好像你是這世界上唯一的受造之物一樣。神無盡的愛、看顧、關心、照顧和供給都是直接針對你的。而撒旦卻不是無所不在的。那麼他怎麼會給你帶來無窮的麻煩呢?撒旦是地獄裡的組織者,在他國度裡所有一切也都井然有序。撒旦有個可以來模仿的好榜樣。儘管神是無所不在、無所不知、無所不能的,但他對他的國度裡一切事情的管理都是通過受造之物和天使來完成的。神給每個得到拯救的人都安排了一個守護天使。這種方式運行的十分順利,因為神的天使對神的旨意完全順服,所以神所揀選得救恩的人都不會迷失掉。

 當路西非爾背叛了神的時候,他同時也帶走了一大群原屬於神而又背叛了神的天使。他們把路西非爾當作神,都歸順了他。撒旦模仿了神的管理模式,通過那些墮落天使來推行他魔鬼的計劃。儘管撒旦不一對一地與你同在,不過他還是能通過他手下的魔鬼來控制你,他能命令那些順服他的魔鬼來完成一些他在你的生活裡面想做的工作。撒旦不會讓他手下的魔鬼放松對你的進攻。在這看不見但是又時刻不在的爭戰中,我們的敵人肆無忌憚地不斷對信徒們進行攻擊,要把信徒們從順服神話語的道路上分離出去。

 使徒的話強調了撒旦的這種攻擊,「你們的仇敵魔鬼遍地遊行。」注意這裡用地是現在時態。雖然你要睡覺,可撒旦不會停下來休息。在你休息的時候,他會來籌劃明天如何來攻擊你。你坐在教堂裡,撒旦也是如此,他不會放過你。他也會坐在教堂外面;他會到那裡面去攻擊你,找機會把神的道在你裡面所栽下的種子拔出來。

 使徒很形像地用「獅子」的比喻向我們描述了撒旦無休無止的毀滅的工作。獅子遍地行走並不是因為它羨慕羊兒的美麗,而因為飢餓的緣故,他遍地行走是因為他知道他要毀吞吃掉他的獵物。撒旦被描述成一個遍地行走的「吼叫」的獅子。「吼叫」是指撒旦以為他已經俘虜了獵物。沒有獅子會大聲吼叫著去捕捉獵物。它不會一覺醒來後,腹中空空地來到曠野裡對正在那裡吃草的動物們大聲宣佈說它正在找東西吃。相反,它會悄悄地快速行動,抓住它的獵物,然後再大聲地吼叫。這種吼叫有兩種含義:第一,它是宣佈勝利的號角;其二,因為獅子膽小,這也是在警告其它的動物它正在享受剛殺死的獵物,離它遠點。

 我們應該把這些事情和撒旦聯繫到一起來看,這一點非常重要。我們因為撒旦會像一位紳士那樣地大聲呼喊他屬於什麼地方,等他說完他的用意後我們還有機會再去找些東西來保護我們自己。我們曾經被他欺騙,想我們還有時間來準備,來安排如何面對他。你看,撒旦就是這樣騙我們的!撒旦直到他能為勝利而吼叫的時候才會向你顯露出他自己來。當他吼叫的時候他不是在向你宣告你已經被他毀滅了。這一點你那時早已經明白了。他在毀滅和搶奪了神的兒女之後,他吼叫是要來藐視神,他是在向神挑戰看神能對他的所作所為做些什麼。當路西非爾第一次背叛神的時候,他說,「我要高舉我的寶座在神眾星以上;我要與至上者同等。」當撒旦把神的一些兒女領進罪裡的時候,他就大聲吼叫來蔑視神,因為他想他已經一步一步接近了他的目標,那就是把神趕下寶座,由他自己來掌管神所創造的宇宙萬有。

 聖經裡給我們舉了許多例子來說明撒旦是如何像吼叫的獅子到處尋找可吞吃的人的。我們首先來看一下撒母耳記下11章裡關於大衛王犯罪的記載。請留意撒旦是怎樣按照他的原則來一步一步偷偷地施展他的詭計的。第一節裡告訴我們說這件事發生在列王外出爭戰期間。神在戰場上給大衛王許多了不起的勝利。在大衛王統治的時候,以色列的版圖得到了最大程度的擴張。這些都是神的勝利,而且很明顯神在這個時期對大衛有很多的祝福。但大衛由於感覺到神對他的祝福,知道神與他同在,於是就沒有像以前那樣謹守。第二節告訴我們說,「一日太陽平西,大衛從床上起來,在王宮的平頂上遊行。」看起來好像大衛正在晚上的時候休息。他從前方戰場上頻頻受到捷報,本已經心滿意足地睡在了床上。他知道神祝福的手依然還停留在他的身上。可不知是什麼原因,他開始無法入睡。可能是因為天氣太熱的緣故吧,因為他起來後爬到了屋頂,那裡應該是最風涼的地方了。這一切看起來都很正常,人想放松一下嘛。他沒有想到這其中會有罪。但當他走到屋頂看到有個漂亮的女人正在洗浴的時候,他把那個婦人找來,跌進了罪裡。大衛是有意做這些的嗎?不是。他預料到這些事情嗎?沒有。大衛是做了什麼錯事才發生了這樣的事情嗎?不是。是我們的敵人撒旦悄悄地跟在神的孩子的旁邊,大衛陷入了他的誘惑,並犯下了使他自己棄絕自己和神的罪。當大衛最沒有注意到罪的時候,撒旦找到了一個攻擊他的機會,而大衛之所以犯罪是因為他當時沒有謹守,不夠小心。

 我們再來看看另一個例子。在路加福音22:31裡,我們的主說,「西門!西門!撒旦想要得著你們,好篩你們如同篩麥子一樣;但我已經為你祈求,叫你不至於失去了信心…」 耶穌的這些話是在門徒們和他一起度過了那段寶貴經歷不久後說的。基督在此之前讓他們準備過逾越節的筵席,這逾越節是為紀念神把以色列人從埃及人手下解救出來而設立的,而在不久以後,我們的主彌賽亞也要親自為人的罪而把自己獻上作為贖罪祭。在逾越節晚餐過後,我們從約翰福音14-16章裡可以知道主當時已經倒空了自己,並和眾門徒一同分享了許多教導。他教導他們在他死而復活之後將會有一位保惠師下來。我們的主心裡面想得都是他的門徒。他的話和他的愛充滿了門徒們的心。可主在那時卻說,「彼得啊,撒旦就在這裡,他想篩你們如同篩麥子一樣。」主走進園中,三次跪倒在天父前禱告,「然而不要成就我的意思,只要成就你的意思。」但是他與天父這種親密的關係在那隊軍兵來捉他,把他帶到大祭司面前的時候就斷絕了。每個人的注意力都集中在衣著講究的大祭司和他面前被告有罪的主耶穌基督的身上。每個人都在聽著他們倆人之間的談話。

 還會有誰注意著那站在火堆旁的彼得嗎?撒旦沒有去聽祭司和耶穌的談話,而是把全部的注意力都集中在彼得的身上。路加福音22:54裡說,「彼得遠遠地跟著。他們在院子裡生了火,一同坐著;彼得也坐在他們中間。有一個使女看見彼得坐在火光裡,就定睛看他,說:『這個人素來也是同那人一夥的。』彼得卻不承認,說:『女子,我不認得他。』」彼得剛剛和基督在一起並且得到最豐盛的祝福後不久,他就不認了主。為什麼?因為彼得沒有謹守、儆醒。撒旦一直都在尋找機會,這次他終於找到了,於是就來攻擊了彼得。

 我們從使徒行傳5章裡可以再次發現這條原則。使徒行傳4:32-37裡記載說有一群信徒聚到一起,把他們所有的財產都放到一起公用,沒有一個人缺乏。他們之間有一種親密的同工關係,因為他們都是被不信的宗教和政治權勢趕出來才在一起互相幫助的。如果說還有一群人保持著謹守和儆醒,那似乎就應該是他們這一小群人了。然而我們往下讀就會發現有個叫亞拿尼亞的人變賣了所有的家產,「把價銀私自留下幾分,他的妻子也知道,其餘的幾分拿來放在使徒腳前。彼得說,『亞拿尼亞!為什麼撒旦充滿了你的心,叫你欺哄聖靈,把田地的價銀私自留下幾分呢?』」審判臨到了亞拿尼亞,不久也臨到了他的妻子撒弗喇。為什麼?因為撒旦一直都在尋找機會要和神當面對抗。

 犯罪並不一定是深思熟慮的結果,很多情況下都是因為信徒的心裡沒有太在意,沒有分辨出紛爭的本質、沒有儆醒,才導致犯罪。他給了撒旦一個可乘之機,而撒旦就抓住這個機會起來攻擊神的兒女和他們的基督徒生活。

 使徒在以弗所書4:27裡指出了這種危險,他給人這樣的命令,「不可給魔鬼留地步。」在歌林多後書2:10,11裡保羅說,「你們赦免誰,我也赦免誰。我若有所赦免的,是在基督面前為那麼赦免的;免得撒旦趁著機會勝過我們…」一個人若要爬山,他不不一定非要有路才能爬到山頂。經驗豐富的爬山運動員可以在冰面上自己開路,只要他能鑿出足夠的小洞讓自己落腳,他就能爬上陡峭的懸崖峭壁。他不會開著推土機先造出一條路再爬上山去。而我們有時卻想除非我們給撒旦準備好了一條康莊大路,負責他絕不會來到我們跟前來攻擊我們,來勝過我們。彼得不信這些,保羅也不相信這一點。保羅為怕給魔鬼留地步而擔心。一場軍事進攻只需要有個登陸的地方就行了,有了這個登陸的地方軍隊就能發動一次成功的攻擊。

 撒旦就一直在找這個登陸的地方,他如果找到這個地方,他就能毀掉你的一生。如果你開始懷疑聖經的權柄,你就已經給撒旦留了地步,撒旦就可那就借助這裡來毀掉你的信心。他不一定非要你去否認神的聖潔和公義才能改變你。他只需要使你有那麼一點點不順服神,就能藉著這裡來毀滅你的生活。

 使徒看到了這種危險,他們知道撒旦無時無刻都在觀察著每個人的腳蹤,你一旦給了他一個機會讓他留了地步,他就會在你的生活中紮根,開始毀滅你的工作。但如果你不給他機會,他就無法做這些事情。撒旦無法抵擋聖靈,也無法抵擋使徒在以弗所書裡所描述的真理的盔甲。

 彼得很擔心那些遭受世界的逼迫、政府的逼迫、和當時宗教勢力的逼迫,他認為信徒們應該知道這些並不是真正危險的事情,也不是我們真正的仇敵所建立的體系。真正的敵人是那撒旦,是他在每時每刻尋找攻擊的機會。大衛王只不過多看了一眼,就給了他機會。彼得只不過稍有一點擔心,也給了撒旦以可乘之機。亞拿尼亞存了點私心,撒了個謊,就讓撒旦抓住了。凡是給他機會的人都會發現撒旦已經來佔據了他們。

 如果你想要抵擋撒旦,你就必須要謹慎,要認清你所面臨的紛爭的實質是什麼。你也必須要儆醒,知道撒旦在每時每刻觀察你。你必須要完全地依靠聖靈,用神給你的盔甲來抵擋那惡者的攻擊。總之,要謹守,儆醒。

\chapter{撒旦的道理}
\label{sec:ch11}
\hyperref[sec:ch10]{[上一章]}
\hyperlink{toc}{[返主目錄]}
\hyperref[sec:ch12]{[下一章]}

\begin{center}
\noindent\fbox{%
    \parbox{0.8\textwidth}{%
        彼得後書1:16-22
            \newline
            1.16 我們從前、將我們主耶穌基督的大能、和他降臨的事、告訴你們、並不是隨從乖巧捏造的虛言、乃是親眼見過他的威榮。
            1.17 他從父 神得尊貴榮耀的時候、從極大榮光之中、有聲音出來向他說、這是我的愛子、我所喜悅的.
            1.18 我們同他在聖山的時候、親自聽見這聲音從天上出來。
            1.19 我們並有先知更確的預言、如同燈照在暗處.你們在這預言上留意、直等到天發亮晨星在你們心裡出現的時候、纔是好的。
            1.20 第一要緊的、該知道經上所有的預言、沒有可隨私意解說的.
            1.21 因為預言從來沒有出於人意的、乃是人被聖靈感動說出 神的話來。
    }%
}
\end{center}

 如果你只知道撒旦假扮別人而又要找他,那應該到什麼地方去找呢?在酒吧的角落裡?在色情用品店?賭場?舞廳?你想過去教堂裡去找嗎?你會在那裡找到他。看起來聽奇怪,不過撒旦確實是更關心你在想什麼,你信什麼,這些對他來說比你做了什麼更重要。撒旦的願望是控制你的心靈,這樣他就能控制你的行為。撒旦對他的目標會非常地專心。所以說他會到教堂去。

 撒旦也有他的道理。在提摩太前書4:1裡保羅說,「聖靈明說,在後來的時候,必有人離棄真道,聽從那迷惑人的邪靈和鬼魔的道理。」這鬼魔的道理說得不是關於鬼魔的道理,而是鬼魔要宣講的道理。使徒約翰在他寫給7所教會的信中,即在啟示錄2:24裡說,「至於你們推雅推喇其餘的人,就是一切不從那教訓,不曉得他素常所說撒旦深奧之理的人,我告訴你們,…」

 宣教給人類社會帶來了最大影響和改變。儘管撒旦可以利用各種先進的現代通訊工具,但沒有什麼比神所定的宣教更有力,給人的生活帶來更大的改變。被呼召出來宣教的人控制的是人的心靈。當然,撒旦也想用各種方式來改變人的思想。他會用高壓政治手段,利用報紙、雜誌、廣播、電視。但這些辦法都沒有宣教更有效。

 因此,他就要來傳播他的假道理。撒旦以公義的使者的身份來佔據講臺,這樣他就能向下面的人發號施令,控制他們的想法。歌林多後書11:13裡有關於撒旦宣講他的道理的記述:「那等人是假使徒,行事詭詐,裝作是基督使徒的模樣。這也不足為怪,因為連撒旦也裝作光明的天使。所以他的差役,若裝作仁義的差役,也不算希奇。他們的結局必然照這他們的行為。」

 看來,撒旦既有自己的道理也有他傳播這些道理的方法。他的方法就是模仿神的方法。他把他的人放到一些地方,好使眾人能來聽這些人的教導。他被那些人本只屬於神的使者的權柄,再讓他們去教導假道理去捕捉人的心思意念。使徒在約翰一書4:2裡再次強調說,「凡靈認耶穌基督是成了肉身來的,就是出於神的,從此那麼可以認出神的靈來。凡靈不認耶穌,就不是出於神,這是那敵基督者的靈。」你可以看到使徒在這段話裡警告說要有假靈出現。這些假靈就是現在那些把自己裝扮成神的使者,而實際上卻是從撒旦來的人。他們來不是要傳講真理,而是要傳講那些撒旦用來蒙蔽人的心眼的假道理。

 彼得在彼得後書2:1也談到了同樣的事情:「從前在百姓中有假先知起來,將來在你們中間也必有假師傅,私自引進陷害人的異端,連買他們的主他們也不承認,自取速速的滅亡。」以上我們所舉出的經文都強調了一個事實,這就是撒旦通過那些表面看來是宣講福音的使者或是屬神的人,而實際上卻是撒旦的使者的人來傳播撒旦的道理。

 主已經很清楚地說過撒旦永遠不會傳講真理,他宣講的只會是謬誤。約翰福音8:44節裡說:「你們是出於你們的父魔鬼。」這句話的語氣非常重,因為基督這裡講的對象是他那個時候的宗教領袖。他們都是正直、有文化、受尊敬的人,他們的能力使得他們擔當瞭如此的重任,有如此大權柄。但基督卻對他們說,「你們是出於你們的父魔鬼。」基督這樣稱呼他們是因為當基督說:「我就是道路、真理、永生」,他們卻說這是謊話。他們對人說要遵從他們的路,要遵從那些外在的法利賽人的律法。基督說:「我就是真理。」而他們卻說,「不,不。你是個騙子;我們的手中有真理,是從摩西那裡得來的。」基督說,「我就是生命。」他們說,「不,你有生命但不是生命,因為你是亞伯拉罕的子孫。如果你想得完全的生命,跟從我們的傳統吧。」他們把基督叫做說謊者。當基督說,「我是神的兒子,」他們說,「你是魔鬼。」他們否認主耶穌口裡說出的每句話。所以基督對他們說,「你們是出於你們的父魔鬼,你們父的私慾,你們偏要行。他從起初是殺人的,不守真理,因他心裡沒有真理。他說謊是出於自己,因他本來說謊的,也是說謊之人的父。」基督是在這裡強調說撒旦在撒謊。還記得撒旦這樣騙夏娃的嗎?「你們不一定死。」不過基督要強調的還遠不止這些。

 撒旦不僅撒謊,而且他有說謊的本性。撒旦建了一個假系統,它完全是與神的真理背道而馳的。說撒旦撒謊不僅是因為他不說真話,還因為他建立了一個假的系統,而這個系統就被稱為「撒旦深奧之理」,是完全違背聖經真理的。在提摩太后書3:13裡保羅說,「這是那作惡的和迷惑人的,必越久越惡,他欺哄人,也必被人欺哄。」假師傅們自己被騙上當,被欺哄了之後,也來起鬨別人想控制別人。

 在帖撒羅尼迦後書2章裡,保羅在談到那不法的人要來臨的時候反復地強調了這一點。在9節他寫到:「這不法的人來,是照撒旦的運動,行各樣的異能、神跡,和一切虛假的奇事,並且在那沉淪的人身上行各樣出於不義的詭詐;因他們不領受愛真理的心,使他們得救。故此,神就給他們一個生發錯誤的新,叫他們信從虛妄」。

 我們談了以上這麼多就是要強調這一點:撒旦有個公開的目標,這就是要控制人的思想。如果他能控制熱的思想,他就能控制人的行動。他建立起了一個騙人的體系,是與神的真理完全相反的。他把他的差役喬裝打扮,好使他們看起來像是公義的使者,在這層外衣之下,他們宣講著騙人的言論。而且撒旦還把這些人推上了高位,讓他們掌權,好使人們都來因為他們的職位而尊敬他們,聽從他們的教導,也就是聽從撒旦的教導,跌進撒旦的騙局。最終使他們按照撒旦的模式來生活。

 在我們閱讀神的話語的時候,我們會發現有些地方常常是撒旦攻擊的對象。而有些地方撒旦則想方設法不讓人去相信。但撒旦在有些情況下也會退後。如果一個人只把基督當做他的好老師,撒旦就會退步。如果有人要相信聖經只是本特別的書,而裡面充滿了謬論,是不可信的,撒旦也會靠後。撒旦會允許人們對神的道理有一點點的了解,但如果人要去了解神的那些基本原則,他就不會再容忍下去了。

 撒旦首先要攻擊的就是聖經,他要否定聖經是神所默示的,是滿有權柄而且沒有謬誤的聖經。或許你對這種說法會感到有些陌生。聖經完全是在聖靈的啟示下寫成的。雖然聖經裡各部書在屬靈的價值方面可能有所不同,但每一部書,從創世紀到啟示錄,都是從神的啟示而來的。民數記和以弗所書或許有所不同,但這並不等於神在以弗所書上的啟示更多一些。

 聖經因為是神所默示的,所以聖經裡面沒有謬誤。無論是從地理、歷史、科學。還是從宗教信仰等方面來說,聖經都是沒有謬誤的,神不會騙人。正因為聖經無謬誤,所以它應該在我們生活和信念上擁有的絕對權柄。我們所做和所信的必須要和聖經的原則相吻合,否則我們就落進了魔鬼撒旦的誘惑。聖經的原則是撒旦所痛恨的原則。

 從提摩太后書4:4中我們可以知道當世人「掩耳不聽真道,偏向荒渺的言語」的時候,那日子就要來了。真道是什麼?就是神的話,是聖經。撒旦的第一個目標是使人不相信由神所默示的聖經的統一性和它的權柄。許多人告訴我說神在他們小的時候就通過「傳統」教會在他們心裡栽下了信心的種子,而當他們帶著對神的話語的簡單的信心來到大學或是學院什麼地方的時候,當他們與那些有知識的人一同坐在班級教室裡的時候,他們發現他們對神的話語的態度改變了。而當他們的大學課程結束的時候就完全拋棄了聖經,他們把這當作是一種先鋒的姿態,反對對他們的所有權柄。那些如此毀滅真理的人都是撒旦的器皿。他們以倡導知識自由為偽裝,而實際上卻在做撒旦的工作。他們宣揚撒旦的道理,毀壞神的計劃的基礎。撒旦不會容許有人按著聖靈的感動而接受聖經的權柄,確認聖經的統一性。要想看看撒旦的工作到底成功到何等地步,我們只需要看看我們國家的普通學校究竟做了多少跟訓練教會工人的工作就能明白了。要找到一個完全靠神的啟示,確信聖經的權柄和統一性的學校簡直就像是在大海浬撈針。很多學校都是在按照他們那種所謂培養「公義的使者」的思想,在有系統地破壞真理的根基。這些人都是撒旦的使者,因為他們都在宣揚撒旦的迷惑和他的教訓。

 撒旦也不會容忍人去接受屬基督的人的教訓。聖經表明耶穌基督是永生神的永生的兒子。耶穌不是受造之物,他的位格與天父一樣。為了拯救我們,他道成肉身,自己降卑為人的樣式,因此把完全的神和完全的人統一在他的身上。他以救主的身份因我們的罪而死。神在基督受洗的時候就肯定說他是他的兒子:「這是我的愛子,我所喜悅的。」但也有人把神稱為騙子,說神在騙人,在迷惑人。這些人教導眾人說耶穌基督是個好人,值得尊敬,可他自己被迷惑、被欺騙了。可他們卻叫我們去跟從那個真的受騙上當,自以為自己是神而其實什麼都不是的人。使徒約翰在約翰一書4:2裡告誡我們說,「凡靈人那耶穌基督是成了肉身來的,就是出於神的,從此你們可以認出神的靈來。凡靈不認耶穌,就不是出於神,這是那敵基督者的靈。你們從前聽見他要來,現在已經在世上了。」使徒在這裡警告那些他已經牧養了很長時間的教會說要知曉撒旦的誘惑。撒旦不會容許人相信聖經的統一性,這正如他不會容忍人去承認神說耶穌基督就是成了肉身的永生神的兒子這個事實一樣。

 在撒旦所否認的各種事實中有一樣就是基督是由童貞女所生的。神道成肉身也只能通過超自然的方式。舊約裡已經預言他要這樣降生(以賽亞書7:14),而新約則見證了他是由童貞女而生的。如果耶穌基督是由童貞女所生,那麼只有耶穌就是神所應許的那位,也就是耶穌自己所說的他是神的兒子的說法能夠解釋這一切。為了否定基督的神性,撒旦就非得去破壞基督由童貞女所生這個事實。這也是聖經中最受攻擊的一個地方。報紙上經常有報導說某個位居高官的人公開否定聖經裡的這一條,這蒙蔽了一些本來敬畏聖經上的話語的信徒。這樣的行為使他們掉進了撒旦的陷阱,儘管他們有時還披著正義使者的外衣,但他們已經成了撒旦所使用的工具。他們都是騙子,而這些騙局到頭來都不會有什麼好結果的。

 還有另一個撒旦也絕不會妥協的領域裡,在任何情況下他都不能容忍人去相信從基督的寶血裡能得到救贖這個真理。如果撒旦痛恨神的話語,如果撒旦痛恨關於基督人性和神性的教導,他最痛恨的該是關於基督為世人灑寶血的教導。彼得在彼得後書2:1裡說,「將來在你們中間也必有假師傅,私自引進陷害人的異端,連買他們的主他們也不承認。」請留意「連買他們的主他們也不承認」這段話。這引出了有關救贖的教導。救贖的意思就是用買賣的方式獲得自由。而根據聖經這贖價就是耶穌基督的血。基督為了我們能夠不再做罪的奴仆,從罪的捆綁下解脫出來,使我們最終脫離罪,不至於受到罪的懲罰,他自己死在了十字架上。因為我們都犯了罪,所以我們欠神的是死的債。「罪的代價乃是死。」耶穌基督來替我們還了這個債,並且一次全部付清。這筆交易是公平的。我們的罪是死,而他就我們撒了自己的血。可以說,所有的教導中,撒旦最恨的就是這一條關於基督的寶血的教導了。

 我們可以很容易地看到撒旦是如何宣揚他的魔鬼的教導的。我們所領受的教導的根基是聖經的權柄。當撒旦來宣揚他的教導的時候,他就派一個人假冒公義的使者,用他的教導去代替神的話語。每一種假學說、支派和所謂的「主義」等等,它們的存在都需要對聖經補充些內容,增添啟示。在每一種受人鼓吹的假教導中,都是有一些人的作品被人提高,超過了神的話語,他們這些人的地位也被高舉到權柄超過聖經的地步,所以他們就成了那些學說的基礎。撒旦就是通過這種方式用人的話代替了神的話語,可受責備的卻是神的話語而不是那些人用自己的筆寫出來的東西。所以說,當我們來到人的面前要把聖經的真理告訴他們的時候,我們一定要讓他們看神的話語。只有這樣才能做成神的工。

 當神把耶穌顯明好讓人都能信他跟從他的時候,撒旦的對策就是要把耶穌基督撇到一邊,不讓人都去注意耶穌,把人的注意力分散到其他人的身上。世界上有許多人沒有拜倒在主耶穌的名下而是拜倒在其他的名下。這些人到底是穆罕默德、佛、孔子還是其他什麼哲學家或是什麼宗教、政治領袖都沒有什麼區別。重要的是撒旦通過這些人把人的心思從耶穌基督那裡搶奪了過來。歌林多教會就遇到過這樣的危險。假師傅來到歌林多教會並沒有宣揚別的福音,但歌林多教會卻因為把這些師傅的地位抬舉過高,甚至超過了基督的權柄,而使信徒們分別去跟從保羅、亞波羅或是磯法,結果使歌林多內部產生了紛爭和分裂。

 當撒旦面對人因著對耶穌基督的信而得救的真理的時候,他就不得不用其他一些東西來假冒救贖的計劃:他那些公義的使者拋掉神的律法,勸說人相信只要他們受洗、領受了主的聖餐並參加教會,他們就得救了;或者如果他們行善到一定的程度就會得救。這些都是撒旦的欺騙,因為彼得曾宣告過聖經的真理說普天下沒有其他的名我們可以靠著得救,我們只有靠著耶穌基督的名才能得救。

 當撒旦發現有人全心地信服聖經裡的教導,相信聖經是在神的啟示下完成的,各卷書都是完整統一的,他相信耶穌基督是永生神的兒子,為拯救我們而道成肉身,他相信只要憑著對耶穌寶血的信心就能得到救贖,撒旦會放過他嗎?撒旦會向神承認這些都是真!他會向神承認自己的失敗!但是撒旦不會放過他,他會想盡辦法在真理的方面迷惑他,讓他看不清救贖完全是通過對耶穌基督的信心靠著他的恩典而來的這個事實。

 撒旦對是血、被救、重生一類的詞語特別憤恨。要是我們傳福音的時候可以迎合別人,不告訴他們本是罪人,把這些詞語都刪去,再告訴他們需要得救,而且他們靠著對耶穌基督的信心就能重生,那麼引領他們相信基督那一定容易地多。人沒有得救是因為他對耶穌的人性並不了解。他得救了是因為他接受耶穌基督為他個人的救主。人沒有得救是因為他「只是把一些事情交給耶穌來照管」;他得救了是因為他接受了耶穌基督並相信他的血能夠救贖人的罪。為了讓福音更接近那些沒有得救的人,我們把那讓人生畏的十字架也放到了一邊,我們也不想對那些說,「你是個失喪的人」,因為我們怕這種話會激怒他。當然,這種話會若得他不高興,但人如果不意識到他本是失喪的人,他就無法來到耶穌基督的面前。

 保羅在這方面是我們的楷模。我們需要時刻儆醒,免得撒旦勝過我們。使徒在加拉太書5章裡有段有趣的話,他在11節裡說:「弟兄們,我若仍舊傳割禮,為什麼還受逼迫呢?若是這樣,那十字架討厭的地方就沒有了。」基督對尼哥底母說,「你們必須重生。」如果你說「現在的人不明白這其中的意思。」是的,他們或許不明白,但你只要肯花1、2分鐘他們就能明白。「人不願意聽到關於血的事情。」是的,但必須有人告訴他們只有靠著血才能得拯救。我們作為敬畏神話語的人,作為尊崇基督人子權柄的人,週圍相信救贖的真道,相信得救是完全依靠對基督寶血恩典的信心的人不應該試圖讓福音的道理看起來對那些不信的人更有吸引力。如果我們這樣做了,我們就變成了那惡者的工具了。

 神愛世人,甚至賜下他的獨生子來拯救世人。神現在把信徒們派送到世界裡就是要告訴世人神的真理。這真理必須按照神的方法,憑著神的權柄來表現出來,這樣人才能認識到他們本是失喪的,他們需要救主,需要耶穌基督為他們的唯一救主。福音是神的能力,這能力能拯救世人。福音能使人在神的面前匍匐拜倒,認識到他們本是不可饒恕的被咒詛的,唯有福音可以赦免他們的罪,並使他們有永生。我們必須要小心,免得我們給魔鬼留了地步,反而使我們阻礙了神的真理。

 我們最後在對那些從未接受過耶穌基督為你個人救主的人說一句。或許你曾經被撒旦的一個使者迷惑了,使你覺得聖經沒有權柄,基督只不過是個好人,你可以靠你自己得到拯救。根據聖經的權柄,只有一個人可以救你而這個人就是耶穌基督。只要你說:「我是個罪人,我願意接受基督為我工人的救主」,那麼他就會拯救你。就是這麼簡單。你願意接受他嗎?

\chapter{撒旦對傳福音的反應}
\label{sec:ch12}
\hyperref[sec:ch11]{[上一章]}
\hyperlink{toc}{[返主目錄]}
\hyperref[sec:ch13]{[下一章]}

\begin{center}
\noindent\fbox{%
    \parbox{0.8\textwidth}{%
        馬太福音13:1-9
            \newline
            13.1 當那一天、耶穌從房子裡出來、坐在海邊。
            13.2 有許多人到他那裡聚集、他只得上船坐下.眾人都站在岸上。
            13.3 他用比喻對他們講許多道理、說、有一個撒種的出去撒種.
            13.4 撒的時候、有落在路旁的、飛鳥來喫盡了。
            13.5 有落在土淺石頭地上的.土既不深、發苗最快.
            13.6 日頭出來一曬、因為沒有根、就枯乾了。
            13.7 有落在荊棘裡的.荊棘長起來、把他擠住了。
            13.8 又有落在好土裡的、就結實、有一百倍的、有六十倍的、有三十倍的。
            13.9 有耳可聽的、就應當聽。
    }%
}
\end{center}

 在撒種的比喻裡(馬太福音13章),主向我們揭示了撒旦對神的話語破壞工作。希伯來書4:12告訴我們「神的道是活潑的,是有功效的,比一切兩刃的劍更快,甚至靈與魂,骨節與骨髓,都能刺入、刨開,連心中的思念和主意都能辨明。」撒旦知道神的話語是活的道,滿有能力。他跟你一樣都明白正是因為神的話語是活的,所以它才能生生不熄,永遠長存。有生命的種子不會永遠在地裡不發芽,它會破土長大結出果子。神的話就是神良善的種子,是有生命的種子,這種子撒在好土地上就會生長,並結出公義的和平果子。

 這就是先知以賽亞在寫以賽亞書55章時的信心,他堅信神會成全他自己的話語,會應允他自己的話語。這也是我們主在馬太福音13章裡的教導,他在8節裡說種子落在好土裡的就會結出果實,有100倍的,有60倍的,也有30倍的。他生恐人們沒有抓住他教導中的真道,有在23節裡解釋說,「撒在好地上的,就是人聽道明白了,後來有結實,有一百倍的,有六十倍的,有三十倍的。」主的話見證了神的道,神的話語落在聖靈預備好的人的心裡,人就接受了,而且一定會結出果子來。因為神的道是活的道,滿有能力,它能自己生生不熄地生長下去。撒旦對這個事實也很清楚,只要他預料到神的道要被傳播開來的時刻,他就要出來把那種子偷走。當然他更願意去控制那傳播別樣的道而不是神的真理的人,那樣他就不必去做把那種下的種子再拔出來的工作了。

 讓我們想象一下撒旦在神的道被傳開時要去做的工作吧:如果神的真道的種子放進了500個人的心田裡,他就必須得派500個鬼魔分別進入那500個不同的人的心裡去把那已栽下的種子再拔出來。如果他能讓這500個人把他的謊言當作是神的真道來領受,那麼這個效率會有多高啊。不過他知道神的道一定會被傳開,真理一定會被向世人顯明,所以他也做好了準備不讓神話語的好種子落到好土地上,這樣它就不能結出果子來了。

 在馬太福音13:3裡,主說有個撒種的出去撒種,而那撒種的人就是他自己(13:37)。「那撒好種的就是人子;田地就是世界;好種就是天國之子。」他對那些已經得拯救的人宣佈說神的話語就是好種子。

 不過,神子剛把這種子播撒下去,撒旦和他的群走卒也就開始出來活動了。在4節裡,耶穌嘆息說「撒的時候,有落在路旁的,飛鳥來吃盡了。」在每塊地的旁邊都有一些記號,為的是不浪費好土地,儘量都種上東西。而田邊的那些小路在人的腳下被踩實了,人在這路上一邊走著一邊撒種,有些就落在路上了。但落在這路上的種子是無法發芽生長的,那些跟在撒種人後面的鳥就把落在這路上的種子吃掉了。主在19節裡解釋說,「凡聽見天國道理不明白的,那惡者就來,把所撒在他心裡的奪了去;這就是撒在路旁的了。」

 落在路旁的好種子有兩個特點。首先,聽者不明白這好種子。「聽見天國道理不明白」。其次,撒旦來把種子奪走,所以這種子就無法生根發芽。「飛鳥來吃盡了。」這樣的種子也不能存活下去。這向我們揭示了撒旦對付所傳的神的道的首要策略。第一,當神的話語一被宣揚的時候,撒旦就要來弄瞎人的心眼,不讓人明白所傳的天國的道理。而如果一個人聽到了真理卻不明白,他就不能預備好去接受、去順服那真理。假如撒旦允許那種子存活下來,那種子就要發芽。從天國降下的雨水將會澆灌它使它茁壯成長;而只要有生命存在的地方,就有可能會結出果子來。所以說撒旦的首要目標就是把神的話語從你的心裡面除掉,這樣當你聽到的時候真理就會從你的這個耳朵進從那個耳朵跑出去。你如果不太注意這真理,接下來撒旦就能設法不讓它得到天國雨水的滋潤,不讓它最終結出果子來。撒旦就這樣弄瞎你的心眼,儘可能地讓你對神的真理充耳不聞。只要沒有了種子,當然就不會有生長,沒有神的話語,當然也就不能結出果子來。每個人生活所要結出的果子都要依靠對神的話語的接受程度,因為沒有活的種子就沒有果子。

 為了不讓你接受神的話語,撒旦會用別的東西來假冒神的話語。如果撒旦看見你伸出手去拿聖經,為出現的一些問題從聖經上尋找答案,他就可能會派個具有傳福音的使者的身份的人來幫助你解決問題,但靠的卻是聖經以外的解決方法。你有多少次在渴慕耶穌基督的真理的時候去尋找某些人,而你得到的卻不是滿足,而是發現你仍然還是非常飢渴?為什麼?是因為從那些人得到的東西裡面沒有生命。那不是神的話語的種子。撒旦會盡一切可能把你同神的話語分開。聖經裡沒有寫到的,沒有說過的,那裡面就沒有神的話語的種子。

 撒旦也可能會帶你到某個地方,讓你的心志在沒有準備好的時候就接受神的話語。他會在你的心裡放進大迷團,以為神的話語是深不可測,無法搞明白的。許多人發現他們在主日學裡或是週日禮拜的時候學到的東西根本就不能幫助他們解決前一天晚上的難題。撒旦在神的話語還沒有被傳揚出去的時候就開始了他的工作,而你對這些都一無所知,你想的是這是段娛樂或休息的好時候,你沒有意識到撒旦正在弄瞎你的心志,弄聾你的耳朵,讓你對那些真理充耳不聞。

 有時撒旦也會讓你帶著心底裡隱秘的罪來聽神的話。只要你裡面還有沒告白的罪,聖靈就不能引領你讓你悔改,他也無法做教導的工作。聖靈的感化和教導不是同時進行的。所以如果撒旦能使你進入到罪的裡面,再把你帶著這為坦白的罪去聽神的道,你實際上無法去真正了解神的話語。不坦白的罪也是撒旦針對那些撒在路邊的種子的詭計的一部份。

 撒旦也可能讓你帶著對某個信徒的敵意或仇恨去參加聚會,而你就被那個人或你自己的事情完全佔據了。那敵意和仇恨會阻止聖靈去澆灌那為榮耀神而要結出的果子。你心底的那些苦澀會阻止你接受神的話語。

 對已傳揚出去的神的話語還有另一種結果。從5節裡我們可以讀到:「有落在土淺石頭地的,土既不深,發苗最快」。撒旦對這樣的人花的工夫要多一些。撒旦無法阻止他去聽從神的話。那人聽了就接受了,那種子就在他心底裡活了。撒旦對付這種人不得不採用別的辦法。種子栽下後就要生根、成長,最終結出果子來,除非生長的過程被抑制住。撒旦會做些什麼?5節說種子落在「石頭地」裡,20和21節主向我們解釋說,「撒在石頭地上的,就是人聽了道,當下歡喜領受;只因心裡沒有根,不過是暫時的;及至為道遭了患難,或是受了逼迫,立刻就跌倒了。「這裡講述的是這樣一種人,他們聽了神的道就接受了,他也為這道而歡喜。他也把這新發現的對基督的愛向眾人傳揚,但對他自己他開始產生懷疑。當他與外人分享在耶穌基督裡的信心的時候遭遇到各種反對、懷疑和冷漠,這些就使得他心底裡的種子乾枯了。主對他的門徒們說過,逼迫一定會來,他向我們警告說各種政治的、社會的和宗教方面的逼迫都會從外面和裡面臨到基督徒的身上。他用這種方式來提醒徒弟們來準備好迎擊魔鬼的攻擊。

 有多少人聽了神的道就信了,而當他們計算了為此要付出的代價之後就有離開了!都多少生意人從未得到屬靈的生長,因為他們害怕自己一旦把自己全部獻給耶穌基督,按照神的原則遵從他的命令之後,他們整個的家產就都沒了。又有多少年輕人在學校裡不敢承認自己相信聖經,因為他們害怕因此而失去了別人的喜愛和尊敬!這樣看來,神的話語沒有結出果子是因為在它發芽之後,逼迫就來壓制住了生長。

 我們的主在7節裡接著說,「有落在荊棘裡的,荊棘長起來,把它擠住了。」這是撒旦破壞神所撒的好種子的第三種辦法。主在22節裡解釋說,「撒在荊棘裡的,就是人聽了道,後來有世上的思慮,錢財的迷惑,把道擠住了,不能結實。」這粒種子也撒在肥沃的土裡,它甚至還沖破了壓在頭上的重重困難發出芽來,雖然長在荊棘從中,但它最後還是長起來了。撒旦對這種人下的工夫最大,因為他得從好土地中把長得很好的種子給拔出來。他要怎麼做?基督在22節裡對我們說:用世上的思慮和錢財的迷惑來使他跌倒。

 如果撒旦不能在神話語的種子發芽前把它除去,或者不能在它還沒有把根扎牢的時候把根拔去,如果那種子已經張大,準備要結出果實了,那麼撒旦就要用用屬世界的東西充滿我們。他要給我們一些新的目標,這些目標同我們從神的話語裡、從耶穌基督那裡所領受的目標是不一樣的。例如,撒旦選中了一個商人,他可能不去逼迫他(撒旦可能試過,但沒能奏效),相反他可能去抬高這個商人,給他受人尊崇的地位和權柄。這人就變得很忙,根本沒有時間來思考神的話語。這樣他的工作就完全佔據了他的時間。或者,撒旦也用過第二中方法去用貧窮來試探他,但撒旦發現這樣只是把這人更推向了神的話語,那人更去依靠主;他這時就會掉轉頭來使這人富有,讓財富去網羅那人。那人就不斷想著積累財富,也就沒有時間來思考神的話語。對這個世界的體貼佔據了我們的心,我們祇想著自己的事情,財富的欺騙使我們的腦子裡充滿了物質的東西。不管我們中了哪個圈套,我們都被自己的地位、責任、成就等等所佔據,我們就忽略了神的話。而我們如果忽略了神 的話,就一定不會結出果子來。

 或許你會想那些物質上的恩惠是來字我們的神。請允許我說,這些東西也可能是撒旦的圈套;撒旦可能就是要用這些來把神的話從你的心底裡連根拔起,好使你的生活結不出果子來。我要再次講,撒旦不在乎你做什麼,但他非常在乎你信什麼,你知道些什麼。他會讓你去做任何你喜歡做的事情,但他不能讓你去信神的話語,接受神的話語,用神的話語來衡量你的生活。他要是這樣做了,那神的話語的好種子就落到了好土地上,生根發芽,並要在神的時刻結出果子來。

 如果你在讀這本書的時候又想到了工作上的事情,有想起要解決明天的問題,清理昨天是事情,這些都不會讓人感到奇怪的。你們中的一些家庭主婦已經被家裡的事情份了心神。你想知識來自聖靈的嗎?不是。魔鬼就在這裡忙地很,拼命不讓神真理的種子被栽種上。你或許上次在聽道的時候感到有些睏倦。你想這是來自神的真平安在使你入睡嗎?不是。撒旦知道神又有好種子要撒,他就在週六晚上誘惑你,使你睏倦,不讓你第二天有好精神去聽道。他促使你晚睡,結果你在星期天清晨禮拜的時候還在打瞌睡。撒旦看見你在瞌睡就會說,「我不用在擔心這個人了。今天早上什麼種子也不會栽進他的心裡。」撒旦的工作就這樣在你的身上作成了。

 馬太福音13章所講的第一個比喻裡教導我們當神的兒子在撒種的時候撒旦在做些什麼。但那並不是撒旦工作的全部。在24節裡主又給我們講了一個比喻,這個比喻和第一個正好相反。他說「天國好像人撒好種在田裡,及至人睡覺的時候,有仇敵來,將稗子撒在麥子裡就走了。到長苗吐穗的時候,稗子也顯出來。」稗子和麥子長得很像,只有到了收割的時候,你才能把他們明辨出來。它們長得雖然像,但稗子不能結出果實來。「田主的僕人來告訴他說:『主啊,你不是撒好種在田裡嗎?從哪裡來的稗子呢?』主人說:『這是仇敵作的。』」我們看主在36節裡是怎麼向我們解釋的。這是個很奇妙的道理,所以基督的徒弟們就說「請把田間稗子的比喻講給我們聽。他回答說:『那撒好種的就是人子;田地就是世界;好種就是天國之子;稗子就是那惡者之子;撒稗子的仇敵就是魔鬼;收割的時候就是世界的末了;收割的人就是天使。』」

 在主的解釋裡想我們顯明瞭撒旦對神的道的第二中對策。撒旦也撒種。我們經常忽略了這樣一個情況,我們每週拿出一兩個小時來聽主的教導,宣揚主的道,而撒旦卻整周時間都在向我們撒種。撒旦很忙。他一點都不懶惰,他會一直堅持撒種。而且他也在為他撒的種子準備土地,好使這些種子也能結出他的果子來。每次你不接受神的話語卻去接受人的話語,你就在接受他的種子。每次你的頭腦被耶穌基督以外的事情所充滿,你就接受了撒旦撒的種子。接下來你又會問自己為什麼你的生活結出的果子那麼少,你的基督徒的生命成長的那麼少。如果沒有種下好種子,離開了神的話,你就不可能有成長。

 主在馬太福音13:8裡說「又有落在好土裡的,就結實,有一百倍的,有六十倍的,有三十倍的。」他在23節裡作了解釋,「撒在好地上的,就是人聽道明白了,後來結實,有一百倍的,有六十倍的,有三十倍的。」請注意,這裡的種子都是一樣的。無論是落在荊棘裡的,落在路旁的,落在石頭地裡的還是落在好土上的,種子都是一樣的。撒種的人也是同一個,都是耶穌基督。所以你不能用種子或是撒種人的原因來解釋結果子的不同。這其中只有一樣不同,這就是土地的準備程度是不同的。路旁的土地根本就沒有做準備,石頭堆裡的土有一點點的準備,荊棘叢裡的土有一些準備;而結出一百倍、六十倍、三十倍的土壤被準備的非常充足。

 神的話語在你生命裡結的果子跟你準備讓聖靈在你的生活和心裡擁有的權柄有直接的關係。如果你的心沒有準備就來到神的面前,那種子就會被撒旦奪走。如果你帶著充滿罪、苦澀、怨恨、嫉妒來到神的面前,神的話語永遠也不會在你裡面結出果子來。但如果聖靈在你的心裡把你准備好接受神的話語,主耶穌基督所撒下的好種子就會結出果子。

\chapter{撒旦如何誘惑人}
\label{sec:ch13}
\hyperref[sec:ch12]{[上一章]}
\hyperlink{toc}{[返主目錄]}
\hyperref[sec:ch14]{[下一章]}

\begin{center}
\noindent\fbox{%
    \parbox{0.8\textwidth}{%
        馬太福音4:1-11
            \newline
            4.1 當時、耶穌被聖靈引到曠野、受魔鬼的試探。
            4.2 他禁食四十晝夜、後來就餓了。
            4.3 那試探人的進前來、對他說、你若是 神的兒子、可以吩咐這些石頭變成食物。
            4.4 耶穌卻回答說、經上記著說、『人活著、不是單靠食物、乃是靠 神口裡所出的一切話。』
            4.5 魔鬼就帶他進了聖城、叫他站在殿頂上、〔頂原文作翅〕
            4.6 對他說、你若是 神的兒子、可以跳下去.因為經上記著說、『主要為你吩咐他的使者、用手托著你、免得你的腳碰在石頭上。』
            4.7 耶穌對他說、經上又記著說、『不可試探主你的 神。』
            4.8 魔鬼又帶他上了一座最高的山、將世上的萬國、與萬國的榮華、都指給他看、
            4.9 對他說、你若俯伏拜我、我就把這一切都賜給你。
            4.10 耶穌說、撒但退去罷。〔撒但就是抵擋的意思乃魔鬼的別名〕因為經上記著說、『當拜主你的 神、單要事奉他。』
            4.11 於是魔鬼離了耶穌、有天使來伺候他。
    }%
}
\end{center}

 撒旦最初的背叛是對神的權柄的挑戰,看看到底神有沒有權柄去統治他所造的,到底神的命令可否不遵從,到底可否不信神。翻開人類歷史的長河,撒旦一直在傳播他的謊言,說他有統治的權柄,人都要聽從撒旦的話,人都要信撒旦。這個問題最終一定要解決的,因為不能有兩個主同時在一個地方稱王掌權。兩個完全相反的觀點不可能都是真理。兩個不能同時都有權被人敬拜。撒旦自己明白他只不過是被造之物,他的生命是被造的,他有一天將不得不承認神才是那創造者,只有他才有那不是被造出來的生命,但是撒旦要想方設法來阻延這個時刻的來臨。而根據神的時間,當這場爭戰無法再拖延下去,必須解決的時候,那個時刻也就到來了。

 在我們的主耶穌出來進行傳道事工之前,他先來到曠野裡受到了撒旦的試探,他在這場與撒旦的爭戰中打敗了魔鬼,為我們一次性地解決了這個問題。從對這件事的簡述中我們有時會以為耶穌來到曠野,是因為他被魔鬼追趕,他想找個地方來躲避一下好使得撒旦找不到他,好使撒旦不能試探他。但事情正好相反。撒旦是被追趕的對象。神配得被遵從嗎?神的話值得相信嗎?神配得敬拜嗎?撒旦知道他與基督相爭戰最後的結果是什麼,當然後來就趕緊逃跑了。基督在曠野裡是受了聖靈的指引;他在那裡是為成全神的旨意,他在那裡是為了痛擊撒旦。從4章的記載總我們可以清楚地知道撒旦試探耶穌的經過。

 根據聖經撒旦只有3條通路進入並掌握人的生命,它們就是肉體的私慾,眼目的情慾和今生的驕傲。希伯來書的作者告訴我們基督凡事都受過試探,只是沒有犯罪。作者沒有向我們強調他受到試探的數量,但他一再強調說耶穌基督受過各種各樣的試探。撒旦用過各種各樣的辦法來攻擊耶穌基督,想讓他聽從撒旦的旨意。因此我們可以很清楚的知道對基督的誘惑來自三個方面。肉體的私慾的方面,他非常飢餓;驕傲的方面,他對神的信心受到誘惑;在眼目的情慾方面,列國都擺在他的眼前。我們對這裡所記載的都不會感到太陌生。不過我們應該注意到,這裡向我們顯示了撒旦對我們誘惑的三種方式。今天,他對你的誘惑也跟以前的如出一轍。

 在馬太福音中記著說基督進入曠野與撒旦在靈裡爭戰並不是出於他自己的意思,而是根據神的旨意,因為他是被聖靈引導而進入曠野的。他靠的不是自己的力量和權能,而是聖靈無盡的能力。他進入曠野不是去尋找自己的東西,而是去解決撒旦背叛神留下的問題。聖經記著說耶穌整整40晝夜沒有進食。一直到後來耶穌才開始飢餓。馬太在這一點上說的十分清楚,「他禁食四十晝夜,後來就餓了。」

 我們無法根據自然規律來解釋為什麼人可以這麼長時間不進食卻仍舊沒有什麼不良反應。不過在約翰福音4章裡有對此的解釋。在基督前往撒瑪利亞的途中,他為了能單獨和一個在靈上十分渴求的人見面,就把徒弟們打發出去到村子裡找食物。徒弟們走後,他見到了那個在靈上十分飢渴的撒瑪利亞婦人。他想那婦人顯示了他就是從神而來,為要滿足人們需求的人子。當這些都過去之後,徒弟們也回來了,他們帶給他一些買來的食物。徒弟們就對他說,「拉比,請吃」(約翰福音4:31)。但他卻回答他們說:「我有食物吃,是你們不知道的。門徒就彼此對問說:『莫非有人拿什麼給他吃嗎?』耶穌說:『我的食物就是遵行差我來者的旨意,作成他的工。』」我們的主想門徒們揭示了他與神的旨意的關系才是那每天支持他的動力。當其他人依靠物質的東西來支持他們的身體的時候,他依靠神的旨意,神也按照他對耶穌的意思來支撐他。主能在曠野裡40天禁食,也是因為他能完全順服神的旨意。他對神的信靠就給了他支撐身體需要的力量。

 撒旦首先來試探的也就是耶穌與神旨意的關係。撒旦把基督在世上的身份作為一種假設,他說「你若是神的兒子」。我們無須提醒你聖經告訴我們魔鬼也信,不過卻是帶著戰驚(雅各書2:19)。當我們現代人毫不遲疑去否定耶穌基督的神性的時候,當他們輕率地否定耶穌基督是永生神的兒子,由童貞女所生的時候,地獄裡的天使卻沒有一個懷疑耶穌基督的這個人的。撒旦在這裡並沒有質疑耶穌的人性,他是把這一點作為假設提了出來。他說:「你若是神的兒子,可以吩咐這些石頭變成食物。」神早已賜下了食物來支持身體的需要。在神創造的時候,他把亞當和夏娃放進了伊甸園,那時他就對他的創造說:「園中各樣樹上的果子,你可以隨意吃」(創世紀2:16)。神支持身體的辦法是靠消耗食物。而撒旦卻在迷惑耶穌去做神不允許的事。耶穌面對的是關於神的命令方面的試探。在一個十分漂亮的外衣下,撒旦藉著合理的邏輯來到耶穌面前說,「你若是神的兒子,可以吩咐這些石頭變成食物。」

 對於今天許多人不願去認定的事情,撒旦卻承認了。今天的人懷疑神的話語的權柄。比如說,聖經說所有事物的存在都是因為神創造了它們。聖經教導我們宇宙萬物的存在都是靠著主耶穌基督的能力。聖經向我們指明耶穌基督就是神的兒子。而現代的人卻毫不遲疑地稱聖經裡說的是假的,神的虛構的。但撒旦還沒有現代人的心腸那麼硬。他承認耶穌基督是創造者,是神的兒子,靠著他口裡的話就會成就當初神創造的時候所顯示的神跡。跟創造宇宙相比,耶穌要把石頭變成食物簡直太容易了,撒旦自己也承認耶穌能做到這一點。所以他就迷惑耶穌說,「吩咐這些石頭變成食物。」

 那麼哪裡有撒旦的誘惑呢?神已經給了人食物來支持身體。基督在禁食40晝夜之後正需要食物。耶穌基督也有權柄讓石頭變成食物。誘惑在那裡呢?撒旦在這裡的誘惑十分巧妙,他是要讓耶穌基督背離神的旨意。因為耶穌基督是在聖靈的指引下來到曠野的,他對聖靈的順服就是支持他身體的力量,而他所忍受的肉體上的飢餓也是神旨意的一部份。飢餓也是神的計劃的一部份。如果耶穌根據自己的意思來滿足他自己的要求,就背離了神向他顯明的其中的關係,就違背了神的旨意。

 撒旦在這裡的詭計是這樣的:假如你是神的兒子,你沒有理由不能做你想做的事情。他在暗示神與子之間的關係可以允許他不按照神的意思自己去行。你既然是神的兒子,怎麼還不能根據自己的意思行呢?你不用總想著神的旨意,不用總依靠神,先滿足你自己的需要吧。撒旦首先想要耶穌做的就是不遵從神,不順服神的旨意。

 神希望我們都能夠順服。神對我們有計劃,聖經向我們說明瞭這個計劃。神很清楚地表明誰尋求聖經誰就會一步一步知道神對他的計劃。有時我們會覺得我們的兒子的身份使我有權柄來判斷神的旨意---去決定我們究竟是否要順服神的旨意,是否要繼續按照我們的好惡來行事,不根據聖經的命令而是根據我們自己的意思。撒旦也來向我們暗示說我們是神的子民,你有對耶穌基督的信心,你就不必去做他說的每句話。你有當兒子的權利。你有自己的思想;按你的意思去做。撒旦對基督的第一個試探就是要使他違背神的意願。這也是撒旦最希望你去做的。

 馬太福音4:5,6了記載了第二個試探。撒旦帶著基督「進了聖城,叫他站在殿頂上,對他說:『』你若是神的兒子;可以跳下去,因為經上記著說:『主要為你吩咐他的使者,用手托著你,免得你的腳碰在石頭上。』」聖城耶路撒冷建在山頂上,城的四週是又高又厚的城牆。城角處距地面的高度足有400英尺。耶穌基督被撒旦安置在城牆邊上,就是現在所稱的「殿頂」。撒旦讓耶穌向下面400英尺的地方看,對他說「你引用了申命記裡的文字拒絕了我第一個誘惑,『人活著,不是單靠食物,乃是靠神口裡所出的一切話。』你的回答顯示出你對神的信心。現在我倒要看看你對神的信心到底有多大!讓我引用一個應許:詩人說主要為你吩咐他的使者,用手托著你,免得你的腳碰在石頭上。如果你真得像你說的那樣相信神的話,那麼就跳下去顯示出你的信心吧!你去試試神,看他到底值不值得你信任。顯示一下你對神的應許的信心吧!」

 這個誘惑跟第一個一樣,聽起來都很有道理,也很符合邏輯。如果我們有神的話語,並且我們從神的話語中發現了相關的應許,我們就有權承受這樣的應許,我們可以依靠這樣的應許。究竟是什麼保證我們當信了主耶穌基督之後我們的罪就得到了神的赦免?除了神的話語之外我們什麼也沒有。是什麼讓我們確信對那些在耶穌基督裡面的人沒有詛咒、沒有審判呢?我們有的也只是神的話語。我們的命運都靠對神話語的相信。我們這些接受耶穌基督的人都已經把我們永遠的命運放到了天父的話語上。神的話語不值得我們信任嗎?撒旦就是這樣用神來試探耶穌。

 在撒旦這些話的背後隱藏地是他對神的話語的狡猾的誘惑。聖經裡是寫有神要派天使保護你,但這個應許是對那些順服神旨意的人說的。

 有些科學家說:「我相信在我的實驗室裡能夠證明的那些東西」,這句話很讓人懷疑。他把自己放到了審判的位置上,又要求他要得到滿足。當神說了話,而我們去實驗這話的時候,我們實際上是在說我們並不相信神。撒旦來到基督面前迷惑他去展示他到底對神的信心有多大。基督回答說,「不可試探主你的神。(7節)」耶穌基督不必先去實驗一下神的話語才去相信。耶穌基督本來就相信神。那些試驗神話語的人其實是在說只有我證明瞭這些是對的我才能相信。這是來自撒旦的誘惑!如果你想去證明它,你就是在說神是個騙子。這正是撒旦想要你去做的。你就屈服在撒旦的第二個試探之下了,撒旦的目的就是要使你懷疑神的話語。

 但基督識破撒旦第二個試探的時候,撒旦又開始了新的試探。從4:8節裡我們讀到:「魔鬼又帶他上了一座最高的山,將世上的萬國與萬國的榮華都指給他看,對他說:『你若俯伏拜我,我就把這一切都賜給你。』」神把管理這個世界的權柄交給了亞當。當亞當、夏娃受了撒旦的誘惑吃了禁果的時候,撒旦就成了這個世界的神。從亞當的墮落到如今,撒旦一直都是那篡權者。在詩篇8篇裡神指明他要把王權從撒旦的手裡奪走,並在耶穌的身上重建起來。人子將是那萬王之王,萬主之主。撒旦知道他有的只是篡權者的權柄,耶穌基督會在這個世界上彰顯榮耀。到那時他就不得不把王權歸還給耶穌基督,耶穌基督將成為萬王之王,萬主之主。只有一種情況例外,那就是使耶穌基督拜他,所有這些就會改變。

 這個誘惑裡面的詭計就是有權被拜的人就有權被遵從。如果耶穌基督顯示出任何要拜撒旦的樣子,那麼順理成章他也要遵從那有權柄受他拜的人。撒旦的陰謀就是不讓神獨有被敬拜的權柄。這是撒旦對耶穌的誘惑的頂峰。他要耶穌違背神的意願,懷疑神的話語,但這些與這個要耶穌拜神以外的人的誘惑相比都微不足道。

 這是撒旦最想從你那裡得到的。當耶穌基督復活一星期後,他向門徒們顯示了他的手和肋骨上的傷口,多馬說到:「我的主!我的神!」撒旦也想聽到這樣的話,比任何事情都更想。如果你拜他,遵從他,按照他的意願去行,那麼他會甘願把世界上的任何東西都交給你。他會控制你的意志、你的心,帶你到一個地方讓你聽他的話並回答「是的,我的主。」當他帶你到那個地方的時候,他會抬頭仰望神說,「又多了一個說我是神的人,我有權讓那些拜我的人遵從我的話。」

 這個世界已經走在不順服的路上了。這個世界在撒旦的權柄之下,應和這撒旦的話要把神趕下寶座。直到耶穌基督來到世上才有了一個完全順服神旨意的人。當基督來到客西馬尼園的時候他對天父說,「不要照我的意思。」他是在對神說,「撒旦宣稱要被遵從,要人們都信他,都敬拜他。人類都回應了他的話,但是我聽從你的旨意好使這個世界和天使都知道你才是神。你才是那該被遵從,該被人麼相信、敬拜的,除了你之外別無他人。」

 神對你的心意是要你遵從他的旨意,就是相信他的話,像耶穌基督一樣來敬拜他。你也可以用你的信心,你的敬拜,你的順服來顯明他才是那該被敬拜的,除了神之外別無他人。

 要去違背神的旨意,懷疑他的話,不把那本屬於他的感謝、讚美都獻給神真的很容易。我們請求你在神的話語的光照下檢驗你與神的旨意之間的關係,你的心裡對神旨意和天父的態度,免得你入了撒旦的迷惑。

\chapter{撒旦誘惑的步驟}
\label{sec:ch14}
\hyperref[sec:ch13]{[上一章]}
\hyperlink{toc}{[返主目錄]}
\hyperref[sec:ch15]{[下一章]}

\begin{center}
\noindent\fbox{%
    \parbox{0.8\textwidth}{%
        約翰壹書2:7-17
            \newline
            2.1 我小子們哪、我將這些話寫給你們、是要叫你們不犯罪。若有人犯罪、在父那裡我們有一位中保、就是那義者耶穌基督.
            2.2 他為我們的罪作了挽回祭.不是單為我們的罪、也是為普天下人的罪。
            2.3 我們若遵守他的誡命、就曉得是認識他。
            2.4 人若說我認識他、卻不遵守他的誡命、便是說謊話的、真理也不在他心裡了。
            2.5 凡遵守主道的、愛 神的心在他裡面實在是完全的、從此我們知道我們是在主裡面.
            2.6 人若說他住在主裡面、就該自己照主所行的去行。
            2.7 親愛的弟兄阿、我寫給你們的、不是一條新命令、乃是你們從起初所受的舊命令.這舊命令就是你們所聽見的道。
    }%
}
\end{center}

 有許多神的兒女都受了撒旦的迷惑,因為撒旦是不可戰勝的,當撒旦來試探某個人的時候,那人就沒有什麼辦法來阻攔他;他們因為撒旦有許多的詭計、陷阱,無論我們這樣掙扎反擊,魔鬼都會最終勝過我們。這是魔鬼的謊言,是用來蒙蔽我們的眼睛,為的是不讓我們看清撒旦在誘惑中所行的每一步。因為如果我們對其中的每一步都了解了,防備了他們,如果我們清楚魔鬼的那些陰謀詭計,我們就可以做好準備,在聖靈的幫助下打敗魔鬼。

 使徒彼得在彼得前書5:8裡說,「你們的仇敵魔鬼,如同吼叫的獅子,遍地遊行,尋找可吞吃的人。」撒旦吼叫是為要吸引人的注意力,我們一分散精神就認不清楚他那些詭計的實質了。在約翰壹書2:16節裡,約翰告訴我們撒旦攻擊我們的途徑。他說撒旦只能通過三種途徑來攻擊我們,而不是很多種途徑,這一點很令我們感到安慰。這三種途徑是撒旦叩開我們生命的門的三種方式,我們都它們都不太陌生。使徒約翰向我們總結說這世界上的事都可歸為以下三種:肉體的情慾,眼目的情慾,今生的驕傲。所有的罪都和這三者有關。肉體的情慾是人從亞當犯罪而來的標記,它的本質就是肉慾、貪欲等等。撒旦可能從這些肉體的情慾入手迷惑你犯罪。

 罪的第二類就是「眼目的情慾」。這顯明瞭人從本質上說不僅是屬肉體的,還是自私的;他的眼睛裡看到了就產生了貪欲,有了貪欲就想把那物據為自己所有。這和肉體的情慾有著許多的不同。人的本性裡就有貪心,而撒旦就可能利用這貪婪自私來迷惑你。

 罪的第三類就是「今生的驕傲」。人的本質是驕傲的,人的心裡渴慕那些可以使他位置昇高,顯得比別人強的東西,這些東西讓他們覺得很喜樂,使他們覺得自己可以不依靠任何人。

 當撒旦來迷惑人的時候,他的誘惑一定是基於人本性上肉體的情慾、自私自立和驕傲這三個方面的。撒旦關注的不僅是讓人受試探,他還十分關注讓人在他的試探下失敗,最終影響人的意志,產生違背神旨意的言語或行為。他最大的目標是讓人臣服在他的誘惑下,跟從他一起來違背神。撒旦知道人的心理愛慕的是什麼,什麼可以讓人聽話。於是撒旦就把種子栽在那人的心裡,再讓那種子發芽成長,那人就愛慕關切它,最終會去服侍他所愛的。這個過程可能需要很長的一段時間。種子種下去過了很長時間後那人或許才會最終愛上那個撒旦所計劃的那物,最終聽從了撒旦的旨意。也許,這個過程會很快,關鍵是讓人順服撒旦所計劃的那個東西。不管人受到任何誘惑,他都可以查驗一下那到底是屬於從哪個方面來的誘惑,這樣就能立刻分辨出來。那到底是來自肉體的情欲,眼目的情慾還是今生的驕傲?人一旦分辨出來那來攻擊他的誘惑是來自哪個方面的,他就可以用神的話語,在聖靈的幫助下回擊那從撒旦而來的攻擊。

 在我們前面所引述的耶穌在曠野裡受誘惑的事情裡,我們看到基督用神的話語來回擊撒旦的攻擊。在對來自肉體的情慾的攻擊方面,他用聖經裡的話回答說「人活著,不是單靠食物,乃是靠神口裡所出的一切話。」在第二個方面,來自眼目的情慾的攻擊中,他用「當拜主你的神,單要事奉他」來對答。在面對來自自私的誘惑中,耶穌說「我願意等到主安排的時候」。撒旦也從驕傲的方面試探耶穌,耶穌回答說「不可試探主你的神」。我們從路加福音4:13裡可以知道,撒旦從那三個方面試探完耶穌後,面對失敗他也沒有什麼更多的途徑來再繼續試探下去了。撒旦的那些試探自從誘惑過亞當夏娃直到現在都很成功,但這些在耶穌的身上都沒有奏效。基督勝過了試探,這是因為他看清了那些誘惑的本質,所以他就可以用聖經裡適當的話來抵擋試探。你會注意到基督並沒有隨便引用聖經裡的話,他分析、理解每個試探的含義,所以才能用聖經裡特別的經文和應許來回擊撒旦。我們許多人在撒旦的誘惑前摔倒了,這並不是因為這些人不敬畏神的話語,而是因為我們對神的話語掌握不牢,不能根據一定情況找到神特別的教導。而基督做到這一切。

 在撒母耳記下11章裡,我們可以讀到一個受到魔鬼的試探而沒有勝過的例子。聖經裡所記述的大衛犯罪的過程是對每個面臨撒旦誘惑的陷阱的神的兒女的警告,如果我們不明白撒旦攻擊的方法,就不會用聖經的原則來進行還擊,那我們就無法勝過撒旦了。大衛王一天晚上來到他的屋頂上乘涼,2節記著說「一日同樣西平,大衛從床上起裡,在王宮的平頂上遊行,看見一個婦人沐浴,容貌甚美。大衛就差人打聽那婦人是誰。有人說:『她是以連的女兒,赫人烏利亞妻拔示巴。』大衛差人去,將婦人接來,那時他的月經才得潔淨。她來了,大衛與她同房…」我們注意到撒旦首先是通過眼目的情慾來攻擊大衛的,第2節裡說大衛看見婦人沐浴。因為大衛和其他眾人一樣,從本性上講都是自私的,所以當他看到的時候就起了貪欲。他在那時可能想到自己是王,對他手下的子民有絕對的權柄,他有王的特權去得到他想要的任何事。

 但神的話語說的很明白。律法裡有不要犯姦淫的罪這一條。在申命記17章裡還有更進一步的指示。在那一段聖經裡,摩西在為以色列建國做准備,他預示說神有一天要為以色列立王。摩西在14節裡說,「『我要立王治理我,像四圍的國一樣。』你總要立耶和華你的神所揀選的人為王。必從你弟兄中立一人,不可立你弟兄以外的人為王。只是王不可以自己加添馬匹,也不可使百姓回埃及去,為要加添他的馬匹;因耶和華曾吩咐你們說:『不可再回那條路去。』」人會相信權力,能力,自己的本領,而不是信靠神,因此神說王不要自己加添馬匹。17節又說,「他也不可為自己多立妃嬪,可怕他的心偏邪;也不可為自己多積金銀。」

 大衛的想法是神所禁止的,不僅有律法根據,更有聖經裡特別的經文的指引。但當大衛向下看時,他的眼裡就滿了貪欲,因為他從本性上說就是自私的。眼目的情慾引發了肉體的情慾,於是他就開始計劃要使他肉體的情慾得到滿足。他看到了,就起了貪念,他就椐為己有。他的心志看到了,接著心裡有了貪心,於是意志有違背了神的話語。

 令大衛進入羞愧的誘惑跟以往撒旦誘惑其他人,使他們遵從撒旦的話,違背神的意思的方法是一樣的。

 我們在聖經中可以找到撒旦的誘惑分為三個步驟。在腓立比書4:8,使徒談到心思意念的事情,「弟兄們,我還有未盡的話:凡是真實的,可敬的,公義的,清潔的,可愛的,有美名的;若有什麼德行,若有什麼稱贊,這些事你們都要思念。」使徒在歌林多後書10:5節裡有告訴我們:「將各樣的計謀,各樣攔阻人認識神的那些自高之事一概攻破了;又將人所有的心意奪回,使他都順服基督。」

 撒旦的攻擊是從心志開始的。他通過肉體或者自私的心理或者是你的驕傲把慾望種到你的心裡。聖經說要在這種子發芽、成長之前就把撒旦栽的種子連根拔起,因為如果你容許這種子在你的心裡存在,它就要結出果子來。所以當撒旦通過心志來攻擊你的時候,你首先要被驗查你的想法,你的心思意念,這是撒旦最先接觸到你的地方。一定要靠聖靈的力量識破撒旦的陰謀,把他栽在你心裡的種子拔掉,不容許它繼續生長產生出罪來。

 接著,撒旦就要進入你的心。撒旦在你的心意裡種下貪婪的種子,讓你產生肉體的情慾或是眼目的情慾或是今生的驕傲,你就有了貪戀的的目標。這就是為什麼所羅門王在經歷過多次與撒旦的爭斗後在箴言書4:23裡所說的「你要保護你心,勝過保守一切,因為一生的果效,是由心發出。」如果撒旦種下的種子發芽,你的心就回開始貪戀撒旦為你設下的目標。所以使徒在約翰壹書2:15裡命令我們說:「不要愛這個世界和世界上的使;人若愛這個世界,愛父的心就不在他裡面了。」雅各也在雅各書4:4裡提醒我們說愛這個世界的人是淫亂的人,因為與世俗為友就是與神為敵。

 一個人不僅需要查驗自己的心思意念,並保持它的清潔,而且還應該查驗自己的喜好。人最終會聽命于自己所知道、所喜愛的東西,除非神的話語和聖靈的指引阻止你。在撒母耳記上15:22裡,撒母耳提醒掃羅說聽命勝于獻祭。如果一個人受到了私利、驕傲以及肉體上的誘惑,並且他也愛慕它們,那麼下一步就會聽命于它們。為了不讓我們去事奉罪,我們一定要淨化我們的喜好。

 保羅在羅馬書13:14裡給了我們一條非常重要的原則,「總要披戴主耶穌基督,不要為肉體安排,去放縱私慾。」因為我們把自己置身于受撒旦攻擊的環境裡,所以撒旦的工作反而為此簡單了許多。保羅警告神的兒女們不要開啟他們的心志,免得撒旦把他的思想灌輸給我們。他警告神的兒女也不要開啟他們的心靈,免得撒旦把攪擾放進我們的心。他還警告說不要把自己放進試探裡。我們讀了很多,我們看了很多,我們也在很多地方違背了保羅在羅馬書13:14裡的教導,我們把許多撒旦能使用的東西放進了我們的心意裡面。我們在撒旦的誘惑面前失敗了,因為我們太體貼自己肉體上的事。撒旦從我們給他的地步中溜了進來,因為我們不聽保羅的勸告。我們順從了自己的喜好,把耶穌基督拋到了一邊,我們貪戀這個世界,自己已經替撒旦完成了一半工作,他只需要按照我們的喜好繼續他敗壞的工作。而後,我們就又問為什麼我們會受到撒旦這樣惡毒的攻擊。那些不都是撒旦做的,他只不過借助了我們給他留的地步而攻擊了我們。使你從不順服到犯罪並不需要做太多的工作。所以說,我們不要為體貼在肉體上的貪欲。

 撒旦並沒有許多種攻擊你的辦法;他一定會從那三個方面來迷惑你:要麼是從肉體的私慾,要麼是從私利的方面,要麼就是讓你驕傲。聖經已經給了我們任何看清這些誘惑的本質,任何考神的話語和聖靈的幫助來面對這些誘惑的方法。我們要謹守我們的心思意念和心靈,不給魔鬼留地步,以免產生出罪來。

\chapter{撒旦是如何指揮的}
\label{sec:ch15}
\hyperref[sec:ch14]{[上一章]}
\hyperlink{toc}{[返主目錄]}
\hyperref[sec:ch16]{[下一章]}

\begin{center}
\noindent\fbox{%
    \parbox{0.8\textwidth}{%
        馬太福音 17:14-21
            \newline
            17.14 耶穌和門徒到了眾人那裡、有一個人來見耶穌、跪下、說、
            17.15 主阿、憐憫我的兒子.他害癲癇的病很苦、屢次跌在火裡、屢次跌在水裡。
            17.16 我帶他到你門徒那裡、他們卻不能醫治他。
            17.17 耶穌說、噯、這又不信又悖謬的世代阿、我在你們這裡要到幾時呢.我忍耐你們要到幾時呢.把他帶到我這裡來罷。
            17.18 耶穌斥責那鬼、鬼就出來.從此孩子就痊愈了。
            17.19 門徒暗暗的到耶穌跟前說、我們為甚麼不能趕出那鬼呢。
            17.20 耶穌說、是因你們的信心小.我實在告訴你們、你們若有信心像一粒芥菜種、就是對這座山說、你從這邊挪到那邊、他也必挪去.並且你們沒有一件不能作的事了。
            17.21 至於這一類的鬼、若不禱告禁食、他就不出來。〔或作不能趕他出來〕
    }%
}
\end{center}

 許多神的兒女因為不了解仇敵的本性、不知道他們在與撒旦進行什麼樣的爭戰而過著失敗的生活。我們的仇敵非常聰明也非常善於偽裝自己,以至於許多人甚至都不知道他的存在。如果我們連要打什麼樣的仗都不知道,那又怎麼可能打贏呢?保羅在以弗所書6:12裡強調說:「我們並不是與屬血氣的爭戰,乃是與那些執政的,掌權的,管轄這幽暗世界的,以及天空屬靈氣的惡魔爭戰。」和我們爭戰的仇敵我們看不見,雖然我們看不見他,但他是一個很有能力、很可怕的對手。

 我們知道撒旦只是個受造之物,他不是全能的。雖然天使是超自然的,但他們並不是神,不能像神那樣無處不在。無論是那些墮落天使還是撒旦自己,他們在墮落的時候都沒有脫去他們的本性。在路西非爾墮落之後,那些原來在他身上的限制還依然存在,他的墮落並沒有使他得到本屬於神的特權。因此撒旦也不能無所不在,同時在各處與神和神的兒女們爭戰。這樣,撒旦就不得不指揮那些跟他一起墮落的天使來和他一同爭戰。

 聖經上關於天使叛亂的事記載的很少。但聖經很明白地說撒旦帶領了一大群天使跟他一起起來叛逆神,那些天使都把撒旦作為首領,他們在聖經裡都被稱為鬼魔,在雅各書裡又被稱為魔鬼。

 保羅在以弗所書6:12裡告訴我們撒旦模仿了神的管理方式把跟隨他的魔鬼安排了不同的職份,有執政的、掌權的、管轄幽暗世界的等等。聖經沒有具體告訴我們這些職份具體的職責有哪些,但我們知道它們都有同一個目的:那就是反對神,不讓聖經所顯示出的神要拯救世人的計划得以實現。

 在啟示錄9:11裡也描述了魔鬼的國度的情況,從前十節裡我們可以知道無數魔鬼從無底坑中出來。他們被放到地上,地上的人倍受他們的折磨。在11節裡寫到:「有無底洞的使者做它們的王,按著希伯來話,名叫亞巴頓;希利尼話,名叫亞玻倫。」亞巴頓和亞玻倫的意思是要來毀滅的人。那些魔鬼在這要來毀滅的人的領導下被釋放出來,為的是要毀滅神的計劃。

 我們對魔鬼的許多事情都一無所知。但我們知道每個活著的人都有守護天使保護著他。聖經對這一點說得很明白。天使的數目很顯然比地球上所有的人,包括那些還有出生的人都多。而且撒旦所率領的那群魔鬼的數目也要比任何時候地球上的人口數目要多。

 很明顯,魔鬼跟人不同。聖經似乎向我們暗示說他們不願出現在人們中間。在馬太福音12;43裡,基督在講到魔鬼的行為是時說,「污鬼離了人身,就在無水之地過來過去,尋求可安歇之處,卻尋不著。」當人把魔鬼從身體裡趕走的時候,那魔鬼就向沙漠或者曠野逃去,遠離人的居所。以賽亞13章裡再次用象征的言語說到了這種事情,在20節裡他講巴比倫城要被滅亡,「其內必永無人煙…」,在21節裡,「只有曠野的走獸臥在那裡,咆哮的獸滿了房屋;鴕鳥住在那裡,野山羊在那裡跳舞。」以賽亞用象征的語言說這個曾經是世界文明中心的地方要被毀滅,沒有人煙,要成為魔鬼的居所。

 可以肯定的是魔鬼在那些把耶穌基督作為個人救主的人週圍,在尊崇神、榮耀神的地方是不會住地很舒服的。這也就是為什麼魔鬼在我們國家裡的活動比世界其他地方要少的原因。如果你同傳道人員談過話,你可能就會從他們的口中知道在異邦人聚居的地方,在那些對福音和神的真道一無所知的地方,魔鬼在對人造成多麼大的影響,對人施加了多麼大壓力。魔鬼對那些居住在異邦的人攻擊地特別厲害,因為他們住的地方被仇敵所佔據,魔鬼不容許傳教人員在那些地方出現。我們需要為那些傳道人誠心禱告的一個重要原因就是他們是在敵人的大本營裡與我們的仇敵作戰。他們會受到仇敵各種各樣的攻擊,不讓他們見證耶穌基督,魔鬼不光想在靈上打敗我們的傳道人,而切還要在肉體上攻擊他們好使他們無法去宣教,使他們身陷各種陷境,不讓他們的宣教結出果子來。福音空白的地方給了撒旦很多掌權的區域,這是真的。

 撒旦蒙蔽我們的一條詭計就是讓我們以為根本沒有像撒旦這樣的人,根本就沒有魔鬼。要是我們否認他們的存在,我們就不會去和他們爭戰。我們對他們的攻擊毫無防備了,他們就可以輕而易舉地打敗我們,毀滅我們。雖然很多人承認有天使存在,但他們不知道魔鬼也是真實存在的,能從各方面影響我們的生活。撒旦想讓我們相信關於魔鬼的道理只是中世紀的產物,在二十世紀的今天人們不該再相信這些觀點了。這就是在我們中間的魔鬼要偽裝自己的詭計,這樣他就能攻擊我們,使我們無法過得勝的生活。

 聖經裡有許多談到魔鬼的地方。如果我們不根據神的話語來看這些問題,那我們可實在是很愚蠢的。幸好我們可以看到神在這些相關事情和真理上對我們的啟示。從列王記上22章裡你會發現一件很有意思的事,撒旦通過他魔鬼的行經能控制政府和政治領域內的活動。神隊這個世界有他自己的心意;那就是讓他的兒子耶穌基督成為萬王之王,萬主之主。神的旨意是要叫全地的萬國萬民都稱頌耶穌基督的權柄。而撒旦一直都在破壞神的計劃,他控制了政府,不讓政府承認神的王權,而是讓他們都順服撒旦自己,這樣他就能控制整個世界。許多愛讀報的人現在都會承認撒旦在干的確實不錯。

 列王記上22章向我們表明瞭撒旦是任何在政治領域內工作的。20節說:「耶和華說:『誰去迷惑亞哈上基列的拉末去陣亡呢?』」神的旨意是要審判罪惡的亞哈王,讓他在戰場上滅亡。要讓亞哈死在戰場上,首先就必須有個仇敵要來進攻以色列,於是就有了這個問題要怎麼安排這一切。「隨後有一個神靈出來,站在耶和華面前,說:『我去迷惑他。』耶和華問他說:『你用什麼法呢?』他說:『我去要在他眾先知口中作謊言的靈。』耶和華說:『這樣,你必能迷惑他,你去如此行吧!』」

 我們看到魔鬼的靈或是魔鬼就通過戛假先知對亞哈說話,亞哈沒有相信從神而來的真先知的話卻聽從了假先知的話。假先知告訴他會在戰場上得勝,這根本就是謊言;而亞哈聽了假先知的話後就興兵打仗,最後敗死在沙場上。魔鬼們就這樣進入了政治領域,他們要推翻亞哈的目的是要阻止大衛的子孫成為萬王之王、萬主之主,重新執掌王權。

 魔鬼的工作並不僅限於更替國家的掌權者來控制政治領域,他們也活動在宗教領域裡。聖經中有好幾個地方都談到這個問題。在利未記17章裡記著說當時人們都追求宗教的表現形式,他們根據摩西律法來進獻祭物,而神卻在7節裡說:「他們不可再獻祭給他們行邪淫所隨從的鬼魔」。在申命記32:17裡又說一個人給偶像獻祭就是在向魔鬼獻祭。保羅在提摩太前書4:1裡說,「聖靈明說,在後來的時候,必有人離棄真道,聽從那迷惑人的邪靈和鬼魔的道理」---即魔鬼所宣揚的道理。在每個被建立的偶像的背後,崇拜的中心都是魔鬼。魔鬼們一直都在不遺余力地宣揚所有那些假宗教和各種形式的偶像。這也是撒旦欺騙人的計劃的一部份。他在宗教領域裡工作,為的是不叫人認識神的權柄,好使他們一直都在撒旦權勢的捆綁之下。

 福音書對魔鬼各種各樣活動的記載最多。魔鬼的活動如此盛行是因為神的兒子降到了人間,要把自己獻上成為救贖者,並要執掌王權。撒旦和他的那些同黨們於是就可是瘋狂活動,想打敗耶穌基督。在這裡我們列舉幾個例子。馬可福音1:23-26裡記著說當基督來到迦百農時,「有一個人被污鬼附著。他喊叫說:『拿撒勒人耶穌,我們與你有什麼相干?你來滅我們嗎?我知道你是誰,乃是神的聖者。』耶穌責備他說:『不要作聲,從這人身上出來吧。』污鬼叫那人抽了一陣風,大聲喊叫,就出來了。眾人都驚訝。」在路加福音4:33裡也記著說當耶穌向眾人顯現他是救主,向他們解釋聖經的教導的時候,有個污鬼就大聲喊叫起來。「在會堂裡有一個人,被污鬼附著,大聲喊著說:『唉!拿撒勒人耶穌,我們與你有什麼相干?你來滅我們嗎?我知道你是誰,那是神的聖者。』」污鬼附在這些人的身上,通過他們來反對耶穌基督,不讓基督說話,不讓以色列人來接受他。

 鬼魔沒有能力把他們自己顯現出來。他們不能自己變做什麼樣子出現在人的面前。他們都居住在我們看不見的世界裡,都是些沒有血肉屬靈的個體。他們要顯現在人的面前就必須搶奪或者控制一個身體,這個身體要麼是人的身體,要麼就是動物的身體。當魔鬼控制了一個人的時候,我們除了看到那個人行為失常之外並不能看到魔鬼的樣子。在迦百農會堂裡的一幕證明瞭這一點。那些被污鬼附著的人開口反對耶穌,而除了迦百農會堂裡的混亂之外,人們並不能看到任何魔鬼的影子。在路加福音9:38-39裡記著說,有個人來到基督面前說:「夫子!求你看顧我的兒子,因為他是我的獨生子。他被鬼抓住,就忽然喊叫;鬼又叫他抽風,口中流沫,並且重重的傷害他,難以離開他。」在這個父親看來,他的兒子很明顯是被鬼附了,證據就是他獨生兒子身上表現出來的混亂的行為。

 被魔鬼附著的人在某些方面會表現出來。實現,我們會發現魔鬼會敗壞肉體,使人產生疾病。在馬太福音9:32-33裡說:「他們出去的時候,有人將鬼所附的一個啞巴帶到耶穌跟前來。鬼被趕出去,啞巴就說出話來。眾人都希奇說:『在以色列中,從來沒有見過這樣的事。』」那些由於魔鬼的工作而產生的耳聾、身體的不適等等都是確實存在的病。但是被鬼附著的人和由於被鬼附而產生的身體的疾病還有所不同。他必須要先把鬼從人的身上趕走,然後在醫治由於被鬼附而產生的疾病。福音書裡很多地方都能夠證明魔鬼能導致身體上的反應。

 魔鬼也會影響人的頭腦。在馬太福音17:15中記著說有個男人來到基督面前說:「主啊,憐憫我的兒子。他還癲癇的病很苦。」這是魔鬼附在他兒子的身上搶奪了他兒子的神志,使他心志不清。

 魔鬼也能影響人的情緒。在馬太福音17:15裡那個父親說他兒子「的病很苦」。路加福音8:26-39裡記載了一件相似的事。一個被鬼附的人離群索居,住在墳塋地裡。人們為了看住他,不得不用鐵鏈、腳鐐鎖住他。這是一幅很生動的畫面。毫無疑問魔鬼影響了他的情緒,使他來到墳塋地,住在這個不潔的地方,因為他自己感到自己是不潔淨的。他有一個罪的情結。這個人落到這個景況不是因為他做了些什麼,而是魔鬼為了達到撒旦的目的和計劃而使他這樣的。在身體範圍內,魔鬼可以使人得病;在精神方面,魔鬼使人瘋狂;在情緒方面,魔鬼使人情緒低落,不想活下去。

 在我前些年的牧道生涯中,離我們不遠的地方有個醫生為一家精神病院工作。他曾經無數次說那些病人更需要我的事工,而不是藥物的醫治。這個醫生對那些不是由身體原因,而是由於靈的方面或是被鬼附了人通常都這麼說。在現在這個所謂基督化的美國裡,我們很多人對這種事都不相信了,不承認在二十世紀的今天撒旦還會對我們產生影響。但撒旦的活動的確很猖獗,這些都是真實的,在與撒旦的爭戰中,我們如果想要取勝,就必須知道這一點。

 魔鬼們可以隨心所欲地控制那些沒有得救的人。沒有得救的那些人都是撒旦國度裡的一員,他們都臣服在撒旦的權勢之下,對那些魔鬼的影響和行為沒有絲毫的反抗的能力。而信徒在為了滿足自己的意願的情況下也會通過魔鬼們的工作為撒旦而控制。因為信徒是神的兒女,都順服在基督的權柄之下,所以基督會阻止魔鬼住在信徒們的心中。但如果信徒自己順服了魔鬼的控制和影響,那麼魔鬼也能來控制信徒的生活。由於我們許多信徒沒有認清楚這一點,不少人都經歷了肉體的痛苦以及精神上的困擾和許多艱難險阻,這都是因為他們沒有認清所有那些困難的真正來源,卻把它們同聖經的原則聯繫起來。

 講這些事情是有些危險的,因為這可能會使人的腦子裡充滿了關於魔鬼的事情。有一次我和一個朋友在飯店裡吃早餐。在我們交談的中間他突然停下來向我抱歉,然後就低下頭來做了一個簡短的禱告。我們又重新交談起來,可幾分鐘後同樣的事情又發生了。我猜我當時的表情看起來有些奇怪,於是那位朋友就向我解釋說他在受到撒旦的攻擊,剛纔是在呼求主耶穌基督來幫助他。我說,「是什麼使你覺得魔鬼在這個地方攻擊你?」他說,「你難道沒看到有滴咖啡撒到我的領帶上嗎?這就是撒旦的攻擊。」第二次的時候是他不小心把一小塊雞蛋卷弄到了膝蓋上。我並不想說那到底是不是來自魔鬼的攻擊。他或許在屬靈的方面比我好,但有個危險就是我們心裡充滿了仇敵的事,以至於草木皆兵。但這還不是最危險的事。最危險的就是我們無法看到撒旦在我們生活中的所作作為。

 我們可以用神給我們的話語來抵禦撒旦的攻擊。我們曾經提過耶穌基督在會堂裡面對那被鬼附的人說到:「不要作聲,從這人身上出來吧!」基督有權柄,他能控制魔鬼。那些魔鬼聽了耶穌的話就出來離開了他們所附的那個人。在馬太福音17章裡記著說,當基督在山上改變形像時,他的門徒遇到一位父親領著他那個被鬼附了的兒子,但是他的門徒無法把鬼趕出來。耶穌基督回來後把那個鬼趕了出去,他的門徒就問他為什麼他們不能把哪個鬼趕出來。基督說,「是因為你們的信心小。我實在告訴你們,你們若有信心像一粒芥菜種,就是對這座山說:『你從這邊挪到那邊。』它也必挪去;並且你們沒有一件不能做的事了。至於這一類的鬼,若不禱告禁食,他就不出來。」在聖經中常常用大山來象征國家,我們相信在這裡也是這樣的。基督要說的是:如果你的信心有芥菜籽那麼大,你就能對撒旦的國度說「離開」,憑著你在耶穌基督裡的信心撒旦的國度就會逃離。但如果你不禁食、不禱告,這些都不會在你們的身上成就。禱告和禁食是完全依靠神的兩種表現。禱告就是:依靠神。因此基督說如果我們認清楚我們有他的權柄,相信我們被基督選來與他同工要打這場與撒旦爭戰的美好的仗,靠神使用他的權柄,我們就可以對任何代表撒旦國度的說,「你離開吧,」而且那人必離開。

 在馬太福音16:17裡基督對西門說,「西門巴約拿,你是有福的!因為這不是屬血肉的指示你的,乃是我在天上的父指示的。」指示給彼得看到的是基督的權柄。基督是那要來掌管一切的,而且他要掌管的不僅是這個世界,他還要掌管撒旦的國度。基督在馬可福音16:16裡說,「信的人必有神跡隨著他們,就是奉我的的名趕鬼。」「奉我的名」的意思並不是指什麼用耶穌的名來做什麼玄妙的事。基督說如果我們近前去與撒旦或者是他的使者去交戰,如果奉靠神,憑著耶穌基督的權柄,我們就可以對撒旦說,「你離開我吧,」撒旦必離你而去。

 我們在肉體的範圍內已經受了很多的苦,在精神和情緒的範圍裡我們因為是基督的使者而受到了撒旦的攻擊。可我們還不知道撒旦正在尋找機會通過我們想要打敗神的計劃,結果我們任由撒旦來壓迫我們,在各個方面給我們帶來痛苦。我們沒有憑著基督的權柄來抵抗撒達對我們的攻擊,我們本來是可以讓他離開我們的。我們現在需要注意雅各的命令:「務要抵擋魔鬼」,我們還要宣揚基督對我們的應許,「魔鬼就必離開你們逃跑了。」(雅各書4:7)

\chapter{基督勝過撒旦}
\label{sec:ch16}
\hyperref[sec:ch15]{[上一章]}
\hyperlink{toc}{[返主目錄]}
\hyperref[sec:ch17]{[下一章]}

\begin{center}
\noindent\fbox{%
    \parbox{0.8\textwidth}{%
        歌羅西書2:9-17
            \newline
            2.9 因為 神本性一切的豐盛、都有形有體的居住在基督裡面.
            2.10 你們在他裡面也得了豐盛。他是各樣執政掌權者的元首.
            2.11 你們在他裡面、也受了不是人手所行的割禮、乃是基督使你們脫去肉體情慾的割禮.
            2.12 你們既受洗與他一同埋葬、也就在此與他一同復活.都因信那叫他從死裡復活 神的功用。
            2.13 你們從前在過犯、和未受割禮的肉體中死了、 神赦免了你們〔或作我們〕一切過犯、便叫你們與基督一同活過來.
            2.14 又塗抹了在律例上所寫、攻擊我們有礙於我們的字據、把他撤去、釘在十字架上.
            2.15 既將一切執政的掌權的擄來、明顯給眾人看、就仗著十字架誇勝。
            2.16 所以不拘在飲食上、或節期、月朔、安息日、都不可讓人論斷你們.
            2.17 這些原是後事的影兒.那形體卻是基督。
    }%
}
\end{center}

 如果你身患絕症,但還想從你的醫生那裡得到一些忠告,那麼我可以告訴你我有一種醫治的辦法。我還有更好的消息。對於那些沒有患上絕症的人來說,他們生來就在罪裡面。罪阻止了人與神之間的溝通,因為神是聖潔的,他不能接受有任何不潔的東西出現在他的面前。你可以在歌羅西書2:13裡找到那個好消息,「神赦免了你們一切過犯」。

 罪人與神之間罪的阻攔已經被除去了。在罪的鴻溝上架起了一座橋樑。所有對罪人的詛咒都被拿走了。罪人死的工價也已經被贖,他們都出死入生了。這就是耶穌基督的福音的好消息。只有永生才能向我們表明神是如何使我們這些罪人如此輕易地就能得到他的恩典,他到底付出了什麼樣的代價。雖然福音的道理本身很簡單,但福音的成就卻相當複雜的事,必須要滿足聖潔的神的要求。

 使徒在他宣揚「神赦免了你們一切過犯」之後又說,「塗抹了在律例上所寫攻擊我們,有礙於我們的總局,把它撤去,釘在十字架上。既將一切執政的、掌權的擄來,明顯給眾人看,就仗著十字架誇勝。」在神宣布罪人得救之前,耶穌基督必須首先要與撒旦爭戰,去打那美好的仗並且得勝,他要宣告仇敵撒旦已被打敗。

 使徒保羅對羅馬人的律法相當熟悉。根據當時嚴格的程序,羅馬當局要對控方的證人進行盤問,而證人的證詞要被呈現到羅馬法庭上。在最後要審判的時候,原告必須要站在被告的面前,在法官的面前宣告他的證詞。在保羅的心裡一定有這樣一個情景:基督抹去了所有對我們不利的證詞,這樣指控我們的人就只剩下了撒旦自己。撒旦知道神說過罪的工價就是死。他不願意放棄對所有那些在他國度裡出生的人。當撒旦作為控告我們的人來到神面前的時候,他就會指出這個人是有罪的。撒旦對我們的控訴是正確的。除了主耶穌基督之外,所有在降生到這個世界裡的人都是有罪的。 保羅在羅馬書3:10裡說,「沒有義人,連一個也沒有。」在23節他又說,「因為世人都犯了罪,虧缺了神的榮耀。」所以說,撒旦要控告我們根本就不用花太多的精力,他可以很容易就找到控告我們的地方。

 使徒說神已經把那些對我們不利的證詞都塗抹掉了。神知道我們事實上都是罪人。神沒有對我們生來就是罪人,我們犯罪,我們在雖的咒詛之下的事實視而不見。神甚至早在撒旦控訴我們之前就知道這一切了。但在神可以接受我們成為他的兒女,在他宣告赦免了我們的罪之前,耶穌基督先要對撒旦對我們的指控做些事情。一個公正的法官不會對那些真實的控告充耳不聞的,他只有對那些誣告才會不理不睬。

 你應該還記得當法利賽人想殺死基督的時候,他們把他帶到了羅馬官員的面前,說耶穌基督有叛亂的罪。對基督審判的第一步是要調查對他的指控。彼拉多把基督從憤怒的人群中帶到了審判的大廳上來盤問他。他像調查人員一樣來查看對基督的那些指控到底是不是真實的。他問基督關於他的國度、關於他的門徒的一些事情,他看基督對羅馬的態度是怎麼樣的,看他是否要陰謀顛覆愷撒的政權。在彼拉多盤問耶穌的時候,他發現根本就沒有什麼證據能夠證明耶穌有罪。所以他就出來對憤怒的人群說「我並沒有查出他什麼該死的罪來。」

 而就在這時候,那群暴民起來催逼彼拉多,要他把耶穌釘在十字架上,彼拉多害怕他們在巡撫的面前控告他,就照著他們的意思做了。當時彼拉多正在接受羅馬當局的調查,他在這種情況下為想洗淨自己的手,就宣告說耶穌受人誣告,不能在法庭上給耶穌定罪,但他最後還是順從了眾人的意願。我們重提這件事的原因就是要告訴你根據羅馬法律,所有的指控必須要在被告被判決前先進行調查,查驗事情的真相。

 使徒在歌羅西書2:14裡指出,我們要受到神的審判,那裡將會有些「有礙于我們的字據」。這就是說要有合法權柄的人來進行審判,對我們進行的控告都是真實的。當撒旦站在神的面前指證我們的時候,會有充足的證據來證明我們是罪人、是神所不能接受的人這一事實。誠實可信的神不會因為我們的確犯過罪就對這個案件不加理睬。神也不能審判了我們,又把我們這些罪人接進榮耀之中,因為我們的罪污會弄臟天國。

 在這個時候,耶穌會到審判所來幫助你。當一個人在羅馬法庭上被判有罪的時候,只有一個解救他的辦法,那就是替這個人承擔所有的處罰。耶穌為了替我們解決撒旦對我們的控訴,他必須要在我們的位置上替我們死。所以使徒說基督為了我們的罪而被釘死在十字架上。

 罪犯被判罪名成立並給他訂了罪後,法庭就會發佈一個正式的文書用來解釋犯罪的性質和定罪的依據以及最後的判罰。如果這人被關在監獄裡,這份文書就會被釘在牢房的門上。而當一個人犯了死罪,這文書就會被要被釘在十字架上。當彼拉多決定要處死耶穌的時候,他用三種語言寫下了一份文書,上面的內容是:「拿撒勒人耶穌,猶太人的王。」這其中的意思是這人被釘死在這裡是因為他犯了叛亂的罪。

 耶穌基督死的時候,在那十字架上還釘有另一條罪名,這不是羅馬人釘的,而是全能的神自己親自釘上的。寫在那上面的都是你我的罪。神要向我們問罪的所有內容都寫在那上面,這樣當榮耀的眾天使經過耶穌基督在骷髏地被釘的那個十字架的時候,他們就都能讀到你我的罪,那意味著耶穌也是為我們的罪死在十字架上。耶穌基督替我們擔當了撒旦以及所有其他人可能在神面前控告我們的那些罪名,他為了你我這些罪被釘死在十字架上。

 使徒在希伯來書2:14裡說:「兒女既有血肉之體,他也照樣親自成了血肉之體,特要藉著死敗壞那掌死權的,就是魔鬼,並要釋放那些一生因怕死而為奴仆的人。」使徒在這裡強調說耶穌是為了使你所有的罪都能得到神的寬恕,拿掉撒旦對你不利的指控,也基督必須要替你在神的面前償還所有債務。你我的罪都是死罪,所以耶穌基督只要替我們死才能償還我們的罪。耶穌基督如果活著就不能替我們贖罪,因為罪的代價是生命,就是死。當耶穌帶著我們所有的罪來到十字架前的時候,對我們一切的控告都被他一個人擔當了。當你再看到十字架的時候,請別忘了是神為了你的罪而把耶穌基督放在了十字架上。

 使徒在歌羅西書2:15裡指出的第二點就是饒恕你所有的過犯是非常必要的事。耶穌基督不僅是為你的罪,也是為了能最後戰勝那在神面前控告你的那些人而死的。這就是使徒所說的:「將一切執政的、掌權的擄來,明顯給眾人看,就仗著十字架誇勝。」執政的、掌權的在聖經裡經常被用來指撒旦和他下面所有那些魔鬼(以弗所書6:12)。為了要使所有的人能得到拯救,耶穌基督必須要戰勝那控告我們的人,這人就是撒旦。

 我們從希伯來書2:14裡可以得知,是魔鬼在想方設法要毀滅人,把人們都控制在死亡的權勢之下。耶穌基督給我們的唯一能戰勝撒旦的辦法就是他為人而死,然後再從死裡復活。撒旦能夠毀壞生命,但是他不能創造生命。撒旦能使人在肉體上死去,但他無法使人再次復活過來。所以使徒在14節裡談到基督的死之後,在歌羅西書2:15裡又談到了基督的復活,讓我們明白基督的復活是神勝過撒旦的結果。撒旦不能來到神的面前為了同樣的罪再次指證你,因為耶穌基督已經抹去了對你的那些控訴,因為耶穌基督已經替你還清的所有的債。

 使徒在歌林多前書15:3裡為福音做了一個總結。這裡麵包括兩個偉大的事實。第一,基督按照聖經為我們的罪而死。他被埋葬就是他為我們死的證據。福音第二個偉大的事實是基督從死裡復活,許多人在他死後又看到了他就是證據。基督的死和基督的復活是我們得到救贖的兩大事實根據。

 保羅在歌羅西書2:15裡說:「明顯給眾人看,就仗著十字架誇勝。」保羅在這裡提到的內容還是跟羅馬法庭的習俗有關。正如我們剛纔所提過的,對罪犯的控訴書要被貼在關押他的牢房的門上,好使所有經過的人都知道他被人控告,已經根據他的罪行對他施行了懲罰。當一個人刑滿被釋放出來之後,那張控訴書就會被人從那門上揭下來,原先審判過他的那個法官要在那張文書上面寫下「tetelestai」,這個詞的意思就是「刑罰已滿」。那個刑滿被釋放的人要把這張文書釘到他自家房門上。如果有人問他怎麼出獄了,他就可以指著那張法官簽發的文書大膽的回答說已經「tetelestai」了。他可以不必擔心他的安全,因為「tetelestai」保證他已經被釋放,他重新獲得了自由。

 當耶穌基督被釘在十字架上,與聖靈分離前的那一剎那,他對天父說,「成了!」原文中其實用的就是「tetelestai」。耶穌基督當時在說什麼呢?耶穌知道你我的罪都已經被神寫進了罪狀並將那罪狀釘到了十字架上,耶穌基督用他自己的血償還了我們欠下的所有的刑罰,並在那罪狀上寫下了「tetelestai」。在神的面前有撒旦對你的罪的控訴。罪的工價就是死,但是耶穌基督自己來到了十字架上,替我們還清了我們欠下的那些罪的債。耶穌用他的死成就了這一切。我們的罪都已經被完全還清了。現在神要對你的刑罰都已經撤消了。但是,如果你拒絕承認耶穌基督是你個人的救主,那情形就如同耶穌基督沒有死一樣,撒旦對你所有的控訴還擺在那裡,你要為你罪付出代價。

 神已經賜給了你那份赦罪的文書,你收到這份文書了嗎?你把它釘到屋門上了嗎?你現在是否安全?還是你把它丟到了一旁,因而使你的罪依然還沒有被償還?認識耶穌基督就是體驗罪得赦免,得到永遠的生命的過程。而離棄基督就是讓撒旦對你永遠控訴下去。神可以赦免所有的罪,但只有一條除外,那就是離棄耶穌基督,不承認他是你的救主,不承認他已經替你償還了所有的債。

\chapter{信徒勝過撒旦的權柄}
\label{sec:ch17}
\hyperref[sec:ch16]{[上一章]}
\hyperlink{toc}{[返主目錄]}
\hyperref[sec:ch18]{[下一章]}

\begin{center}
\noindent\fbox{%
    \parbox{0.8\textwidth}{%
        以弗所書2:1-10
            \newline
            2.1 你們死在過犯罪惡之中、他叫你們活過來.
            2.2 那時、你們在其中行事為人隨從今世的風俗、順服空中掌權者的首領、就是現今在悖逆之子心中運行的邪靈.
            2.3 我們從前也都在他們中間、放縱肉體的私慾、隨著肉體和心中所喜好的去行、本為可怒之子、和別人一樣.
            2.4 然而 神既有豐富的憐憫.因他愛我們的大愛、
            2.5 當我們死在過犯中的時候、便叫我們與基督一同活過來.(你們得救是本乎恩)
            2.6 他又叫我們與基督耶穌一同復活、一同坐在天上、
            2.7 要將他極豐富的恩典、就是他在基督耶穌裡向我們所施的恩慈、顯明給後來的世代看.
            2.8 你們得救是本乎恩、也因著信、這並不是出於自己、乃是 神所賜的.
            2.9 也不是出於行為、免得有人自誇。
            2.10 我們原是他的工作、在基督耶穌裡造成的、為要叫我們行善、就是 神所預備叫我們行的。
    }%
}
\end{center}

 我們可以經常聽到有人把撒旦尊稱為王。如果我們這樣稱呼他,我們其實是在把些本不屬於他的權柄,那些本屬於神的權柄歸給了他。如果神的兒女把那些權柄歸給了撒旦,他就順服了那他認為所掌權的,就會在撒旦的搖動下偏離正路。我們一直鬥爭根據聖經的教導在了解我們的仇敵撒旦以及他做事的方法和他的目的.如果我們要打敗這個仇敵,我們就必須了解神給了我們怎樣勝過撒旦的權柄,因為我們是屬耶穌基督的。

 神的話語很清楚地表明耶穌基督是萬物的創造者。聖經以一條很簡明的敘述開始,「起初神創造天地。」聖經接著繼續敘述神子創造的工作。在歌羅西書1:16,保羅斷言說,「因為萬有都是靠他(就是耶穌基督)造的,無論是天上的,地上的;能看見的,不能看見的;或是有位的,主治的,執政的,掌權的,一概都是藉著他造的,又是為他造的。」使徒講述有位的、主治的、執政的、掌權的時候,他所指的是聖經裡反復提到過的天使的國度。在約翰福音1章裡,使徒說「萬物都是藉著他造的;凡被造的,沒有一樣不是藉著他造的。」耶穌基督就是創造者,他不僅創造了整個宇宙,天使的國度也是他建立的。

 路西非爾也是神所創造的,以賽亞書14章和以西結書28章裡記著說他起來背叛了創造他的神。作為一個被造之物,他本應該順服創造他的神,但他卻宣稱自己和神沒有關係,而且他還帶領著無數天使一起來反對神。這些跟隨路西非爾的天使就成了魔鬼,在耶穌在世期間與神的使者作對。

 我們從福音書中可以得知耶穌基督幾乎時時刻刻都在同那些魔鬼做斗爭。這些魔鬼都是屬靈的,具有人格,他們視撒旦為王,處處與神作對,破壞神的計劃,反對耶穌基督。所以基督許多次神跡都與趕鬼有關也就不足為奇了。耶穌把人從撒旦的捆綁下釋放出來,使這些人免受肉體和精神上的痛苦,不被魔鬼所控制。結果,耶穌基督趕走了污鬼,醫治好了病人。

 這一點非常重要,因為耶穌基督來到世上並不僅是來救贖世人的,他也是要來做王掌權的。他來的身份不僅是救主,也是主。他來到世上要奪取撒旦從亞當手裡強去的王權,並建立起一個嶄新的國度。倘若耶穌基督要到地上做萬王之王、萬主之主,他必須要勝過那篡權的撒旦。耶穌每次針對魔鬼的神跡都是對他勝過撒旦的權柄的考驗。這些神跡的顯明了基督的權柄。

 當基督面對魔鬼的時候,魔鬼們都毫無疑問地承認了他的權柄。使徒雅各(雅各書2:19)說魔鬼也信,卻是戰驚。雖然人對基督有所懷疑,不相信他,但地獄中的每個魔鬼卻都對耶穌基督絕對的神性毫不懷疑。沒有魔鬼出來否定基督作王掌權和審判的權柄。而且那些魔鬼也知道最後的審判就要來臨。這也就是為什麼雅各說魔鬼也信但卻是戰驚的含義。那些魔鬼知道耶穌基督要來彰顯他的權柄。

 在福音書中我們可以找到不少污鬼拜服在基督權柄之下的記述,那些污鬼都很清楚地知道基督有絕對的權柄來管轄他們。

 馬可福音5章裡記著基督曾經遇到過一個被鬼附的人。基督從9節開始問附在那人身上的污鬼說,「你名叫什麼?回答說:『我名叫群,因為我們多的緣故。』就再三的求耶穌,不要叫他們離開那地方。在那裡山坡上,有一大群豬吃食。鬼就要求耶穌說:『求你打發我們往豬群裡附著豬去。』耶穌准了他們,污鬼就出來,進入豬裡去。於是那群豬闖下山崖,投在海浬,淹死了。豬的數目約有二千。」足足附在兩千多頭豬上的污鬼竟然當初都附在那一個人的身上!不過我們在讀過這一段之後,我們應該會注意到那些鬼只有在得到基督的准許之後才能從那人的身上跑出來,在得到了基督的命令之後,他們就趕緊跑出來進入豬的身上。他們都明白耶穌基督的權柄,也都順服他的權柄。

 在馬可福音1:23,24裡也記載著與此同屬一類的事。當基督走進迦百農會堂裡的時候,「…在會堂裡,有一個人被污鬼附著。他喊叫說:『拿撒勒人耶穌,我們與你有什麼相干?你來滅我們嗎?我知道你是誰,乃是神的聖者。』」請你注意當基督站在那個被鬼附的人的面前的時候,是那污鬼第一個認出了基督。這些鬼都知道耶穌神的聖者。他們也知道耶穌基督要來審判他們這些污鬼,因為他們都問耶穌說:「你來滅我們嗎?」25節繼續說到:「耶穌責備他說:『不要作聲,從這人身上出來吧。』污鬼叫那人抽了一陣風,大聲喊叫,就出來了。」這個鬼不願離開,反抗了一番才從那人身上出來。他與基督的抗爭的表現就是使那人抽風,但他最終也無法抵擋基督的權柄,還是出來了。這些只是福音書裡面眾多例子中的幾個,我們希望能通過這些例子告訴你基督對撒旦的國度擁有絕對的權柄。

 我們前面已經提到過保羅在歌羅西書2:15裡所說過的那些話,這些話非常重要。耶穌基督再來的時候要擄來那些虛有權柄的掌權者、執政的人,而且明顯地給眾人看,仗著十字架誇勝。直到耶穌再來之前,撒旦都一直在地上擁有絕對的權柄。他把自己立為空中的掌權者,這個世界的神,對這個世界有絕對的權柄。耶穌基督在地上解救那些受魔鬼捆綁的人的時候,向魔鬼的權柄發出了挑戰。不過要等到耶穌基督再來的時候,等他親自為權柄而與撒旦爭戰的時候,藉著他的死和他的重生,他超過撒旦的絕對的權柄才會最終得以彰顯。基督的再來就是神對神子的權柄的回答。基督的死和復活都證明撒旦只是個篡權者。所以說基督的再來是神拯救世人脫離撒旦權勢捆綁的基礎。

 我們更進一步會繼續發現,天父在基督的死而復活之後就建立起基督絕對的權柄。保羅在腓立比書2:9-11裡說:「所以,神將他升為至高,又賜給他那超乎萬名之上的名,叫一切在天上的,地上的,和地底下的,因耶穌的名,無不屈膝,無不口稱耶穌基督為主,使榮耀歸與父神。」友人曾經對10節中所提到的三個區域產生過疑問。不過看來天國的事物是屬神的範疇;地上的事物指的是自然界,而地底下是撒旦的勢力範圍,或者說是魔鬼的國度。在這三個區域裡只有一位主,他有絕對的權柄,而這位主就是耶穌基督。撒旦沒有絕對的權柄;只有耶穌基督才是主,除他以外再沒有別人配得上這個名。保羅在以弗所書1:19-22 裡的禱告也給了我們同一個真理:「知道他向我們這信的人所顯的能力是何等的浩大,就是照他在基督身上所運行的大能大力,使他從死裡復活,叫他在天上坐在自己的右邊(從這裡開始注意),遠超過一切執政的、掌權的、有能的、主治的,和一切有名的;不但是今世的,連來世的也都超過了。有講萬有服在他的腳下…」保羅在這裡是向我們證實我們在腓立比書2章裡所讀到的,就是神通過基督的復活,明白地彰顯了所有的權柄都屬於他的兒子,而其他那些宣稱自己有權柄的都是在說謊。

 當耶穌基督一來到這個世界的時候,他就向撒旦發出了挑戰:他顯明撒旦沒有權柄被崇拜,沒有權柄使人信他,他也向撒旦對他的捆綁發出了挑戰。當耶穌來到十字架上的時候,他就進入了與撒旦的爭戰當中,通過他的復活他要打敗撒旦,彰顯他的權柄。而神也要把他抬為至高,讓他坐在象征權柄的寶座上,所有的人都必須遵從耶穌,信靠他。

 當我們繼續看以弗所書2章的時候,我們就會發現基督的權柄現在也屬於信徒,因為信徒是屬耶穌基督的。我們自己並沒有任何權柄能勝過撒旦。詩人在詩篇第八篇中告訴我們說天使的等級比人高,在神的創造中人就「比天使微小一點」。神作為創造者,他不是受造之物,他理所當然是至高者。在神以下是眾天使,再往下就是人類。在人類之下是動物,動物以下是植物。人不可能有權柄去管轄天使,因為在創造的時候天使的位置就比人高。所以,我們人既沒有大於天使的權柄,也沒有大于那些墮落的天使的權柄,沒有大於撒旦或其他魔鬼的權柄。從被造的角度來說,我們人比天使的等級低。如果人有勝過撒旦和整個撒旦的國度的權柄,那就是說一定有人給了他一種更大的權柄。

 這一點非常重要,我們若不意識到我們做為人本身並沒有這樣的權柄,我們就會過一個不斷失敗的生活,就要臣服在撒旦的權勢之下。我們沒有這樣的權柄,是耶穌基督賜給了我們的這樣的權柄。基督的死使他位居萬有之上。我在這裡想再次提醒你保羅在以弗所書1:20裡說過的:「就是照他在基督身上所運行的大能大力,使他從死裡復活,叫他在天上坐在自己的右邊,遠超過一切執政的、掌權的、有能的、主治的,和一切有名的…又將萬有放在他的腳下(1:22)…:他叫你們活過來…(2:1)」耶穌基督已經把他的生命和我們聯繫在一起,我們的身上打上了他的烙印。在聖靈的工作下,耶穌死的時候,我們也死了;當耶穌基督被埋葬的時候,我們也被埋葬了;當耶穌基督從死裡復活,坐在父的右邊的上,我們也一同復活並和他一同坐席。基督和信徒永遠不會被分離開來。耶穌基督做的一切事情我們也都有份參與。所以使徒說神叫他和我們都活過來,坐在父的右邊。萬有也都服在耶穌基督和你的腳下,因為(5節)「當我們死在過犯中的時候,便叫我們與基督一同活過來。他又叫我們與耶穌基督一同復活,一同坐在天上。」

 保羅在教導些什麼?基督復活的時候,我們也跟他一同復活。當神讓基督坐在他的右邊的時候,我們也與基督一同坐席。而且當神把掌管天使的國度和撒旦的國度的權柄交給耶穌基督的時候,這權柄也交給了我們,因為我們都在耶穌基督的裡面。此時,信徒就有了常人所沒有的勝過撒旦的權柄。撒旦雖然因為耶穌基督在創造的時候把他立為天使長,但撒旦畢竟也只是個天使而已。

 我們從上面可以知道信徒有了基督的權柄,可以勝過撒旦。可只要你還把你的仇敵當作是你的主,你就要聽從他的命令並順服他,因為你承認撒旦有那樣的權柄,而實際上耶穌基督通過他的死證明瞭撒旦所宣稱的權柄都是虛假的。撒旦是這個世界上最好的偽造專家,是個的大騙子。他使很多人相信他們都是無助的,他們無法抵擋撒旦,沒人能跟撒旦作對,撒旦想怎麼就怎麼,這些都是魔鬼的謊言,那些人就被他迷惑了。作為信徒,你有權柄勝過撒旦和撒旦的權能,因為你已經和耶穌基督聯為一體,你也有耶穌基督一樣的權柄。而這權柄是神恩賜給我們的。

 基督到底有些什麼勝過撒旦的權柄?在馬太福音16章,撒旦還沒來得及準備好彼得就立刻回答說:「你是基督,永生神的兒子」。為什麼?因為撒旦最不願意聽到的就是彼得稱耶穌基督是主。當彼得說完之後,基督就對眾門徒說他要到耶路撒冷去受死。在22節裡記著說,「彼得就拉著他,勸他說:『主啊,萬不可如此!這事必不臨到你身上。』」原文中這個「拉」字原本的意思是拉住某個人的肩膀,用力搖動。彼得在想方設法要主耶穌基督恢復理智。基督轉過神來,說:「撒旦,退我後邊去吧!你是絆我腳的,因為你不體貼神的意思,只體貼人的意思。」基督此刻認出撒旦在使用彼得做他的工具,用彼得的嘴說撒旦的話。於是基督直接對撒旦說:「撒旦,退我後邊去吧!你是絆我腳的。」他就這樣正面地回擊了撒旦對他說的話和他的陰謀。

 這究竟會起到什麼作用?彼得從中有學到了什麼?讓我們翻到彼得前書5:8。彼得雖然從中得到了學習,但這個代價是很高的。「務要謹守,警醒;因為你們的仇敵魔鬼,如同吼叫的獅子,遍地遊行,尋找可吞吃的人。你們要用堅固的信心來抵擋他,因為知道你們在世上的眾弟兄也是經歷這樣的苦難。」彼得想要他的羊都知道撒旦在時刻尋找機會攻擊他們。那他們該做些什麼?逃跑?他們跑不過撒旦。藏起來?那也躲不過撒旦。彼得在9節裡說:「用堅固的信心抵擋他。」注意「抵擋」這個詞。在原文中這是個詞的詞性很強烈。當撒旦把誘惑擺放在我們面前的時候,我們就想最好的辦法就是逃避。有些事情我們應該避開,但聖經絕沒有告訴我們在在魔鬼的面前逃跑。

 神戰勝撒旦的辦法不是逃脫。這根本是不可能的事。天使是屬靈的,他們能在眨眼之間從東跑到西,你又怎麼可能跑得過他們呢?當你跑地精疲力竭的時候,你會發現他正在前面等著你。神對付魔鬼的辦法是抵擋,是與其抗爭。記住,耶穌基督面對誘惑的時候也是這樣做的。基督沒有在岩石中間奔跑逃避撒旦,而是就在那裡迫使撒旦不得不與他正面交鋒。這也是彼得所學到的。他看到基督在面對撒旦通過他而施展的誘惑時候的態度是堅決抵抗,斥責撒旦。所以彼得也告訴他的羊說務要用堅固的信心來抵擋魔鬼。

 讓我們返回到雅各書4:7:「故此,你們要順服神。務要抵擋魔鬼,魔鬼就必離開你們逃跑了。」和剛纔我們看到的是同樣的話。因為撒旦被比作是吼叫的獅子,我們可以想象撒旦是無所畏懼的。可其實我們沒有意識到撒旦是個膽小鬼。當獅子殺死動物之後,它吼叫的意思其實是要嚇跑週圍那些等著想殺死它把它的獵物搶走的豺狼。它害怕豺狼來搶走他的獵物才發出吼聲的。這吼聲並不是什麼勇敢的證據,而是膽小的證據。不過撒旦用假象欺騙了你。你把他想做是無所畏懼,對你和你做的事情毫無懼怕,你正中了他的圈套。雅各告訴我們,如果我們起來積極地反對撒旦,他就必離開我們。這就是信徒的權柄。信徒因為和耶穌基督同坐在寶座上,就能行基督的權柄,這權柄是從神的寶座上來的,撒旦根本沒法抵抗地住。

 現在讓我們來看以弗所書6:13:「要拿起神所賜的全副軍裝,好在磨難的日子抵擋仇敵,並且成就了一切,還能站立得住。」這裡講地也是要抵擋魔鬼。保羅在這裡的意思是要我們用信心做盔甲去抵擋魔鬼。撒旦是膽小鬼,所以當你奉耶穌基督的名去抵擋他的時候,他就要逃跑,而你憑著基督的權柄就可以站立得住。憑著耶穌基督的聖名,你會無往不利。

 神已經拯救了我們,使我們成了他的兒女,並把我們帶進他的家好使我們能過一種得勝的生活。他沒有讓我們在他的家裡像個懦夫那樣面對仇敵,就好像那仇敵是不可戰勝,而你又是那麼地無助,每次仇敵逼迫我們都使我們跌倒。神的話語清楚的宣告耶穌基督要完全地勝過撒旦。神把所有的權柄都給了耶穌基督,而基督也賦予了我們同樣的權柄。因此,我們可以像耶穌看著彼得斥責說「撒旦,退我後邊去吧」那樣有效地抵擋撒旦。

 假如你的一個朋友正受到撒旦的迷惑,他說:「撒旦,退我後邊去吧!」你很可能會覺得很奇怪。這也顯示出我們對神計劃要我們勝過魔鬼撒旦的事了解地有多麼少。聖經裡已經把這條真理講地很明白了,神給了我們和他兒子相同的權柄。神希望我們能真的與我們能用他給我們的權柄去與我們的仇敵相抗爭。我們自己並沒有任何權柄,我們不可能離開聖靈的幫助與魔鬼單打獨斗。你不必擔心也不必害怕,更不必逃跑,因為你可以運用那與神子同樣的權柄。你可以使用那從基督被釘的十字架上發出的能力去抵擋魔鬼。

 我們中許多人始終都在試探之中,因為沒有用神給我們的權柄,整日在恐懼中生活。下次當你覺得撒旦又來攪擾你的時候,你就憑著對神的應許的信心對那個逼迫你的說,「我憑著基督的權柄和他的寶血抵擋你。」你會聽到那個魔鬼在聽到神給你的權柄之後四散奔逃的聲音。你是神的兒女。你已經和耶穌基督同坐在寶座上。基督的權柄也給了你,神希望你用這權柄去抵擋魔鬼。

 神會說謊嗎?神會不給你充足的武器就讓你去戰場嗎?當然不會。所以,如果神告訴你憑著基督的權柄去抵擋魔鬼,「魔鬼就必離開你們逃跑了。」

\chapter{把仇敵趕跑}
\label{sec:ch18}
\hyperref[sec:ch17]{[上一章]}
\hyperlink{toc}{[返主目錄]}
\hyperref[sec:ch19]{[下一章]}

\begin{center}
\noindent\fbox{%
    \parbox{0.8\textwidth}{%
        雅各書4:1-8
            \newline
            4.1 你們中間的爭戰鬥毆、是從那裡來的呢.不是從你們百體中戰鬥之私慾來的麼。
            4.2 你們貪戀、還是得不著.你們殺害嫉妒、又鬥毆爭戰、也不能得.你們得不著、是因為你們不求。
            4.3 你們求也得不著、是因為你們妄求、要浪費在你們的宴樂中。
            4.4 你們這些淫亂的人哪、〔淫亂的人原文作淫婦〕豈不知與世俗為友、就是與 神為敵麼.所以凡想要與世俗為友的、就是與 神為敵了。
            4.5 你們想經上所說是徒然的麼. 神所賜住在我們裡面的靈、是戀愛至於嫉妒麼。
            4.6 但他賜更多的恩典.所以經上說、『 神阻擋驕傲的人、賜恩給謙卑的人。』
            4.7 故此你們要順服 神.務要抵擋魔鬼、魔鬼就必離開你們逃跑了。
            4.8 你們親近 神、 神就必親近你們。有罪的人哪、要潔淨你們的手。心懷二意的人哪、要清潔你們的心。
    }%
}
\end{center}

 最近有一個參加過越戰的老兵送給我一顆從他身上取出的31口徑的子彈。這顆子彈的製作非常精良,但它擊中目標的時候,它會鑽進目標再撕裂出一個口子鑽出來,給人造成致命的傷勢。但如果現在你把它放進槍膛裡用強頂著我然後按動扳機,我根本不會害怕,因為這顆子彈裡的彈藥已經被拆除了。如果你並不知道這一點,你就會拼命跑到什麼地方藏起來,忍受著恐懼的煎熬。可如果你知道實情,你就會穩如泰山地呆在那裡。

 許多人都生活在恐懼中,在我們的仇敵魔鬼的面前被打敗,這是因為他們不明白聖經裡的原則,那些魔鬼只不過是被拆除了彈藥的子彈,根本傷害不了你。可每次魔鬼出現的時候,人們都在他面前怯懦起來,被他打敗。他們認為自己肯定要被打敗,他們不明白聖經的原則,那就是耶穌基督通過他的死已經使撒旦如同那拆除了彈藥的子彈,主已經給了我們勝利。讓我們一起來進一步研究以下雅各書4:7裡給我們的應許吧。

 雅各書4:7說:「務要抵擋魔鬼,他們就必離開你們逃跑了。」這裡的「抵擋」並不是一種消極的招架,並不是說我們在白費力氣保護自己,而是另一幅圖畫:耶穌基督的斗士出來與仇敵爭戰,他有完全的信心那領我們進入得勝的生活的必然使我們得勝有餘。當使徒教導我們「抵擋名貴」的時候,他就像是那站在陣前的將軍,他滿有權柄地命令說,「沖!」他沒有叫我們後退,他也沒有發出警告告訴士兵去隱蔽起來,他在命令我們去與撒旦爭戰,因為他已經得到保證,只要他依照聖經的原則積極抵抗,魔鬼一定會在我們的進攻下四散奔逃。

 彼得前書5:9裡也有相似的命令,彼得在那裡實際在說:「你要轉過頭來面對他。你要抵擋他。如果你積極地回擊他,他就會逃跑。那四處來威逼你的就會反過來被你趕跑。」使徒在這裡還特別強調說要用「堅固的信心抵擋他」。

 在他講到信心的時候,他不是在說神的話語向我們揭示的那全部的屬天的真理,而是指神給聖徒的信心,他是在說要靠信心站裡得住。使徒正在從全部的真理中挑選那適用于我們與撒旦爭戰的真理。那麼我們要相信的這些真理到底是什麼呢?

 第一個真理就是關於信徒的權柄的真理。在猶大書8、9節裡說:「這些做夢的人,也像他們污穢身體,輕慢主治的,譭謗在尊位的。天使長米迦勒為摩西的屍首,與魔鬼爭辯的時候,尚且不敢用譭謗的話責備他,只說:『主責備你吧!』」還記得我們前面所強調過的一點嗎?神已經給了那些在耶穌基督裡的信徒大於我們的仇敵的權柄。當路西非爾被造的時候,他被造為神的天使長。現在雖然路西非爾背叛了神成了魔鬼撒旦,但那些在天使的國度裡的天使們仍然還認地他在被造時候被賜給的權柄。米迦勒也做過天使長。根據聖經,他在被造的時候的職位只比路西非爾低一點,他直接對路西非爾負責,而路西非爾則直接對神負責。儘管路西非爾背叛了神成了撒旦,但米迦勒從沒忘記路西非爾在被造的時候比他的職份更高。

 在撒旦墮落很久以後,米迦勒和撒旦進入到摩西的身體裡彼此爭戰,就是那時米迦勒也不敢單獨與路西非爾爭戰。因為他沒有像撒旦那樣墮落,所以米迦勒此刻的職位比撒旦高地多,但他還是呼求神說:「主責備你吧!」可見,即使是沒有墮落的天使也沒有超過撒旦的權柄。那些沒有墮落的天使相信神派遣他們去爭戰一定會取勝。米迦勒並不是按照自己的意願去和撒旦爭戰的,也不是為了達到什麼個人目的,是神派他在摩西的身上與撒旦較量。可他即使是天使長也沒有足夠的權柄勝過撒旦,他只有憑著信心去依靠神。

 以弗所書2章裡的偉大真理就是當耶穌基督從死裡復活、昇天、坐在榮耀的寶座上的時候,像你我這樣的信徒也同他一樣有了權柄。神把耶穌從死裡復活,他也把你從死裡拯救過來。神把耶穌基督帶進了熱藥裡,他也同樣把你接進了那榮耀裡。保羅在以弗所書2:6裡說:「他又叫我們與基督耶穌一同復活,一同坐在天上」。因為我們與那坐在神的寶座之上的耶穌是相連的,那寶座的權柄也就屬於我們這些神的兒女,我們可以用這權柄去同魔鬼爭戰。我們掌握著那超過魔鬼的權柄,這權柄比米迦勒的還大。但正像米迦勒不敢與撒旦單打獨斗,必須要依靠神給他勝利一樣,我們也要記得我們需要神的權柄,並靠著相信去行使這一權柄。

 我們怎麼能知道這些都是真的呢?神的話語就是這麼說的。這是我們必須要相信的東西。要打敗撒旦,第一步就是邁出信心的一步,相信神,相信神的話。當彼得給那些正同吼叫的獅子爭戰的膽怯的群羊所寫的就是要他們「要靠堅固的信心」。他是在告訴他們,當他們談到仇敵的腳步近了的時候,他們要相信神說的話---他們有超過撒旦的權柄,他們可以讓撒旦逃跑。如果你認為你在撒旦面前孤單無助,如果你根本沒有得勝的希望,如果你想每次撒旦要你做的事你都沒有辦法違抗,如果你想他每次迷惑你的時候你都要跌倒,你就真的要去遵從他,要被他打敗,要跌倒了。但如果你相信你已經得到了超過撒旦的權柄,那麼你就能主動地去抵擋他,不讓他把你帶進毀滅與墮落的道路中去。

 如果我們要勝過撒旦,那麼我們還應該明白一個事實,那就是撒旦實際是個已經被打敗了的仇敵。我們要是去查一下神的話語,我們就會很奇怪地發現聖經前後在這一事實上的看法是完全一樣的。例如,讓我們翻到馬太福音8:29,我們會發現所有在地獄裡的魔鬼都知道他們的結果是被打敗。當基督來到迦百農鄉間的時候,他們遇到兩個從墳塋地裡出來的惡鬼,沒人敢從那裡經過。29節裡記著說:「他們喊著說:『神的兒子,我們與你有什麼相干?時候還沒到,你就上這裡來叫我們受苦嗎?』」受苦這個詞一定是和末日的審判有關。這些魔鬼面對面地遇到耶穌基督,他們意識到他要來審判他們,並要使他們永遠離開神。他們知道在永生的神面前要有審判,像他們這樣的魔鬼要受到懲罰。因此,當他們遇到基督的時候,他們就問他是否是來審判他們的。這一節想我們顯明瞭魔鬼們知道撒旦已經被打敗了,因為他們都是屬撒旦的,他們要一起吞下被打敗的苦果。

 撒旦也知道他要被打敗的命運。約翰在啟示錄12:12裡說,「所以諸天和住在其中的,你們都快樂吧!只是地與海有禍了,因為魔鬼知道自己的時候不多,就氣忿忿地下到你們那裡去了。」顯然,約翰在這裡描繪的是撒旦在最後一點時間裡的活動。約翰把撒旦在最後一段時間裡的窮兇極惡歸結為撒旦明白他要被下到火湖裡,所以他要抓緊一切時間去做他想做的事情。撒旦明白他一定要失敗的事實,他明白他要被審判,並被永遠地下到火湖裡。魔鬼們和撒旦都不否認這個事實,他們都明白這一點也相信這一點。

 神在另一個地方也談到了撒旦的失敗。在歌羅西書2:15裡,保羅說基督藉著他的死打敗了那些掌權著,也就是將一切執政的和掌權的擄來,拿去他們的權柄,而且要明顯地跟人看,仗著十字架誇勝。在歌林多後書2:14裡保羅也說,「感謝神!常率領我們在基督裡誇勝,並藉著我們在各處顯揚那因認識基督而有的香氣」。這是一幅將軍從戰場上凱旋的畫面。使徒在這裡把基督描繪成為得勝者,他再一次講到那普天都知道的真理:耶穌基督靠著他的死而復活勝過了撒旦,所有那些信他的人都會在他的引領下得勝。當彼得權柄那些在獅子吼叫下膽怯的群羊的時候,他也告誡他們要靠信心站立住,他要他們緊緊地依靠真理,明白神不僅給了他們勝過撒旦的權柄,而且他們要凱旋著列隊遊行,因為耶穌基督是那得勝者。

 我們知道耶穌基督從死裡復活是我們的信心的基石。當保羅向歌林多人總結福音的信息的時候,他一針見血地指出了福音的核心,他告訴我們說他在傳的是基督為我們的罪自己受死,第三天從死裡復活。通過基督的寶血,基督借他的死給了我們得拯救的基礎。基督的復活不僅象征著神把基督的死當做是我們的罪的贖價,並且信徒藉著他的死可以在每日裡勝過我們的仇敵。

 如果我們要勝過魔鬼,我們除了相信信徒已經被賜給了高于天使的權柄,相信撒旦早已經被打敗之外,還必須相信第三個真理,這就是神在雅各書4:7裡給我們的絕對的應許:「務要抵擋魔鬼,魔鬼就必離開你們逃跑了。」我們靠著神的話語給我們彰顯的真理,積極地抵擋撒旦,撒旦就一定會逃跑。在埃及的時候,當把血塗抹在門柱和門楣上的時候,天使就不能進入那家。同樣在地獄裡的魔鬼也沒有任何力量能進入被血塗抹的大門。當你在經受撒旦攻擊的時候,無論是在迷惑你犯罪,或是在你的心裡增加苦悶和失落,還是受別的東西吸引要你遠離耶穌基督的愛,不管是什麼樣的攻擊,當你懇求用耶穌基督的寶血來塗抹時,撒旦就不能再繼續攻擊你了。他只能轉身逃跑,因為他厭惡看到血。你得勝的基礎就是基督的死的工價,他的死不僅使你從罪的咒詛中解脫出來,而切使你可以脫離魔鬼隊你的攻擊。

 作為基督裡的信徒,我們也能祈求基督藉著他的死而復活給我們帶來的好處。我們只需要向仇敵們提醒說那死而復活的榮耀的基督是我們的首領,我們都接受他的命令。撒旦在戰場上掉頭逃跑是因為他本是個懦夫,他不敢去打那沒有希望取得勝利的仗。當神的兒女懇求基督的十字架,懇求那死而復活的基督給我們勝利的時候,撒旦就註定要失敗了。但可惜的是,當我們本來可以把握住就放在我們手邊的勝利的時候,我們卻離開了信心的原則,赤手空拳地去與撒旦相搏斗,然後又去問為什麼我們會失敗。你能想象還有什麼比讓我們的戰士赤手空拳到戰場上去更愚蠢的事情嗎?但這就是許多人試圖與撒旦相抗爭的辦法。你把武器丟到身後,想借助你自己的聰明才智去打敗敵人。沒有人真的自己做到了這一點。如果你想品嘗勝利的滋味,你就要憑著信心,相信神說的你在基督裡的位置。

 神沒有叫你去躲避或去逃跑或者用智慧戰勝仇敵。他叫你用神的盔甲武裝自己,然後去主動地抵擋撒旦,他要你相信神的話語,要你相信你所得到的權柄和你的勝利,他已經答應我們要使我們過得勝的生活,撒旦必然或落荒而逃。但願你在受到仇敵攻擊的時候,因為你相信神對你說你可以得勝有餘而能去主動地去抵擋撒旦。

\chapter{與魔鬼相交}
\label{sec:ch19}
\hyperref[sec:ch18]{[上一章]}
\hyperlink{toc}{[返主目錄]}
\hyperref[sec:ch20]{[下一章]}

\begin{center}
\noindent\fbox{%
    \parbox{0.8\textwidth}{%
        申命記 18:9-11
            \newline
            18.9 你到了耶和華你 神所賜之地、那些國民所行可憎惡的事、你不可學著行。
            18.10 你們中間不可有人使兒女經火、也不可有占卜的、觀兆的、用法術的、行邪術的、
            18.11 用迷術的、交鬼的、行巫術的、過陰的。
    }%
}
\end{center}

 人對未來都有著難以滿足的好奇感。儘管連現在的那些問題他都無法一一解決,可還想著將來的重擔。他想盡力知道前面路途中要發生的事情。這並不是一種新現象,自從有了人類開始它就一直延續著。如果我們翻開聖經前面一些章節,我們就不難發現人們一直都在探索著未來的事情。

 神要我們對他的計劃充滿信心。神的話語是真實的,他的話裡有許多預言,我們可以對這些預言仔細領會,加以理解。可如果一個人對這些事情的追求超越了神的話,那他就是被迷惑了,是撒旦通過他的眾魔鬼的行動欺騙了他。

 任何與魔鬼交易的行為在聖經裡都是被明令禁止的。這是個眾所週知的道理,甚至在舊約裡神就對以色列人下了這樣的命令:「行邪術的女人,不可容她存活。」(出埃及記28:18)很明顯神對行邪術的事情很認真,因為他要那行邪術的女人付出死的代價。所以我們一定要在斷定什麼是行邪術的女人這方面格外小心。我們的思想受中世紀神學以及新英格蘭時代習俗的影響很大。我們頭腦中的行邪術的女人就是專門咒詛別人的,就是說這個巫婆如果降下了咒語的話,就能給人帶來身體、精神和感情上的傷害。但這並不是神所談到的那種人。在舊約中,這種人其實是指那些知道或者能測算出未來的人。舊約中的行邪術的女人是指人(男的或女的)把自己交由魔鬼附體,通過魔鬼來顯出未來的事情。

 以色列人當時對神給他們的關於他的計劃的啟示並不知足,於是有了游戲餓想更加了解這些事的人,他們最後就通過魔鬼來得到關於未來的啟示。他們否認神給了他們足夠的啟示,不聽神的話,勾結魔鬼去了解那些在神的眼裡不適合他們在那個時候知道的東西,他們的行為使他們受到了死的判決。

 申命記18章裡講到了關於行邪術的問題。在9節神通過摩西說:「你到了耶和華你神所賜之地,那些國民所行可憎惡的事,你不可學著行。」我們知道神拯救他的兒女離開埃及而進入的迦南地是個各種族聚居的地方,但他們有一點是相同的:他們都崇拜魔鬼並在崇拜儀式的掩蓋下與鬼相交。他們的崇拜甚至還用人做祭物。神既然帶領了以色列民族來到這塊土地,就知道他們會受到那些外邦人的影響,於是就告戒他們不要那樣去行。10和11節裡描述了各種邪術:「你們中間不可有人使兒女經火…」這裡指的就是把兒女做為活祭獻給迦南的神靈。今天你如果去Byblos,你還會看到古代迦南地崇拜的遺跡,導遊將會指給你看當初把兒女獻給魔貴的那塊石板。這是神所禁絕的。

 除了用活人獻祭之外,還有其他令神難過、被神禁絕的事情。神說,「你們中間不可有…占卜的、觀兆的、用法術的、行邪術的、用迷術的、交鬼的、行巫術的,過陰的,凡行這些事的,都為耶和華所憎惡」。神在這裡列舉出許多種同魔鬼相交的形式,這些都是魔鬼在以色列人要去的迦南地所行的。占卜就是觀察鳥雀或動物的內臟來預測未來的一種方式。如果你讀過關於羅馬歷史的書,你就會對那些通過獻祭牲畜然後觀察它們的內臟來預測羅馬人武力征服結果的事情感到陌生。這些在羅馬人的軍事行動中有相當重要的地位。

 觀兆也是神所禁止的。這和通過研究星象來判定某個人的命運有關。這在當今社會十分流行。最近我看到一個報導說是一家報紙由於偶然沒有刊登占星的部份而使遭到大量讀者寫信向他們抱怨。人們說他們在那天沒辦法作出任何決定!神禁止以色列人去「觀兆」,因為這是屬魔鬼的。

 神還提到了用法術。用法術的人就是他在魔鬼的控制下使另一個人也受到魔鬼的控制,這和我們平常所說的巫師差不多。

 行邪術也是神所禁止的。這樣的人知道許多事。所以別人就覺得要沒有和他們先討教一番就很難做任何決定。

 用迷術的人是指那些借助魔鬼的能力行神跡的人。埃及的智者就有這樣的能力,他們可以模仿神要摩西行的神跡。但以理在巴比倫宮廷裡也遇到過這一類的事情。在他的身邊就有一群受魔鬼控制的巴比倫的智者,這些人通過尼布甲尼撒控制著這個王國。這些用迷術的人是另一種同魔鬼相交的人。

 交鬼的人與鬼相交,受鬼支使的人。人如果不甘心地順從是不會受魔鬼控制的,魔鬼沒有能力去主宰並控制人的意志。交鬼的人順從魔鬼,因而可以從它們那裡知道未來的事情,作為交換他們要讓魔鬼附在他們的身上。我們現在把這些人稱為靈媒。

 行巫術的就是男人行邪術。

 過陰的人能從死人那裡得到消息。魔鬼是聯繫這兩個世界的紐帶,他們也能知道將來的事情。

 從申命記18:10、11節我們就能知道當時這些東西在以色列有多麼盛行。所有這些東西的目的都是要使人們在做決定或行動的時候不去求告神的話語。有人或許會因為這種東西只是在迷信的時代才流行的。但如果我們查驗神的話語,我們就會知道直到新約時代這些東西依然十分盛行。在當今的時代裡,那些不認識耶穌基督是他們的個人救主的人群中,它們也是主要的宗教崇拜形式。

 讓我們再來看一下撒母耳記上28:1-6裡掃羅與魔鬼相交的經歷吧。對許多人來說,這一段一直不太好懂。首先,我們要注意它發生的時代背景。以色列的王掃羅遭到他最強悍的對手非利士人攻擊。掃羅本來一直都靠神的先知撒母耳來引導他,給他出主意。但撒母耳此時已經死了,所以掃羅感到在這次危機中他失去了前進的方向。掃羅此時有神通過摩西給他們的聖經,而這已足夠引導他了。可掃羅不聽從神在他的話語中給他的啟示,反而去追求別的引領。7節中他命令說:「當為我找一個交鬼的婦人,我好去問她。」他想通過交鬼的向那些死人請教解決問題的辦法。我們注意到聖經上並沒有說這種事是不可能的,聖經上說要禁止做這樣的事。掃羅知道神是禁止做這樣的事的,3節中記著說他把那些交鬼的和行巫術的人都趕出了國。儘管他明白這些道理,他還是招來了隱多珥這個婦人。這個婦人並不是巫婆,如果按字面翻譯的話她是魔鬼的女人。在掃羅允諾不害她之後,她說:「我為你招誰上來呢?」掃羅回答說:「為我招撒母耳上來。」為什麼掃羅想見屬神的先知撒母耳呢?很顯然撒母耳可能知道神在將來要行的旨意。「婦人看見撒母耳,就大聲呼叫。這一反映說明當與鬼相交的婦人看到撒母耳後一下子大驚失色,她希望那與她相交的鬼能再次出現並通過那個鬼與撒母耳相交。而當她見到撒母耳的時候她感到非常吃驚,因為她沒能和撒母耳相交,而來的也不是那與她相交的鬼。掃羅問她「看見了什麼?」婦人向掃羅描述了那人的樣子,掃羅就說那一定就是撒母耳。請注意掃羅並沒有親眼見到撒母耳,是那個婦人見到了撒母耳。掃羅並沒有通過魔鬼與撒母耳相交,他也沒有受到魔鬼的控制。撒母耳的顯現並不是因為他受了魔鬼的召喚,而是神迫 來想掃羅宣佈審判,神要把掃羅從王座上趕下來。通過這件事可以說明那些把自己交托給魔鬼的人的確有可能通過魔鬼的工作知道未來要發生的一些事情。

 我們從使徒行傳8:9裡同樣找到一個新約時代的例證。腓利是個傳道人,他在撒瑪利亞傳講基督,告訴人們基督是救主,靠著基督可以罪得赦免,不再受魔鬼撒旦的捆綁。當地有個人叫西門,是個行邪術的。他與魔鬼相交,行了許多神跡,可以揭示未來的事情。撒瑪利亞人受他的迷惑很大,他也因此而狂妄自大,說他的能力是從神而來的。人們對西門的評價是(10節):「這人就是那稱為神的大能者。」這些人都無法分清哪個是由魔鬼來的,哪個是從神而來的神跡。撒旦的目的就是要人相信他就是神,讓人們把本屬於神的讚美和崇拜都歸給了那個人。我們都還記得撒旦最渴望的就是與至高者同等,去接受那本屬於神的敬拜。神通過他的神跡以及他通過先知所彰顯的未來證明他是神。對魔鬼來說也是這樣,他也在努力地通過手下魔鬼去行神跡,去揭示未來的事情。西門就是個例證。他是撒旦手中的工具,撒旦通過這個人使得撒瑪利亞的居民把他當作了神。

 唯一能脫離魔鬼影響的就耶穌基督福音的力量。使徒行傳8:12裡說:「他們信…連男帶女受了洗。西門自己也信了;既受了洗,就常與腓利在一起,看見他所行的神跡和大異能,就甚驚奇。」撒瑪利亞那些本來聽從了西門的人因為信了耶穌基督就脫離了魔鬼的捆綁,就連行邪術的西門自己也不再受魔鬼的控制了。福音因為使徒所行出的神跡而得到了驗證。神就是這樣用腓利來向那些與魔鬼相交的人證明耶和華才是神,撒旦只是在冒名頂替他。

 撒旦的計劃從舊約時代一直到今天都進行的十分順利。他通過這些東西使人受迷惑,在真理的面前瞎了眼睛。許多人因為撒旦那群魔鬼能夠知曉將來的事情,與死人相交就把他當作真正的神。神的話語告訴我們,越是到了末世,撒旦的工作越猖狂。約翰在啟示錄12:12裡告訴我們說撒旦知道他的日子不多了,就氣氛地去到處欺騙毀滅。而我們對他的方式方法有了解的那麼少,所以許多次當我們遇到他的時候,我們自己都還沒有準備好。

 不久前有出版了一本受到讀者廣泛歡迎的書,該書的作者聲稱自己能預知未來。她曾經做過一些預言,很多都應驗了。這些預言只可能來自兩個方面:一是從神而來,而是從撒旦而來。神在新約裡說的很清楚,他來不是給人什麼新的啟示。神已經將啟示的大門關上了,因為啟示在耶穌基督的身上都得到了成就。那就只有一種可能了。這些預言是來自魔鬼撒旦。其中的一些預言已經開始迷惑了許多人,甚至許多信徒都開始相信她是帶者屬神的權柄而來的。像她這樣的人也被政府裡的一些高官請去。這就是魔鬼的行經!它控制著國家,影響著政府遠離神。

 還有其他一些魔鬼的活動也很盛行,這其中包括占星、星象等很多種。它們是撒旦以來控制人的心靈的另一種方式。很多人都見過占卜板,它看起來一點也沒什麼能傷害人的。你只要根據要求去做你就很快可以得到你想要的答案。可如果人把自己的全部都交給它來安排,他就把自己放到了撒旦的控制之下。即使你是神的兒女,如果你把自己的一切都根據那塊占卜板來安排,魔鬼也能控制你,掌握你。撒旦就是通過這樣看來無害的東西改變了人的行為,控制了人的心靈。

 神要和撒旦進行一場爭奪人心的較量。撒旦知道如果他能控制人的心,他就能控制他們的意願。撒旦在竭盡所能去影響我們的心志,而我們在另一方面又在丟棄神的話語的權柄,自己去尋求別的事或人來做為我們行為的指南。這就是為什麼使徒保羅在提摩太前書4:1說,「聖靈明說,在後來的時候,必有人離棄真道,聽從那迷惑人的邪靈和鬼魔的道理。」今天的悲劇就是人們覺得自己有資格靠神的話語坐在審判席上,而他可以隨自己的喜好去接受或拒絕,他們根本不知道撒旦已經在爭奪他們心靈的爭戰中佔了上風,已經使他們在他的控制之下。在講臺上也有這樣的爭戰,撒旦的使者拒絕神的真道,宣講魔鬼的道理。使徒保羅也是因此才那樣反復強調心志的重要性。他在腓立比書2:5裡說:「你們當以基督耶穌的心為心。」在4:8裡他又說:「凡是真實的,可敬的,公義的,清潔的,可愛的,有美名的;…這些事你們都要思念。」

 許多年前當我還在神學院的時候,一個全國靈媒會議在達拉斯舉行了,他們施展了各種異能。因為大會是公開的,我們中的六個人就決定去進去看看。我們走進了漆黑的會場,在後面的地方靜靜地坐下來。主持人介紹了很多靈媒的種類,但他在與鬼相交的過程中卻連連失敗。會場裡出現了騷動,顯然他們自己也沒有想到會出現這樣的事。那個主持人起身後叫人把燈打開,說是有什麼東西在攔阻他。他接著指向我們說那阻攔他的力量就是從我們這裡發出的。我們被迫離開了那裡。我猜在我們離開之後他們一定是又恢復了正常,因為這個大會一直持續到週末才結束。六個信徒裡聖靈的顯現阻止了撒旦能力的發揮。

 在談到那些我們看不到、感覺不到、摸不到也聞不到的東西的時候,我們可能都會感覺有些怪。可除非你知道撒旦正在每時每刻竭力控制你的心志,不讓你看到神的真理,你就不可能真正理解到你已身陷其中的爭戰的實質是什麼。

 你還記得天父在山上改變形像的時候對門徒們說的話嗎?他們在那裡看到了一個很大的神跡,他們看到基督在他們面前改變形像。但撒旦也能給人魔鬼的能力去行神跡。神對他的門徒說,「這是我的愛子,我所喜悅的,你們要聽他。」面對撒旦的迷惑和迷惑,我們出了完全把自己叫托給神的話語,依靠基督耶穌的權柄之外,再沒有什麼別的辦法可以抵擋撒旦。

\chapter{撒旦的結局}
\label{sec:ch20}
\hyperref[sec:ch19]{[上一章]}
\hyperlink{toc}{[返主目錄]}

\begin{center}
\noindent\fbox{%
    \parbox{0.8\textwidth}{%
        啟示錄 20:1-10
            \newline
            20.1 我又看見一位天使從天降下、手裡拿著無底坑的鑰匙、和一條大鍊子。
            20.2 他捉住那龍、就是古蛇、又叫魔鬼、也叫撒但、把他捆綁一千年、
            20.3 扔在無底坑裡、將無底坑關閉、用印封上、使他不得再迷惑列國、等到那一千年完了.以後必須暫時釋放他。
            20.4 我又看見幾個寶座、也有坐在上面的、並有審判的權柄賜給他們.我又看見那些因為給耶穌作見證、並為 神之道被斬者的靈魂、和那沒有拜過獸與獸像、也沒有在額上和手上受過他印記之人的靈魂.他們都復活了、與基督一同作王一千年。
            20.5 這是頭一次的復活。其餘的死人還沒有復活、直等那一千年完了。
            20.6 在頭一次復活有分的、有福了、聖潔了.第二次的死在他們身上沒有權柄.他們必作 神和基督的祭司、並要與基督一同作王一千年。
            20.7 那一千年完了、撒但必從監牢裡被釋放、
            20.8 出來要迷惑地上四方的列國、〔方原文作角〕就是歌革和瑪各、叫他們聚集爭戰.他們的人數多如海沙。
            20.9 他們上來遍滿了全地、圍住聖徒的營、與蒙愛的城.就有火從天降下、燒滅了他們。
            20.10 那迷惑他們的魔鬼、被扔在硫磺的火湖裡、就是獸和假先知所在的地方.他們必晝夜受痛苦、直到永永遠遠。
    }%
}
\end{center}

 1967年6月,以色列和阿拉伯國家之間爆發了戰爭。由於聖經裡預言過這場戰爭,所以我們每天都在仔細地關注著局勢的發展。從開羅傳來的消息來看,以色列好像受到了重創,每份報導都在說又有多少架以色列的飛機被摧毀,多少以色列的坦克失蹤,埃及軍隊又向以色列開進了多遠。那些報導預測說當天夜裡埃及軍隊就能進駐Tel Aviv地區。隨著戰爭的繼續,埃及堅信一定會取勝。不過到了週末,以色列在空軍完全被摧毀,制空權完全被阿拉伯聯軍控制的情況下,只用了兩個小時時間就取得了這場戰爭的最後勝利。可阿拉伯聯軍雖然知道敗局已定,卻仍然頑固地抵抗。

 一個人知道他已經被打敗了,可這不一定會阻止他繼續抗爭到底。撒旦已經被打敗, 對他的判決早已做出了,可這已經註定的命運並沒有使他停止與神為敵,與神的兒子作對,反對神的兒女。聖經裡詳細地告訴了我們他的那些攻擊。

 聖經裡屢次提到了撒旦的最後命運。聖經告訴我們對大膽的審判是神和耶穌基督造就定下的。讓我們回到約翰福音12章,主在這裡向門徒們宣告了他必定會最後得勝,而且他還通過一個預言性的啟示揭示了撒旦是個已經被打敗了的仇敵。當主談到他要在十字架受死的時候,他就像他說的那粒麥種落到田裡一樣,用自己的死換來了更大的收成。主說:「現在這世界受審判,這世界的王要被趕出去。我若從地上被舉起來,就要吸引萬人來跟從我。」主剛剛談到了他的死,接著又談到了他的復活。當他說「我若從地上被舉起來」的時候,他指的並不是我要從十字架上被舉起來受死。他講的是重生,他要靠聖靈的力量從墳墓中復活,從死亡的權勢下,撒旦的權勢下被舉起來,坐在全能父的右邊。他還應許當他被舉起的時候他還要吸引眾人去跟隨他。即使是在對那些被撒旦吸引過去的人,耶穌基督也說要把他們吸引過來,因為他已經戰勝了撒旦。復活就是他得勝的證據。主在31接想撒旦宣佈了判決:「現在這世界受審判,這世界的王要被趕出去。」

 基督的死就是對撒旦審判的方式,而基督的十字架就是要審判撒旦的地方。我們的仇敵早在這個世界被創造之前就開始謀劃要背叛神,他現在已經被帶到了審判席上。耶穌基督通過他的死和復活就宣告了神對這個仇敵的審判。神派自己的兒子去解決罪的問題,他也一定不會放過那罪的源頭。神把對我們的審判放到了別人的身上,並把魔鬼永遠地治罪,不叫我們的心靈再受到魔鬼的捆綁。新約的作者們屢次都提到了那對撒旦和他的同夥的審判已經宣判了。

 在猶大書6節說:「又有不守本位,離開自己住出的天使,主用鎖鏈把他們永遠拘留在黑暗裡,等候大日的審判。」已經宣告了審判,那些跟隨撒旦一同叛亂的天使都在這受審判之列。神已經咒詛了他們,最後審判的日子也已經定下來了,神只等著那天來臨。

 彼得在彼得後書2:4裡也講到了同樣的審判:「就是天使犯了罪,神也沒有寬容,曾把他們丟在地獄,交在黑暗坑中,等候審判。」彼得在這裡是引用神要審判背叛的天使的事實來證明神也要審判那些悖逆神的人。在彼得看來,這審判已經被定下了,只是執行審判的日子還沒有來到而已。

 我們在魔鬼對神的應答中也找到了見證。在馬太福音8:28裡講到基督越過加利利海進入加拉大地區,耶穌在那裡遇到了兩個被鬼附的人,這兩個人異常兇惡,以至於沒人敢從那裡經過。我們注意一下29節中魔鬼對基督不由自主發出的應答:「神的兒子,我們與你有什麼相干?時候還沒有到,你就上這裡來叫我們受苦嗎?」這些魔鬼知道自己已經被審判。他們也知道耶穌基督就是那審判他們的人,靠他的話他們就被定了罪,而他們將來一定要承擔那罪罰。他們還知道一些神的計劃和耶穌基督要來這世上掌權的時間。耶穌要來這世上顯明他的權柄,他要做的第一件事就是捆綁魔鬼撒旦,除掉他的黑暗權勢。他們明白他們受審判之時就是耶穌掌權之日。耶穌第一次來的時候要受以色列人的屏棄,他們從這一點就可以推斷出雖然那審判的人已經出現在他們面前,可審判的日子還沒有到來。因此他們就向基督告白說他是那審判官,他們要受到那屬神的審判,時候到了那先前對他們的審判就要由耶穌基督來執行。

 啟示錄20章向我們描述了執行那審判的開始的一幕。從啟示錄1911-16裡我們知道耶穌基督那時已經再次來到世上,他是騎在白馬上的得勝者。他是萬王之王,萬主之主。在他再臨後他要征服那些悖逆的國家(啟示錄19:15),20:1裡說:「我又看見一位天使從天降下,手裡拿著無底洞的鑰匙和一條大鏈子。他捉住那龍,就是古蛇,又叫魔鬼,也叫撒旦,把它捆綁一千年,扔在無底洞裡,將無底洞關閉,用印封上,使它不得再迷惑列國。」我們把這事看作是確實的。

 撒旦的工作確實阻延了主耶穌基督在這地上作王掌權,在地上建立起神的國度。當先知們告訴以色列人彌賽亞要來的時候,魔鬼撒旦就用假先知來干擾神的先知去給以色列人傳講這個信息。當耶穌基督來到世上以彌賽亞的身份為以色列人而把自己獻為活祭的時候,又是撒旦攪動宗教領袖去反對他,勸說人們說耶穌是被鬼附體,是個不敬畏神的騙子。因此這個民族就跟從了撒旦棄絕了耶穌基督。

 直到耶穌基督再來,捆綁了撒旦和他的黑暗勢力的時候,耶穌基督才可能在這地上建立起公義的國度。人的本性是墮落。撒旦因此能迷惑人,甚至迷惑神的聖者。撒旦能夠讓人背離正道,不順服神,把他們愛神的心奪去。在這個千熹年的時代,撒旦仍然很猖狂地在迷惑人,阻止耶穌基督來到這個世上稱王掌權。因此當耶穌基督再來成就神的旨意的時候,他就一定要把我們的仇敵魔鬼撒旦除去。就像基督說的那樣他要把撒旦扔在無底洞裡,用印封上,好叫主再臨的時候撒旦不能再出來迷惑列國。

 捆綁撒旦是向諸天諸地表示耶穌基督才是那萬王之王,萬主之主。神已經因為耶穌基督的死而復活把權柄都交給了他。對我們來說,耶穌的復活就是他成為我們的主和拯救者的證據。不過神還要在耶穌再來的時候向我們再次證實基督的權柄:這就是捆綁撒旦並除掉他的黑暗勢力。福音書中大部份的神跡都和魔鬼有關,許多都是為那些受魔鬼捆綁而耳聾眼瞎腿瘸的人趕鬼的事情。這就是給以色列民族的證據,它們說明基督的權柄超越了撒旦的權柄,因為基督能夠進入撒旦的國度並把那些受魔鬼捆綁的人解救出來。這也是耶穌基督再來作王的時候要做的事情。

 我們從上面可以知道撒旦自己不僅要被捆綁,那些跟從他的魔鬼也要被捆綁。這樣世界自從亞當犯罪以來就會第一次不再受到魔鬼勢力的影響。世界會一片美好,因為撒旦的壞種子不能再去影響神播撒的好種子了。沒有了撒旦謊言的迷惑,人也就能生活在公義和聖潔裡。耶穌基督也要在這世上稱王掌權,做萬王之王,萬主之主,因為撒旦已經被他捆綁起來。

 但這只是撒旦最後結局中的第一步。從啟示錄20:3裡我們知道撒旦在被關押了一千年之後要被放出來一小段時間。7-9節裡講述了他被放出來那段時間裡要做的事情。在千熹年,地球在經歷了大爭戰之後的人數凋零之後,人口的數目要重新膨脹起來。但是那些在千熹年代出生的人生下來就從他們的父母那裡繼承了罪的本性。千熹年並不是天堂。罪的本性並沒有從那些進入千熹年的人中剔除掉。人帶著罪的本性在地上掌權,他們的兒女也都帶著同樣墮落的罪。所有這些人都需要得到拯救。

 到那時福音要在整個世界裡被傳揚,基督將作為救主再來。人那時抬頭仰望就能看到因為這世界的罪而被釘死的主,他們就會通過神的使者所傳的道認識基督。

 到時還會有有些小的叛亂發生,它們還會變為大叛亂。但他們只是證明了任何背叛神的權柄的人都會立即受到審判,他的肉體要滅亡,因為基督要馬上對這些叛亂的人進行審判並把他們都殺死。所以那時在大眾中會有這樣的人,他們因為沒有外界的迷惑,因為害怕被審判定罪而在表面上順服神,而實際上心底裡卻存著叛逆的心。叛亂沒有機會會得逞。為了把那些得救的人和沒有得救的人分開,把那些順服基督和叛逆基督的人的分開,無底洞的們要被打開,撒旦要被暫時放出來。他要去做那一千年前被基督阻止了的事情。他會再次去迷惑列國,像從前那樣把自己作為王並作出許諾,如果有人跟隨了他,他就把他們從要順服耶穌基督的責任中解脫出來。那些與基督為敵的,那時就第一次有機會能夠糾結在一起來背叛基督了。他們會群擁到撒旦的面前稱他為他們的王他們的主,相信撒旦能帶領他們把主耶穌基督從王座上趕下來,不讓他們受到審判。

 在10節我們可以看到撒旦最後的命運。「那迷惑他們的魔鬼要被扔在硫磺的火湖裡,就是獸和假先知所在的地方。他們必晝夜受痛苦,直到永永遠遠。」作為第一個叛逆神的天使,撒旦要和所有那些不信神的叛逆者都要被神從眼前趕走,把他們都下在火湖裡,讓他們在那裡永遠地遭受痛苦。

 我們的主曾經幾次教導過我們撒旦的結局。他指出撒旦要被下到硫磺的火湖裡。主還說那個地方是所有失落的人要去的地方。他也講到了在那裡的人要受很多的痛苦,還強調說那些與神腓利受到了屬神的審判的人永遠地都會那樣失落下去。我們平常所想象的那位溫柔的主耶穌基督講地最多的就是那永遠的責罰,那責罰顯明瞭撒旦的結局,也警告了人們不要也進入到那樣的結局中去。

 許多人對火湖都有所懷疑。當我們講到火的時候,我們想到的是燃燒,能燃燒的物質在火中化為了灰燼。所以有人就可能懷疑在這個世界上有什麼東西能讓那火永遠地燃燒下去而不被消耗殆盡。有人也就因此去計算這個世界的物質如果都燃燒的話會燒多長時間,並作出結論說當所有的物質都消耗掉了之後,地獄的火就要熄滅,那時永遠的審判就結束了。

 有一種天體叫做矮白星,天文學家對它了解得很少。這些天體是由於壓力的原因體積被壓縮得很小,大拇指大小的矮白星重量就可以達到幾噸。在這中壓縮的過程裡會釋放出大量的熱。這種壓縮還會產生膨脹,壓縮地越厲害膨脹地就越厲害,產生的熱量也越大,而膨脹而產生的熱量又會維持並加強壓縮的過程。所以這種矮白星由於這種週而復始的轉換永遠也不會冷卻下來。天文學家說這種矮白星永遠維持著這種狀態,因為壓力和氣體都被轉化成液體,所有的物質都一液態的形式存在而不會有所改變。那位天文學家告訴我們根據他們的計算,矮新星跟那永遠不滅的地獄裡的火湖很相像。

 我們不知道撒旦和跟隨他的天使們是否會真的永遠被關押在這樣的天體上,但那位天文學家的見證的確向我們證明瞭神嘴裡出來的一切話都是真理。很難想象在火湖裡的天使會不想起他們在第一次聽了路西非爾的迷惑餓跟他一同背叛神之前他們擁有的特權。在痛苦中他們也一定會喊著說:「噢,要是我當初沒有聽從那些誘惑,沒有跟路西非爾一同來背叛神的話,我就不會犯罪,現在我就要在神的面前服侍神了。」即使是在地獄裡,他們也會承認耶穌基督是主,是拯救者,他有權柄要人聽從他的話,他們不會在去承認撒旦的權柄。不過,那時已經太晚了。

 我們的主曾警告我們說會有人來跟從撒旦一起來到那為撒旦和跟從他的天使所準備的地方。這個世界本來是為人的居住而造的。火湖是為撒旦在神造人之前和跟隨他一起叛逆神的天使而造的。當亞當背叛神的時候,他的命運就改變了;他不能在來到神的面前,而要與撒旦一起面對被毀滅的命運。而無比仁慈的神又為亞當準備了祭,要讓亞當所有的子孫都因為耶穌基督的緣故可以不受火湖的煎熬,脫離背棄神的責罰。神借耶穌基督的死赦免了罪;他還提供了一個新家(天父的家);他給了人新的命運(和神在一起);他使我們能得享神的榮光,而不是火湖裡的熊熊火光。火湖不是為人準備的,但對於那些跟從撒旦的人來說那就是他們的命運。我們完全可以脫離這種命運,因為耶穌基督借自己的死而復活已經把你從罪中拯救出來了。天使沒有這樣的救贖,那些與撒旦一起叛亂的天使,他們的命運就是和撒旦一同下到火湖裡。可仁慈的神卻給了你救贖。你的決定是什麼:去天國還是去火湖?跟從耶穌基督還是撒旦?

 使徒保羅向我們描述了那屬於耶穌基督的榮耀,借此來提醒我們當他順服來到十字架前受死的時候,「神將他升為至高,又賜給他那超乎萬名之上的名,叫一切在天上的,地上的,和地底下的,因耶穌的名,無不屈膝,無不口稱耶穌基督為主,使榮耀歸與父神。(腓立比書2:9-11)」在以後的時代,沒有人會不承認主耶穌基督擁有至高的權柄。諸天都要稱他為榮耀的主和拯救者。天使要敬拜服侍他。即使是那些被驅逐的人也要承認他們本來所愛的所服侍的撒旦其實是個騙子,耶穌基督才是真正的主。

 沒有人會再違背主的旨意,因為諸天都明白他的權柄都要順服他。也不會再有別的主,只有耶穌基督做主直到永遠。所有的榮耀、權柄、愛和讚美都要永遠地給主。

\end{document}
